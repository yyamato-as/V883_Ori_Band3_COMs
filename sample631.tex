%% Beginning of file 'sample631.tex'
%%
%% Modified 2021 March
%%
%% This is a sample manuscript marked up using the
%% AASTeX v6.31 LaTeX 2e macros.
%%
%% AASTeX is now based on Alexey Vikhlinin's emulateapj.cls 
%% (Copyright 2000-2015).  See the classfile for details.

%% AASTeX requires revtex4-1.cls and other external packages such as
%% latexsym, graphicx, amssymb, longtable, and epsf.  Note that as of 
%% Oct 2020, APS now uses revtex4.2e for its journals but remember that 
%% AASTeX v6+ still uses v4.1. All of these external packages should 
%% already be present in the modern TeX distributions but not always.
%% For example, revtex4.1 seems to be missing in the linux version of
%% TexLive 2020. One should be able to get all packages from www.ctan.org.
%% In particular, revtex v4.1 can be found at 
%% https://www.ctan.org/pkg/revtex4-1.

%% The first piece of markup in an AASTeX v6.x document is the \documentclass
%% command. LaTeX will ignore any data that comes before this command. The 
%% documentclass can take an optional argument to modify the output style.
%% The command below calls the preprint style which will produce a tightly 
%% typeset, one-column, single-spaced document.  It is the default and thus
%% does not need to be explicitly stated.
%%
%% using aastex version 6.3
\documentclass[linenumbers, twocolumn, twocolappendix, astrosymb, times]{aastex631}

%% The default is a single spaced, 10 point font, single spaced article.
%% There are 5 other style options available via an optional argument. They
%% can be invoked like this:
%%
%% \documentclass[arguments]{aastex631}
%% 
%% where the layout options are:
%%
%%  twocolumn   : two text columns, 10 point font, single spaced article.
%%                This is the most compact and represent the final published
%%                derived PDF copy of the accepted manuscript from the publisher
%%  manuscript  : one text column, 12 point font, double spaced article.
%%  preprint    : one text column, 12 point font, single spaced article.  
%%  preprint2   : two text columns, 12 point font, single spaced article.
%%  modern      : a stylish, single text column, 12 point font, article with
%% 		  wider left and right margins. This uses the Daniel
%% 		  Foreman-Mackey and David Hogg design.
%%  RNAAS       : Supresses an abstract. Originally for RNAAS manuscripts 
%%                but now that abstracts are required this is obsolete for
%%                AAS Journals. Authors might need it for other reasons. DO NOT
%%                use \begin{abstract} and \end{abstract} with this style.
%%
%% Note that you can submit to the AAS Journals in any of these 6 styles.
%%
%% There are other optional arguments one can invoke to allow other stylistic
%% actions. The available options are:
%%
%%   astrosymb    : Loads Astrosymb font and define \astrocommands. 
%%   tighten      : Makes baselineskip slightly smaller, only works with 
%%                  the twocolumn substyle.
%%   times        : uses times font instead of the default
%%   linenumbers  : turn on lineno package.
%%   trackchanges : required to see the revision mark up and print its output
%%   longauthor   : Do not use the more compressed footnote style (default) for 
%%                  the author/collaboration/affiliations. Instead print all
%%                  affiliation information after each name. Creates a much 
%%                  longer author list but may be desirable for short 
%%                  author papers.
%% twocolappendix : make 2 column appendix.
%%   anonymous    : Do not show the authors, affiliations and acknowledgments 
%%                  for dual anonymous review.
%%
%% these can be used in any combination, e.g.
%%
%% \documentclass[twocolumn,linenumbers,trackchanges]{aastex631}
%%
%% AASTeX v6.* now includes \hyperref support. While we have built in specific
%% defaults into the classfile you can manually override them with the
%% \hypersetup command. For example,
%%
%% \hypersetup{linkcolor=red,citecolor=green,filecolor=cyan,urlcolor=magenta}
%%
%% will change the color of the internal links to red, the links to the
%% bibliography to green, the file links to cyan, and the external links to
%% magenta. Additional information on \hyperref options can be found here:
%% https://www.tug.org/applications/hyperref/manual.html#x1-40003
%%
%% Note that in v6.3 "bookmarks" has been changed to "true" in hyperref
%% to improve the accessibility of the compiled pdf file.
%%
%% If you want to create your own macros, you can do so
%% using \newcommand. Your macros should appear before
%% the \begin{document} command.
%%
\usepackage{xspace}
\usepackage{txfonts}
\newcommand{\vdag}{(v)^\dagger}
\newcommand\aastex{AAS\TeX}
\newcommand\latex{La\TeX}
\newcommand{\hcotp}{HCO$_2^+$}
\newcommand{\methanol}{CH$_3$OH\xspace}
\newcommand{\acetaldehyde}{CH$_3$CHO\xspace}
\newcommand{\methylformate}{CH$_3$OCHO\xspace}
\newcommand{\dimethylether}{CH$_3$OCH$_3$\xspace}
\newcommand{\acetone}{CH$_3$COCH$_3$\xspace}
\newcommand{\ethyleneoxide}{$c$-C$_2$H$_4$O\xspace}
\newcommand{\propenal}{$t$-C$_2$H$_3$CHO\xspace}
\newcommand{\propanal}{$s$-C$_2$H$_5$CHO\xspace}

%% Reintroduced the \received and \accepted commands from AASTeX v5.2
%\received{March 1, 2021}
%\revised{April 1, 2021}
%\accepted{\today}

%% Command to document which AAS Journal the manuscript was submitted to.
%% Adds "Submitted to " the argument.
%\submitjournal{PSJ}

%% For manuscript that include authors in collaborations, AASTeX v6.31
%% builds on the \collaboration command to allow greater freedom to 
%% keep the traditional author+affiliation information but only show
%% subsets. The \collaboration command now must appear AFTER the group
%% of authors in the collaboration and it takes TWO arguments. The last
%% is still the collaboration identifier. The text given in this
%% argument is what will be shown in the manuscript. The first argument
%% is the number of author above the \collaboration command to show with
%% the collaboration text. If there are authors that are not part of any
%% collaboration the \nocollaboration command is used. This command takes
%% one argument which is also the number of authors above to show. A
%% dashed line is shown to indicate no collaboration. This example manuscript
%% shows how these commands work to display specific set of authors 
%% on the front page.
%%
%% For manuscript without any need to use \collaboration the 
%% \AuthorCollaborationLimit command from v6.2 can still be used to 
%% show a subset of authors.
%
%\AuthorCollaborationLimit=2
%
%% will only show Schwarz & Muench on the front page of the manuscript
%% (assuming the \collaboration and \nocollaboration commands are
%% commented out).
%%
%% Note that all of the author will be shown in the published article.
%% This feature is meant to be used prior to acceptance to make the
%% front end of a long author article more manageable. Please do not use
%% this functionality for manuscripts with less than 20 authors. Conversely,
%% please do use this when the number of authors exceeds 40.
%%
%% Use \allauthors at the manuscript end to show the full author list.
%% This command should only be used with \AuthorCollaborationLimit is used.

%% The following command can be used to set the latex table counters.  It
%% is needed in this document because it uses a mix of latex tabular and
%% AASTeX deluxetables.  In general it should not be needed.
%\setcounter{table}{1}

%%%%%%%%%%%%%%%%%%%%%%%%%%%%%%%%%%%%%%%%%%%%%%%%%%%%%%%%%%%%%%%%%%%%%%%%%%%%%%%%
%%
%% The following section outlines numerous optional output that
%% can be displayed in the front matter or as running meta-data.
%%
%% If you wish, you may supply running head information, although
%% this information may be modified by the editorial offices.
\shorttitle{V883 Ori Band 3}
\shortauthors{Yamato et al.}
%%
%% You can add a light gray and diagonal water-mark to the first page 
%% with this command:
%% \watermark{text}
%% where "text", e.g. DRAFT, is the text to appear.  If the text is 
%% long you can control the water-mark size with:
%% \setwatermarkfontsize{dimension}
%% where dimension is any recognized LaTeX dimension, e.g. pt, in, etc.
%%
%%%%%%%%%%%%%%%%%%%%%%%%%%%%%%%%%%%%%%%%%%%%%%%%%%%%%%%%%%%%%%%%%%%%%%%%%%%%%%%%
\graphicspath{{./}{figures/}}
%% This is the end of the preamble.  Indicate the beginning of the
%% manuscript itself with \begin{document}.

\begin{document}

\title{ALMA Band 3 Observations of the V883 Ori disk}

%% LaTeX will automatically break titles if they run longer than
%% one line. However, you may use \\ to force a line break if
%% you desire. In v6.31 you can include a footnote in the title.

%% A significant change from earlier AASTEX versions is in the structure for 
%% calling author and affiliations. The change was necessary to implement 
%% auto-indexing of affiliations which prior was a manual process that could 
%% easily be tedious in large author manuscripts.
%%
%% The \author command is the same as before except it now takes an optional
%% argument which is the 16 digit ORCID. The syntax is:
%% \author[xxxx-xxxx-xxxx-xxxx]{Author Name}
%%
%% This will hyperlink the author name to the author's ORCID page. Note that
%% during compilation, LaTeX will do some limited checking of the format of
%% the ID to make sure it is valid. If the "orcid-ID.png" image file is 
%% present or in the LaTeX pathway, the OrcID icon will appear next to
%% the authors name.
%%
%% Use \affiliation for affiliation information. The old \affil is now aliased
%% to \affiliation. AASTeX v6.31 will automatically index these in the header.
%% When a duplicate is found its index will be the same as its previous entry.
%%
%% Note that \altaffilmark and \altaffiltext have been removed and thus 
%% can not be used to document secondary affiliations. If they are used latex
%% will issue a specific error message and quit. Please use multiple 
%% \affiliation calls for to document more than one affiliation.
%%
%% The new \altaffiliation can be used to indicate some secondary information
%% such as fellowships. This command produces a non-numeric footnote that is
%% set away from the numeric \affiliation footnotes.  NOTE that if an
%% \altaffiliation command is used it must come BEFORE the \affiliation call,
%% right after the \author command, in order to place the footnotes in
%% the proper location.
%%
%% Use \email to set provide email addresses. Each \email will appear on its
%% own line so you can put multiple email address in one \email call. A new
%% \correspondingauthor command is available in V6.31 to identify the
%% corresponding author of the manuscript. It is the author's responsibility
%% to make sure this name is also in the author list.
%%
%% While authors can be grouped inside the same \author and \affiliation
%% commands it is better to have a single author for each. This allows for
%% one to exploit all the new benefits and should make book-keeping easier.
%%
%% If done correctly the peer review system will be able to
%% automatically put the author and affiliation information from the manuscript
%% and save the corresponding author the trouble of entering it by hand.

%\correspondingauthor{August Muench}
%\email{greg.schwarz@aas.org, gus.muench@aas.org}

\author[0000-0003-4099-6941]{Yoshihide Yamato}
\affiliation{Department of Astronomy, Graduate School of Science, The University of Tokyo, 7-3-1 Hongo, Bunkyo, Tokyo 113-0033, Japan}

%% Note that the \and command from previous versions of AASTeX is now
%% depreciated in this version as it is no longer necessary. AASTeX 
%% automatically takes care of all commas and "and"s between authors names.

%% AASTeX 6.31 has the new \collaboration and \nocollaboration commands to
%% provide the collaboration status of a group of authors. These commands 
%% can be used either before or after the list of corresponding authors. The
%% argument for \collaboration is the collaboration identifier. Authors are
%% encouraged to surround collaboration identifiers with ()s. The 
%% \nocollaboration command takes no argument and exists to indicate that
%% the nearby authors are not part of surrounding collaborations.

%% Mark off the abstract in the ``abstract'' environment. 
\begin{abstract}

% \textcolor{red}{abstract here.} 250 word limit

\end{abstract}

%% Keywords should appear after the \end{abstract} command. 
%% The AAS Journals now uses Unified Astronomy Thesaurus concepts:
%% https://astrothesaurus.org
%% You will be asked to selected these concepts during the submission process
%% but this old "keyword" functionality is maintained in case authors want
%% to include these concepts in their preprints.
%% \keywords{Classical Novae (251) --- Ultraviolet astronomy(1736) --- History of astronomy(1868) --- Interdisciplinary astronomy(804)}

%% From the front matter, we move on to the body of the paper.
%% Sections are demarcated by \section and \subsection, respectively.
%% Observe the use of the LaTeX \label
%% command after the \subsection to give a symbolic KEY to the
%% subsection for cross-referencing in a \ref command.
%% You can use LaTeX's \ref and \label commands to keep track of
%% cross-references to sections, equations, tables, and figures.
%% That way, if you change the order of any elements, LaTeX will
%% automatically renumber them.
%%
%% We recommend that authors also use the natbib \citep
%% and \citet commands to identify citations.  The citations are
%% tied to the reference list via symbolic KEYs. The KEY corresponds
%% to the KEY in the \bibitem in the reference list below. 

\section{Introduction} \label{sec:intro}
\begin{itemize}
    \item COMs are key to understanding the molecular evolution from the ISM to planetary materials...
    \item Young bursting star like V883 Ori provide a unique opportunity to explore the COMs chemistry in young protoplanetary disks
    \item Many COMs lines have been detected in Band 5, 6, and 7 observations.
    \item However, its column density and abundance ratio estimation is suffered from the dust opacity effect in higher band of ALMA
    \item we report the ALMA Band 3 observations, where the dust opacity effect should be minimized, and evaluate the column density of COMs correctly
\end{itemize}

\section{Observation and data reduction} \label{sec:observation}

\subsection{ALMA Band 3 Observations}
V883 Ori was observed in Band 3 during ALMA Cycle 8 (project code: 2021.1.00357.S, PI: S. Notsu). Observations were performed in a total of four executions, one of which was with an extended antenna configuration and other three of which are with a compact antenna configuration. The observation dates, number of antennas, on-source integration time, precipitable water vapor (PWV), baseline coverage, angular resolution, Maximum recoverable scale (MRS), and calibrator information are listed in Table \ref{tab:observations}. 

The correlator setup included eleven spectral windows (SPWs) in Frequency Division Mode (FDM), one of which was dedicated to continuum acquisition with a wide bandwidth of 937.5\,MHz for an accurate determination of the continuum level. The frequency resolution of the continuum SPW was 0.488\,MHz, resulting in a velocity resolution of $\sim$1.5\,km \,s$^{-1}$. The continuum SPW included many spectral lines detected as well, which are analyzed in this work. Other ten SPWs targeted specific spectral lines with narrower bandwidths of 58.59\,MHz or 117.19\,MHz, but also covered many COMs lines. The frequency resolution of these SPWs was 0.141\,MHz, resulting in a velocity resolution of $\sim$0.4--0.5\,km\,s$^{-1}$. 
% For all SPWs, the spectral averaging factor was set to 1, and thus the native channel width was half the spectral resolution. 
The detailed properties of the SPWs are summarized in Table \ref{tab:cube_properties}. We note that two SPWs (8 and 9) were partially overlapping each other, and only the wider SPW 9 is used for analysis. We also note that the primary target of this observation was two protonated carbon dioxide (\hcotp) lines, which will be presented in future work (Notsu et al. in prep.). This paper focuses on a large number of COMs lines covered with these SPWs.


\subsection{Calibration and Imaging}
The initial calibrations were performed by the ALMA staff using the standard ALMA calibration pipeline version 2021.2.0.128. Subsequent self-calibration and imaging were conducted using the Common Astronomy Software Applications \citep[CASA;][]{CASA} version 6.2.1.7. The pipeline-calibrated data were first imaged without any deconvolutions (i.e., dirty imaging) or continuum-subtractions to specify the line-free channels precisely by visually inspecting the dirty image cubes. These line-free channels are averaged to obtain continuum visibilities. Self-calibration was subsequently conducted using the continuum visibilities with the CASA tasks \texttt{gaincal} and \texttt{applycal}. One round of the phase-only and phase plus amplitude self-calibration were first performed on the data with a compact antenna configuration. The data with extended antenna configurations were then concatenated and self-calibrated together. Two rounds of the phase-only self-calibration and one round of the phase plus amplitude self-calibration were performed on the combined data. The solutions were then applied to the spectral line visibilities.
% , which were then continuum-subtracted using the CASA task \texttt{uvcontsub}. 

Each of the SPWs was imaged with the CASA task \texttt{tclean} \citep{Hogbom1974} using Briggs weighting (\texttt{robust} $=$ 0.5). The \texttt{tclean} task was run in parallel mode using \texttt{mpicasa} implementation. As a specific weighting scheme, \texttt{briggsbwtaper} is adopted with independent weight densities for each channel (i.e., \texttt{perchanweightdensity = True}) to achieve more uniform sensitivity over whole channels. For each SPW, a common restoring beam across all channels is used. A few channels at the edge of the SPW are removed during the imaging process because of the large deviation of the beam size caused by a slight difference in the frequency coverage between the data with compact and extended antenna configurations\footnote{see \url{https://casadocs.readthedocs.io/en/stable/notebooks/memo-series.html\#Correcting-bad-common-beam}.}. All SPWs were CLEANed down to 3\,$\times$\,RMS with the native channel widths. Automatic masking algorithm with \texttt{automultithresh} was also adopted to generate the CLEAN mask. The typical beam size and the RMS noise level were $\sim0\farcs3$--$0\farcs4$ and 0.6--1\,mJy\,beam$^{-1}$, respectively. The RMS noise level was measured on the images without the primary beam correction. The beam sizes and RMS noise levels for each SPW are listed in Table \ref{tab:cube_properties}. Finally, the continuum component is subtracted from these image cubes using the CASA task \texttt{imcontsub} to produce the spectral-line-only image cubes. The continuum subtraction on the image plane rather than the visibility plane (e.g., with the CASA task \texttt{uvcontsub}) mitigates the Jorsater \& van Moorsel (JvM) effect \citep{JvM, Czekala2021} on the line emission, which is critical for the flux scale, as much as possible (see Appendix \ref{}). Throughout this paper, we use ...

% \subsection{JvM Correction}\label{subsec:JvM_correction}
% The restored images with the default CLEANing process were further processed to account for the Jorsater \& van Moorsel \citep[][hereafter JvM]{JvM} effect: the JvM correction. Due to the inconsistency between the units of the residual image (in Jy per dirty beam) and the model image (in Jy per CLEAN beam), the flux scale in the CLEANed image can be incorrect, particularly for the faint emission. We followed the approach described in \citet{Czekala2021}, where the correction is made by rescaling the residual image by the ratio of the CLEAN beam and dirty beam volumes ($\epsilon$). The JvM $\epsilon$ is $\approx$ 0.29 for all SPWs as listed in Table \ref{tab:cube_properties}. The small $\epsilon$ values indicate that the effect is severe, as expected for the dataset combined with different antenna configurations \citep{Czekala2021}. While the JvM correction recovers the correct flux scale on the CLEANed image, which is critical for determining molecular column densities and temperatures, \citet{Casassus2022} cautioned that the JvM correction may exaggerate the S/N of the emission by artificially reducing the noise level. We thus show the spectra extracted from the JvM-uncorrected images as well in Appendix \ref{}, to ensure that the detections of molecular line emission are statistically significant. 
% After the JvM correction, the primary beam correction was also applied to the JvM-corrected image cubes. Throughout this paper, we use the JvM-corrected, primary-beam-corrected images, unless otherwise stated. 






\begin{deluxetable*}{ccccccccc}
\label{tab:observations}
\tablecaption{Observational Details}
\tablehead{\colhead{Date} & \colhead{\# of Ant.} & \colhead{On-source Int.} & \colhead{PWV}  & \colhead{Baseline} & \colhead{Ang. Res.} & \colhead{MRS} & \colhead{Bandpass/Amplitude Cal.} & \colhead{Phase Cal.} \\
\colhead{} & \colhead{} & \colhead{(min)} & \colhead{(mm)} & \colhead{(m)} & \colhead{($\arcsec$)} & \colhead{($\arcsec$)} & \colhead{} & \colhead{}}
\startdata
2021 Nov. 21 & 44 & 43 & 2.2 & 41--3396 & 0.3 & 5.2 & J0423-0120 & J0541-0541 \\
2021 Nov. 21 & 44 & 43 & 2.0 & 41--3396 & 0.3 & 5.2 & J0538-4405 & J0541-0541 \\
2021 Nov. 22 & 43 & 43 & 3.8 & 41--3396 & 0.3 & 4.7 & J0423-0120 & J0541-0541 \\
2022 Jan. 20 & 41 & 47 & 3.6 & 14--740  & 1.5 & 17.4 & J0423-0120 & J0541-0541
\enddata
\end{deluxetable*}

% \begin{deluxetable*}{cccccccccc}
% \label{tab:cube_properties}
% \tablecaption{Properties of Image Cubes}
% \tablehead{\colhead{SPW} & \colhead{Cent. Freq.} & \colhead{\# of Chan.} & \colhead{Bandwidth} & \multicolumn{2}{c}{Channel Width}  & \colhead{Vel. Res.} & \colhead{Beam Size (P.A.)} & \colhead{RMS} & \colhead{JvM $\epsilon$} \\
% \colhead{} & \colhead{(GHz)} & \colhead{} & \colhead{(MHz)} & \colhead{(MHz)} & \colhead{(km\,s$^{-1}$)} & \colhead{(km\,s$^{-1}$)} & \colhead{} & \colhead{(mJy\,beam$^{-1}$}) & \colhead{}}
% \startdata
% 0 & 85.160960 & 478 & 58.229 & 0.122 & 0.43 & 0.43 & 0$\farcs$42$\times$0$\farcs$31 (-75$\arcdeg$) & 1.1 & 0.29 \\
% 1 & 85.530202 & 478 & 58.229 & 0.122 & 0.43 & 0.43 & 0$\farcs$42$\times$0$\farcs$31 (-75$\arcdeg$) & 1.0 & 0.29 \\
% 2 & 85.851713 & 478 & 58.229 & 0.122 & 0.43 & 0.43 & 0$\farcs$42$\times$0$\farcs$31 (-75$\arcdeg$) & 0.99 & 0.28 \\
% 3 & 85.924957 & 478 & 58.229 & 0.122 & 0.43 & 0.43 & 0$\farcs$41$\times$0$\farcs$31 (-75$\arcdeg$) & 0.99 & 0.28 \\
% 4 & 86.669481 & 478 & 58.229 & 0.122 & 0.42 & 0.42 & 0$\farcs$41$\times$0$\farcs$31 (-75$\arcdeg$) & 0.94 & 0.29 \\
% 5 & 86.752980 & 958 & 116.824 & 0.122 & 0.42 & 0.42 & 0$\farcs$41$\times$0$\farcs$31 (-75$\arcdeg$) & 0.94 & 0.29 \\
% 6 & 87.327244 & 478 & 58.229 & 0.122 & 0.42 & 0.42 & 0$\farcs$41$\times$0$\farcs$31 (-75$\arcdeg$) & 0.98 & 0.29 \\
% 7 & 96.490854 & 478 & 58.229 & 0.122 & 0.38 & 0.38 & 0$\farcs$37$\times$0$\farcs$28 (-75$\arcdeg$) & 1.0 & 0.28 \\
% 8 & 96.743210 & 478 & 58.229 & 0.122 & 0.38 & 0.38 & 0$\farcs$37$\times$0$\farcs$27 (-75$\arcdeg$) & 0.99 & 0.30 \\
% 9 & 96.754197 & 958 & 116.824 & 0.122 & 0.38 & 0.37 & 0$\farcs$37$\times$0$\farcs$27 (-75$\arcdeg$) & 0.99 & 0.30 \\
% 10 & 97.979676 & 3839 & 937.036 & 0.244 & 0.75 & 1.5 & 0$\farcs$36$\times$0$\farcs$26 (-73$\arcdeg$) & 0.59 & 0.33
% \enddata
% \end{deluxetable*}

\begin{deluxetable*}{ccccccccc}
\label{tab:cube_properties}
\tablecaption{Properties of Image Cubes}
\tablehead{\colhead{SPW} & \colhead{Cent. Freq.} & \colhead{\# of Chan.} & \colhead{Bandwidth} & \multicolumn{2}{c}{Channel Width}  & \colhead{Vel. Res.} & \colhead{Beam Size (P.A.)} & \colhead{RMS} \\
\colhead{} & \colhead{(GHz)} & \colhead{} & \colhead{(MHz)} & \colhead{(MHz)} & \colhead{(km\,s$^{-1}$)} & \colhead{(km\,s$^{-1}$)} & \colhead{} & \colhead{(mJy\,beam$^{-1}$})}
\startdata
0 & 85.160960 & 478 & 58.229 & 0.122 & 0.43 & 0.43 & 0$\farcs$42$\times$0$\farcs$31 (-75$\arcdeg$) & 1.1 \\
1 & 85.530202 & 478 & 58.229 & 0.122 & 0.43 & 0.43 & 0$\farcs$42$\times$0$\farcs$31 (-75$\arcdeg$) & 1.0 \\
2 & 85.851713 & 478 & 58.229 & 0.122 & 0.43 & 0.43 & 0$\farcs$42$\times$0$\farcs$31 (-75$\arcdeg$) & 0.99 \\
3 & 85.924957 & 478 & 58.229 & 0.122 & 0.43 & 0.43 & 0$\farcs$41$\times$0$\farcs$31 (-75$\arcdeg$) & 0.99 \\
4 & 86.669481 & 478 & 58.229 & 0.122 & 0.42 & 0.42 & 0$\farcs$41$\times$0$\farcs$31 (-75$\arcdeg$) & 0.94 \\
5 & 86.752980 & 958 & 116.824 & 0.122 & 0.42 & 0.42 & 0$\farcs$41$\times$0$\farcs$31 (-75$\arcdeg$) & 0.94 \\
6 & 87.327244 & 478 & 58.229 & 0.122 & 0.42 & 0.42 & 0$\farcs$41$\times$0$\farcs$31 (-75$\arcdeg$) & 0.98 \\
7 & 96.490854 & 478 & 58.229 & 0.122 & 0.38 & 0.38 & 0$\farcs$37$\times$0$\farcs$28 (-75$\arcdeg$) & 1.0 \\
8 & 96.743210 & 478 & 58.229 & 0.122 & 0.38 & 0.38 & 0$\farcs$37$\times$0$\farcs$27 (-75$\arcdeg$) & 0.99 \\
9 & 96.754197 & 958 & 116.824 & 0.122 & 0.38 & 0.37 & 0$\farcs$37$\times$0$\farcs$27 (-75$\arcdeg$) & 0.99 \\
10 & 97.979676 & 3839 & 937.036 & 0.244 & 0.75 & 1.5 & 0$\farcs$36$\times$0$\farcs$26 (-73$\arcdeg$) & 0.59
\enddata
\end{deluxetable*}


\section{Data Analysis and Results}
\subsection{Spectrum Extraction and Line Identification}\label{subsec:line_identification}
Figure \ref{fig:spectra_gallery} shows the spectra which are integrated over 1\farcs2 aperture and corrected for the line broadening due to the Keplerian rotation of the disk. The original disk-integrated spectra are shown in Figure \ref{} in Appendix \ref{}. To obtain the spectra corrected for the Keplerian line broadening, we used \texttt{integrated\_spectrum()} function implemented in \texttt{GoFish} package \citep{GoFish}. This method first deprojects the disk and then aligns the Dopper-shifted spectra at each position within the disk to a common velocity (systemic velocity of the source) to recover the single-peak spectra for each transitions, facilitating the identification of the blended transitions. In this procedure, we assumed a position angle (32\arcdeg) and inclination angle (38.3\arcdeg) of the disk, a central stellar mass of $1.29\,M_\odot$, and a distance of 400\,pc, based on previous works \citep{Cieza2016, Tobin2023}. We also assumed that the emission is originating from the midplane of the disk and did not consider the vertical extent of the emission. The uncertainties of the spectra were calculated per channel basis within the \texttt{integrated\_spectrum()} function taking the spectral decorrelation \citep{Yen2016} into account.

As shown in Figure \ref{fig:spectra_gallery}, numerous spectral features are detected in the disk-integrated spectra. As many COMs transitions exists in narrow frequency ranges and they are usually overlapped between different species, a visual identification of the lines is challenging. Alternatively, we fitted a synthetic spectra to the observed spectra to identify each of these features. The details of the synthetic spectral model is described in Appendix \ref{}. Briefly, the model assumed a common emitting region size, a common excitation temperature, and a common line width under the local thermodynamic equilibrium (LTE) conditions, and the emitting region size, excitation temperature, line width, and the column density of each species are varied to fit the observed spectra and test if they can be reproduced with the model. For the construction of the model, the spectroscopic data of the molecular lines are taken from the Jet Propulsion Spectroscopy \citep[JPL;][]{JPL} and the Cologne Database for Molecular Spectroscopy \citep[CDMS;][]{CDMS1, CDMS2} as detailed in Appendix \ref{}. Only the transitions with an Einstein A coefficient for spontaneous emission of $\geq10^{-8}$\,s$^{-1}$ and an upper state energy of $\leq1000$\,K are considered in the model.

The identified transitions are listed in Table \ref{} in Appendix \ref{}. In summary, we firmly identified 7 COM species: \methanol, \methylformate, \dimethylether, \acetaldehyde, \acetone, \ethyleneoxide, and \propenal. We also tentatively identified 8 COM species and isotopologues: \propanal, C$_2$H$_3$CN, CH$_2$DOH, CH$_3^{18}$OH, CH$_3$O$^{13}$CHO, CH$_3^{13}$CHO, CH$_2$DCHO, and CH$_3$CDO. The identification of species is based on the number of transitions detected; if multiple transitions of the species are detected at $>3\sigma$ in the stacked spectra, we regard that species as a firm identification. If only one transition is detected at $>3\sigma$ in the stacked spectra, it is considered as a tentative identification. In addition, one transition of each sulfur-bearing molecule (OCS and SO$_2$) was detected.

\begin{figure*}
\epsscale{1.15}
\plotone{V883_Ori_stacked_spectra_model.pdf}
\caption{Spectra corrected for Keplerian rotation toward the V883 Ori disk (grey). The model spectra for each species are shown in colored lines. }
\label{fig:spectra_gallery}
\end{figure*}

\begin{figure*}
\epsscale{1.15}
\plotone{V883_Ori_mom0_gallery.pdf}
\caption{Velocity-integrated intensity maps of the molecular line emission in the V883 Ori disk. The velocity range for integration are $\pm3.5$ km s$^{-1}$ with respect to the source systemic velocity $v_\mathrm{sys}=4.25$ km s$^{-1}$ \citep{Tobin2023} including \dimethylether and \acetone, where multiple blended transitions are integrated together. The molecular species, transitions, and upper state energy levels are indicated in the upper-left corner of each panel. The black contours start from 5$\sigma$ with steps of 2.5$\sigma$, where $\sigma$ are the noise level of each map measured on the emission-free region. The synthesized beam and a scale bar of 80 au are shown in the lower left and right corner of each panel, respectively.}
\label{fig:moment_zero_gallery}
\end{figure*}


% The uncertainties of the original spectra were calculated in each channel as $\sqrt{\mathrm{\Omega_\mathrm{aperture}/\Omega_\mathrm{beam}}} \times \sigma$, where $\Omega_\mathrm{aperture}$, $\Omega_\mathrm{beam}$ and $\sigma$ are the aperture area, the beam area, and the image RMS, respectively. 

% Figure \ref{fig}(a) shows the continuum image of the V883 Ori disk at $\sim$ 3\,mm. The continuum emission was imaged with a Briggs weighting (robust $=0.5$). The beam size and RMS noise level of the continuum image were 0.4 x 0.3 and xx mJy beam-1, respectively. The continuum emission associated with the V883 Ori disk is marginally spatially resolved, and its extent is in agreement with the previous observations with higher angular resolution \citep{Cieza2016}. 

% Figure \ref{fig}(b)--(d) shows the velocity-integrated intensity maps (contour) and the velocity centroid maps (color) of three representative molecular lines (\methanol, \acetaldehyde, \methylformate).

% Figure \ref{fig:} shows the disk-averaged spectra toward the V883 Ori disk. These spectra were extracted from the central   

\subsection{Spatial Distribution}
Figure \ref{fig:moment_zero_gallery} shows the velocity-integrated intensity map (zeroth moment map) of the selected transitions without significant blending with other species. The molecular line emission are all confined to the disk region and marginally spatially resolved. While the spatial extent of the emission is consistent with the previous observations in Bands 6 and 7 \citep{vantHoff2018, Lee2019, Tobin2023} considering the difference in the beam size, the central emission cavities seen in the Band 6 and 7 data are not detected in the present Band 3 data. 


\subsection{Column Density Retrieval}
\subsubsection{Spectral fit}
To derive the disk-averaged column density of each species, we first employ a simultaneous fit of an LTE spectral model of different species to the extracted spectra, taking the spectral blending and line optical depth into account. The same model as that used for line identification is used (see Appendix \ref{} for details), where all species (tentatively) identified as in Section \ref{subsec:line_identification} are considered.  
\subsubsection{Rotation diagram}








\section{Discussion} \label{sec:discussion}
\begin{itemize}
    \item comparison with Band 6/7 data
    \item comparison with protostars/comets
\end{itemize}

\section{Summary} \label{sec:summary}


\begin{acknowledgments}
We thank Tomohiro Yoshida, Kiyoaki Doi, and Gianni Cataldi...
\end{acknowledgments}

%% To help institutions obtain information on the effectiveness of their 
%% telescopes the AAS Journals has created a group of keywords for telescope 
%% facilities.
%
%% Following the acknowledgments section, use the following syntax and the
%% \facility{} or \facilities{} macros to list the keywords of facilities used 
%% in the research for the paper.  Each keyword is check against the master 
%% list during copy editing.  Individual instruments can be provided in 
%% parentheses, after the keyword, but they are not verified.

\vspace{5mm}
\facilities{ALMA}

%% Similar to \facility{}, there is the optional \software command to allow 
%% authors a place to specify which programs were used during the creation of 
%% the manuscript. Authors should list each code and include either a
%% citation or url to the code inside ()s when available.

\software{GoFish}

%% Appendix material should be preceded with a single \appendix command.
%% There should be a \section command for each appendix. Mark appendix
%% subsections with the same markup you use in the main body of the paper.

%% Each Appendix (indicated with \section) will be lettered A, B, C, etc.
%% The equation counter will reset when it encounters the \appendix
%% command and will number appendix equations (A1), (A2), etc. The
%% Figure and Table counter will not reset.

\appendix

\section{Detected transitions}
\startlongtable
\begin{deluxetable*}{ccCCCCc}
\tablecaption{Detected Transitions for Each Species\label{tab:transitions}}
\tablehead{\colhead{Species} & \colhead{Transition} & \colhead{$\nu_0$ (GHz)} & \colhead{log$_{10}A_\mathrm{ul}$ (s$^{-1}$)} & \colhead{$E_\mathrm{up}$ (K)} & \colhead{$g_\mathrm{up}$} & \colhead{SPW}}
\decimals
\startdata
\hline 
\multicolumn{7}{c}{Methanol (\methanol)} \\
\hline 
CH$_3$OH$^\ddagger$ & $2_{-1,2}$ -- $1_{-1,1}$ E $v_t=1$ & 96.492163 & -5.5963 & 298.42 & 20 & 7 \\
CH$_3$OH$^\ddagger$ & $2_{0,2}$ -- $1_{0,1}$ E $v_t=1$ & 96.493551 & -5.4706 & 307.52 & 20 & 7 \\
CH$_3$OH$^\ddagger$ & $2_{1,1}$ -- $1_{1,0}$ E $v_t=1$ & 96.501713 & -5.5967 & 420.32 & 20 & 7 \\
CH$_3$OH & $2_{0,2}$ -- $1_{0,1}$ A $v_t=1$ & 96.513686 & -5.4709 & 430.6 & 20 & 7 \\
CH$_3$OH$^\dagger$ & $2_{1,2}$ -- $1_{1,1}$ E $v_t=0$ & 96.739358 & -5.5923 & 12.54 & 20 & 9 \\
CH$_3$OH$^\dagger$ & $2_{0,2}$ -- $1_{0,1}$ A $v_t=0$ & 96.741371 & -5.4676 & 6.96 & 20 & 9 \\
CH$_3$OH & $2_{0,2}$ -- $1_{0,1}$ E $v_t=0$ & 96.744545 & -5.4676 & 20.09 & 20 & 9 \\
CH$_3$OH & $2_{-1,1}$ -- $1_{-1,0}$ E $v_t=0$ & 96.755501 & -5.581 & 28.01 & 20 & 9 \\
CH$_3$OH & $2_{1,1}$ -- $1_{1,0}$ A $v_t=0$ & 97.582798 & -5.5807 & 21.56 & 20 & 10 \\
CH$_3$OH$^\dagger$ & $21_{6,16}$ -- $22_{5,17}$ A $v_t=0$ & 97.677684 & -5.8404 & 729.28 & 172 & 10 \\
CH$_3$OH$^\dagger$ & $21_{6,15}$ -- $22_{5,18}$ A $v_t=0$ & 97.678803 & -5.8403 & 729.28 & 172 & 10 \\
\hline 
\multicolumn{7}{c}{Methanol (CH$_2$DOH)} \\
\hline 
CH$_2$DOH$^\ddagger$ & $2_{1,1}$ -- $2_{0,2}$ e0 & 86.668751 & -5.3322 & 10.6 & 5 & 5 \\
CH$_2$DOH$^\ddagger$ & $4_{1,3}$ -- $4_{0,4}$ o1 & 97.870192 & -5.3363 & 44.47 & 9 & 10 \\
CH$_2$DOH & $4_{0,4}$ -- $3_{1,3}$ e0 & 98.031213 & -5.616 & 21.45 & 9 & 10 \\
\hline 
\multicolumn{7}{c}{Methyl Formate (\methylformate)} \\
\hline 
CH$_3$OCHO & $7_{6,1}$ -- $6_{6,0}$ E $v_t=1$ & 85.157135 & -5.6248 & 228.15 & 30 & 2 \\
CH$_3$OCHO$^\dagger$ & $7_{5,3}$ -- $6_{5,2}$ A $v_t=1$ & 85.185466 & -5.3602 & 220.9 & 30 & 2 \\
CH$_3$OCHO$^\dagger$ & $7_{5,2}$ -- $6_{5,1}$ A $v_t=1$ & 85.186063 & -5.3602 & 220.9 & 30 & 2 \\
CH$_3$OCHO & $7_{4,3}$ -- $6_{4,2}$ E $v_t=1$ & 85.506219 & -5.2142 & 214.56 & 30 & 1 \\
CH$_3$OCHO & $7_{5,3}$ -- $6_{5,2}$ E $v_t=1$ & 85.55338 & -5.3527 & 220.04 & 30 & 1 \\
CH$_3$OCHO & $7_{6,1}$ -- $6_{6,0}$ E $v_t=0$ & 85.919209 & -5.6138 & 40.44 & 30 & 0 \\
CH$_3$OCHO$^\dagger$ & $7_{6,2}$ -- $6_{6,1}$ E $v_t=0$ & 85.926553 & -5.6137 & 40.42 & 30 & 0 \\
CH$_3$OCHO$^\dagger$ & $7_{6,2}$ -- $6_{6,1}$ A $v_t=0$ & 85.927227 & -5.6136 & 40.42 & 30 & 0 \\
CH$_3$OCHO$^\dagger$ & $7_{6,1}$ -- $6_{6,0}$ A $v_t=0$ & 85.927227 & -5.6136 & 40.42 & 30 & 0 \\
CH$_3$OCHO & $7_{4,4}$ -- $7_{3,5}$ E $v_t=0$ & 96.507882 & -6.1542 & 27.16 & 30 & 7 \\
CH$_3$OCHO & $8_{4,5}$ -- $8_{3,6}$ A $v_t=0$ & 96.709259 & -5.9454 & 31.89 & 34 & 9 \\
CH$_3$OCHO & $7_{4,3}$ -- $7_{3,5}$ E $v_t=0$ & 96.776715 & -6.4523 & 27.17 & 30 & 9 \\
CH$_3$OCHO & $5_{4,2}$ -- $5_{3,3}$ A $v_t=0$ & 96.794121 & -6.1026 & 19.47 & 22 & 9 \\
CH$_3$OCHO & $8_{5,3}$ -- $7_{5,2}$ E $v_t=1$ & 97.577303 & -5.0823 & 225.36 & 34 & 10 \\
CH$_3$OCHO & $8_{3,6}$ -- $7_{3,5}$ A $v_t=1$ & 97.597161 & -4.9359 & 214.94 & 34 & 10 \\
CH$_3$OCHO$^\dagger$ & $8_{7,2}$ -- $7_{7,1}$ E $v_t=1$ & 97.65127 & -5.4962 & 240.82 & 34 & 10 \\
CH$_3$OCHO$^\dagger$ & $10_{4,7}$ -- $10_{3,8}$ E $v_t=0$ & 97.65127 & -5.9015 & 43.17 & 42 & 10 \\
CH$_3$OCHO & $8_{4,5}$ -- $7_{4,4}$ A $v_t=1$ & 97.661401 & -4.9937 & 219.6 & 34 & 10 \\
CH$_3$OCHO$^\ddagger$ & $10_{4,7}$ -- $10_{3,8}$ A $v_t=0$ & 97.69426 & -5.8957 & 43.16 & 42 & 10 \\
CH$_3$OCHO$^\ddagger$ & $12_{1,11}$ -- $12_{0,12}$ A $v_t=1$ & 97.727054 & -6.2029 & 234.75 & 50 & 10 \\
CH$_3$OCHO & $8_{6,3}$ -- $7_{6,2}$ E $v_t=1$ & 97.738738 & -5.2243 & 232.1 & 34 & 10 \\
CH$_3$OCHO & $8_{4,4}$ -- $7_{4,3}$ A $v_t=1$ & 97.752885 & -4.9924 & 219.61 & 34 & 10 \\
CH$_3$OCHO$^\ddagger$ & $8_{4,4}$ -- $8_{3,5}$ E $v_t=1$ & 97.871147 & -5.967 & 219.26 & 34 & 10 \\
CH$_3$OCHO$^\ddagger$ & $10_{4,7}$ -- $10_{3,8}$ A $v_t=1$ & 97.878933 & -5.8854 & 230.76 & 42 & 10 \\
CH$_3$OCHO & $8_{5,4}$ -- $7_{5,3}$ E $v_t=1$ & 97.885663 & -5.0787 & 224.73 & 34 & 10 \\
CH$_3$OCHO & $8_{4,4}$ -- $7_{4,3}$ E $v_t=1$ & 97.897118 & -4.9875 & 219.26 & 34 & 10 \\
CH$_3$OCHO$^\ddagger$ & $21_{4,17}$ -- $21_{3,18}$ A $v_t=1$ & 98.066305 & -5.7833 & 337.99 & 86 & 10 \\
CH$_3$OCHO$^\ddagger$ & $8_{4,5}$ -- $7_{4,4}$ E $v_t=1$ & 98.176293 & -4.9853 & 218.74 & 34 & 10 \\
CH$_3$OCHO & $8_{7,1}$ -- $7_{7,0}$ E $v_t=0$ & 98.182336 & -5.4902 & 53.78 & 34 & 10 \\
CH$_3$OCHO$^\dagger$ & $8_{7,2}$ -- $7_{7,1}$ A $v_t=0$ & 98.190658 & -5.4899 & 53.76 & 34 & 10 \\
CH$_3$OCHO$^\dagger$ & $8_{7,1}$ -- $7_{7,0}$ A $v_t=0$ & 98.190658 & -5.4899 & 53.76 & 34 & 10 \\
CH$_3$OCHO$^\dagger$ & $8_{7,2}$ -- $7_{7,1}$ E $v_t=0$ & 98.19146 & -5.49 & 53.76 & 34 & 10 \\
CH$_3$OCHO & $8_{6,2}$ -- $7_{6,1}$ E $v_t=0$ & 98.270501 & -5.218 & 45.15 & 34 & 10 \\
CH$_3$OCHO$^\dagger$ & $8_{6,3}$ -- $7_{6,2}$ E $v_t=0$ & 98.278921 & -5.2178 & 45.13 & 34 & 10 \\
CH$_3$OCHO$^\dagger$ & $8_{6,3}$ -- $7_{6,2}$ A $v_t=0$ & 98.279762 & -5.2178 & 45.13 & 34 & 10 \\
CH$_3$OCHO$^\dagger$ & $8_{6,2}$ -- $7_{6,1}$ A $v_t=0$ & 98.279762 & -5.2178 & 45.13 & 34 & 10 \\
CH$_3$OCHO$^\dagger$ & $9_{0,9}$ -- $8_{1,8}$ A $v_t=1$ & 98.423165 & -5.701 & 212.64 & 38 & 10 \\
CH$_3$OCHO$^\dagger$ & $8_{5,3}$ -- $7_{5,2}$ E $v_t=0$ & 98.424207 & -5.0722 & 37.86 & 34 & 10 \\
CH$_3$OCHO$^\dagger$ & $8_{5,4}$ -- $7_{5,3}$ E $v_t=0$ & 98.431803 & -5.072 & 37.84 & 34 & 10 \\
CH$_3$OCHO$^\dagger$ & $8_{5,4}$ -- $7_{5,3}$ A $v_t=0$ & 98.43276 & -5.0719 & 37.84 & 34 & 10 \\
CH$_3$OCHO & $8_{5,3}$ -- $7_{5,2}$ A $v_t=0$ & 98.435802 & -5.0719 & 37.84 & 34 & 10 \\
CH$_3$OCHO & $8_{4,5}$ -- $7_{4,3}$ E $v_t=0$ & 98.443186 & -6.4657 & 31.9 & 34 & 10 \\
\hline 
\multicolumn{7}{c}{Methyl Formate (CH$_3$O$^{13}$CHO)} \\
\hline 
CH$_3$O$^{13}$CHO$^\dagger$ & $8_{6,2}$ -- $7_{6,1}$ E $v_t=0$ & 97.541168 & -5.2186 & 44.76 & 34 & 10 \\
CH$_3$O$^{13}$CHO$^\dagger$ & $8_{-6,3}$ -- $7_{-6,2}$ E $v_t=0$ & 97.549352 & -5.2185 & 44.74 & 34 & 10 \\
CH$_3$O$^{13}$CHO$^\dagger$ & $8_{6,3}$ -- $7_{6,2}$ A $v_t=0$ & 97.55014 & -5.2185 & 44.74 & 34 & 10 \\
CH$_3$O$^{13}$CHO$^\dagger$ & $8_{6,2}$ -- $7_{6,1}$ A $v_t=0$ & 97.550183 & -5.2185 & 44.74 & 34 & 10 \\
CH$_3$O$^{13}$CHO$^\ddagger$ & $8_{-5,4}$ -- $7_{-5,3}$ E $v_t=0$ & 97.702479 & -5.0727 & 37.52 & 34 & 10 \\
CH$_3$O$^{13}$CHO$^\ddagger$ & $8_{5,4}$ -- $7_{5,3}$ A $v_t=0$ & 97.703323 & -5.0726 & 37.51 & 34 & 10 \\
CH$_3$O$^{13}$CHO$^\dagger$ & $8_{5,3}$ -- $7_{5,2}$ A $v_t=0$ & 97.706423 & -5.0726 & 37.51 & 34 & 10 \\
CH$_3$O$^{13}$CHO & $8_{-3,6}$ -- $7_{-3,5}$ E $v_t=0$ & 97.874331 & -4.9221 & 27.04 & 34 & 10 \\
CH$_3$O$^{13}$CHO$^\ddagger$ & $8_{3,6}$ -- $7_{3,5}$ A $v_t=0$ & 97.878443 & -4.9215 & 27.03 & 34 & 10 \\
CH$_3$O$^{13}$CHO$^\ddagger$ & $8_{4,4}$ -- $7_{4,3}$ E $v_t=0$ & 98.019644 & -4.9931 & 31.65 & 34 & 10 \\
\hline 
\multicolumn{7}{c}{Acetaldehyde (\acetaldehyde)} \\
\hline 
CH$_3$CHO & $9_{1,8}$ -- $9_{0,9}$ E $v_t=1$ & 85.947624 & -5.559 & 249.81 & 38 & 0 \\
CH$_3$CHO$^\dagger$ & $12_{2,11}$ -- $12_{1,11}$ E $v_t=1$ & 87.303557 & -6.1944 & 285.13 & 50 & 6 \\
CH$_3$CHO$^\dagger$ & $11_{4,7}$ -- $12_{3,9}$ E $v_t=1$ & 87.303714 & -6.2602 & 303.24 & 46 & 6 \\
CH$_3$CHO & $5_{2,3}$ -- $4_{2,2}$ E $v_t=0$ & 96.475524 & -4.6168 & 23.03 & 22 & 7 \\
CH$_3$CHO & $11_{4,8}$ -- $12_{3,10}$ E $v_t=0$ & 96.48895 & -6.1541 & 97.14 & 46 & 7 \\
CH$_3$CHO$^\dagger$ & $5_{3,3}$ -- $4_{3,2}$ A $v_t=2$ & 96.716114 & -4.7374 & 420.58 & 22 & 9 \\
CH$_3$CHO$^\dagger$ & $5_{3,2}$ -- $4_{3,1}$ A $v_t=2$ & 96.717473 & -4.7374 & 420.58 & 22 & 9 \\
CH$_3$CHO$^\dagger$ & $5_{2,3}$ -- $4_{2,2}$ A $v_t=1$ & 96.718409 & -4.6093 & 228.3 & 22 & 9 \\
CH$_3$CHO & $7_{0,7}$ -- $6_{1,6}$ A $v_t=0$ & 96.765371 & -5.5556 & 25.78 & 30 & 9 \\
CH$_3$CHO & $5_{2,4}$ -- $4_{2,3}$ E $v_t=1$ & 96.800291 & -4.613 & 226.45 & 22 & 9 \\
CH$_3$CHO & $5_{3,2}$ -- $4_{3,1}$ E $v_t=2$ & 97.612131 & -4.9808 & 419.01 & 22 & 10 \\
CH$_3$CHO & $19_{2,18}$ -- $18_{3,15}$ E $v_t=0$ & 97.796104 & -6.1248 & 183.87 & 78 & 10 \\
CH$_3$CHO & $10_{4,7}$ -- $11_{3,8}$ A $v_t=1$ & 97.941422 & -6.1583 & 290.44 & 42 & 10 \\
CH$_3$CHO & $21_{3,18}$ -- $20_{4,17}$ A $v_t=0$ & 98.20169 & -6.01 & 235.39 & 86 & 10 \\
CH$_3$CHO & $6_{3,3}$ -- $7_{2,5}$ E $v_t=0$ & 98.368631 & -6.245 & 39.81 & 26 & 10 \\
\hline 
\multicolumn{7}{c}{Acetaldehyde (CH$_3^{13}$CHO)} \\
\hline 
CH$_3^{13}$CHO$^\ddagger$ & $5_{2,3}$ -- $4_{2,2}$ A $v_t=0$ & 96.494465 & -4.5971 & 22.74 & 11 & 7 \\
\hline 
\multicolumn{7}{c}{Acetaldehyde (CH$_3$CDO)} \\
\hline 
CH$_3$CDO & $5_{1,4}$ -- $4_{1,3}$ E $v_t=0$ & 97.81231 & -4.5235 & 15.86 & 11 & 10 \\
CH$_3$CDO & $5_{1,4}$ -- $4_{1,3}$ A $v_t=0$ & 97.828514 & -4.5234 & 15.81 & 11 & 10 \\
\hline 
\multicolumn{7}{c}{Acetaldehyde (CH$_2$DCHO)} \\
\hline 
CH$_2$DCHO & $11_{2,9}i$ -- $11_{1,10}i$ & 98.412832 & -5.2329 & 66.88 & 23 & 10 \\
\hline 
\multicolumn{7}{c}{Dimethyl Ether (\dimethylether)} \\
\hline 
CH$_3$OCH$_3$$^\dagger$ & $16_{3,14}$ -- $15_{4,11}$ AA & 97.990629 & -5.8984 & 136.6 & 330 & 10 \\
CH$_3$OCH$_3$$^\dagger$ & $16_{3,14}$ -- $15_{4,11}$ EE & 97.993382 & -5.8983 & 136.6 & 528 & 10 \\
CH$_3$OCH$_3$$^\dagger$ & $16_{3,14}$ -- $15_{4,11}$ EA & 97.996098 & -5.8984 & 136.6 & 132 & 10 \\
CH$_3$OCH$_3$$^\dagger$ & $16_{3,14}$ -- $15_{4,11}$ AE & 97.996174 & -5.8984 & 136.6 & 198 & 10 \\
\hline 
\multicolumn{7}{c}{Acetone (\acetone)} \\
\hline 
CH$_3$COCH$_3$$^\dagger$ & $17_{6,11}$ -- $17_{5,12}$ AE & 97.929123 & -4.7694 & 110.64 & 210 & 10 \\
CH$_3$COCH$_3$$^\dagger$ & $17_{6,11}$ -- $17_{5,12}$ & 97.929247 & -4.7694 & 110.64 & 140 & 10 \\
CH$_3$COCH$_3$$^\dagger$ & $17_{7,11}$ -- $17_{6,12}$ AE & 97.930235 & -4.7694 & 110.64 & 70 & 10 \\
CH$_3$COCH$_3$$^\dagger$ & $17_{7,11}$ -- $17_{6,12}$ & 97.930344 & -4.7693 & 110.64 & 140 & 10 \\
CH$_3$COCH$_3$$^\dagger$ & $17_{6,11}$ -- $17_{5,12}$ EE & 98.052399 & -4.7674 & 110.59 & 560 & 10 \\
CH$_3$COCH$_3$$^\dagger$ & $17_{7,11}$ -- $17_{6,12}$ EE & 98.053535 & -4.7674 & 110.59 & 560 & 10 \\
\hline 
\multicolumn{7}{c}{Ethylene Oxide (\ethyleneoxide)} \\
\hline 
$c$-C$_2$H$_4$O$^\ddagger$ & $11_{9,3}$ -- $11_{8,4}$ & 96.501033 & -5.0737 & 146.75 & 115 & 7 \\
$c$-C$_2$H$_4$O$^\ddagger$ & $12_{9,4}$ -- $12_{8,5}$ & 97.728742 & -4.9909 & 169.45 & 75 & 10 \\
\hline 
\multicolumn{7}{c}{Propenal (\propenal)} \\
\hline 
$t$-C$_2$H$_3$CHO$^\ddagger$ & $11_{2,10}$ -- $10_{2,9}$ & 97.815592 & -4.3286 & 36.42 & 23 & 10 \\
$t$-C$_2$H$_3$CHO$^\ddagger$ & $11_{8,3}$ -- $10_{8,2}$ & 97.947054 & -4.6391 & 159.91 & 23 & 10 \\
$t$-C$_2$H$_3$CHO$^\ddagger$ & $11_{8,4}$ -- $10_{8,3}$ & 97.947054 & -4.6391 & 159.91 & 23 & 10 \\
$t$-C$_2$H$_3$CHO$^\ddagger$ & $11_{7,5}$ -- $10_{7,4}$ & 97.947549 & -4.5377 & 129.05 & 23 & 10 \\
$t$-C$_2$H$_3$CHO$^\ddagger$ & $11_{7,4}$ -- $10_{7,3}$ & 97.94755 & -4.5377 & 129.05 & 23 & 10 \\
$t$-C$_2$H$_3$CHO$^\ddagger$ & $11_{9,2}$ -- $10_{9,1}$ & 97.948051 & -4.793 & 194.87 & 23 & 10 \\
$t$-C$_2$H$_3$CHO$^\ddagger$ & $11_{9,3}$ -- $10_{9,2}$ & 97.948051 & -4.793 & 194.87 & 23 & 10 \\
$t$-C$_2$H$_3$CHO$^\ddagger$ & $11_{10,1}$ -- $10_{10,0}$ & 97.950017 & -5.0727 & 233.93 & 23 & 10 \\
$t$-C$_2$H$_3$CHO$^\ddagger$ & $11_{10,2}$ -- $10_{10,1}$ & 97.950017 & -5.0727 & 233.93 & 23 & 10 \\
$t$-C$_2$H$_3$CHO$^\ddagger$ & $11_{6,5}$ -- $10_{6,4}$ & 97.950286 & -4.4656 & 102.3 & 23 & 10 \\
$t$-C$_2$H$_3$CHO$^\ddagger$ & $11_{6,6}$ -- $10_{6,5}$ & 97.950286 & -4.4656 & 102.3 & 23 & 10 \\
$t$-C$_2$H$_3$CHO$^\ddagger$ & $11_{5,7}$ -- $10_{5,6}$ & 97.957003 & -4.4126 & 79.67 & 23 & 10 \\
$t$-C$_2$H$_3$CHO$^\ddagger$ & $11_{5,6}$ -- $10_{5,5}$ & 97.957003 & -4.4126 & 79.67 & 23 & 10 \\
$t$-C$_2$H$_3$CHO$^\dagger$ & $11_{4,8}$ -- $10_{4,7}$ & 97.972119 & -4.3735 & 61.15 & 23 & 10 \\
$t$-C$_2$H$_3$CHO$^\dagger$ & $11_{4,7}$ -- $10_{4,6}$ & 97.972119 & -4.3735 & 61.15 & 23 & 10 \\
$t$-C$_2$H$_3$CHO & $11_{3,9}$ -- $10_{3,8}$ & 98.001321 & -4.3451 & 46.74 & 23 & 10 \\
$t$-C$_2$H$_3$CHO$^\ddagger$ & $11_{3,8}$ -- $10_{3,7}$ & 98.019044 & -4.3448 & 46.74 & 23 & 10 \\
\hline 
\multicolumn{7}{c}{Propanal (\propanal)} \\
\hline 
$s$-C$_2$H$_5$CHO$^\ddagger$ & $14_{4,11}$ -- $14_{3,12}$ E & 87.307042 & -5.1587 & 62.19 & 29 & 6 \\
$s$-C$_2$H$_5$CHO$^\ddagger$ & $14_{4,11}$ -- $14_{3,12}$ A & 87.307594 & -5.1587 & 62.19 & 29 & 6 \\
$s$-C$_2$H$_5$CHO$^\ddagger$ & $29_{6,23}$ -- $29_{5,24}$ E & 96.793208 & -4.8841 & 245.73 & 59 & 9 \\
$s$-C$_2$H$_5$CHO$^\ddagger$ & $29_{6,23}$ -- $29_{5,24}$ A & 96.794186 & -4.8841 & 245.73 & 59 & 9 \\
$s$-C$_2$H$_5$CHO$^\dagger$ & $12_{5,7}$ -- $12_{4,8}$ E & 98.378025 & -5.0192 & 53.26 & 25 & 10 \\
$s$-C$_2$H$_5$CHO$^\dagger$ & $12_{5,7}$ -- $12_{4,8}$ A & 98.378025 & -5.0191 & 53.26 & 25 & 10 \\
$s$-C$_2$H$_5$CHO$^\dagger$ & $29_{7,22}$ -- $29_{6,23}$ A & 98.39443 & -4.847 & 250.45 & 59 & 10 \\
$s$-C$_2$H$_5$CHO$^\dagger$ & $29_{7,22}$ -- $29_{6,23}$ E & 98.394692 & -4.847 & 250.45 & 59 & 10 \\
\hline 
\multicolumn{7}{c}{Others} \\
\hline 
C$_2$H$_3$CN & $9_{1,8}$ -- $8_{1,7}$ & 87.312812 & -4.2778 & 23.13 & 57 & 6 \\
SO$_2$$^\ddagger$ & $7_{3,5}$ -- $8_{2,6}$ & 97.702334 & -5.7413 & 47.84 & 15 & 10 \\
OCS & $7$ -- $6$ & 85.139103 & -5.7658 & 16.34 & 15 & 2
\enddata
\tablenotetext{\dagger}{Blended with the other transitions of the same species.}
\tablenotetext{\ddagger}{Blended with transitions of other species.}
\end{deluxetable*}





\section{Spectra}

\section{LTE spectral model}
Here we describe the details of the spectral model used for the column density derivation. We assume that the excitation condition of observed COM emission is well approximated by local thermodynamic equilibrium (LTE), where the gas temperature is equal to the excitation temperature ($T_\mathrm{ex}$) of the emitting molecules. This assumption should be reasonable as the typical gas density in protoplanetary disks () is higher than the typical critical density of the COM transitions. For example, the critical density of xx-yy transition of \methanol is calculated to be zzz cm-3 at a gas kinetic temperature of 100 K \citep[ref][]{}. We further assume that different species are sufficiently cospatial to share the same emitting region size and excitation temperature. This assumption is also broadly supported by the similar emission extent of different COMs shown in Figure \ref{fig:moment_zero_gallery}. 

The model intensity $I_\nu$ are computed based on a solution of the basic radiative transfer equations for an isothermal, uniform slab:
\begin{equation}
    I_\nu = (B_\nu(T_\mathrm{ex}) - B_\nu(T_\mathrm{CMB})) (1 - e^{-\tau_\nu}),
\end{equation}
where $B_\nu$ is the Planck function for a blackbody radiation, $T_\mathrm{CMB} = 2.73$ K is the temperature of the cosmic microwave background, and $\tau_\nu$ is the optical depth of the line emission at a frequency $\nu$. Mimicking the formulation described in Appendix A of \citet{Yamato2022}, the line optical depth $\tau_\nu$ is computed considering different transitions from various species as
\begin{equation}
    \tau_\nu = \sum_{i}\tau_{0, i} \exp\bigg(-\frac{(\nu - \nu_\mathrm{c})^2}{2 \sigma_\nu^2}\bigg),
\end{equation}
where $i$ is the indices for different transitions, $\tau_{0, i}$ is the optical depth at the line center of $i$th transition, $\nu_\mathrm{c}$ is the central frequency of the transition, and $\sigma_\nu$ is the line width in terms of frequency. The central frequency $\nu_\mathrm{c}$ is calculated as $\nu_\mathrm{c} = \nu_0 (1 - v_\mathrm{sys}/c)$, where $c$ is the speed of light and $v_\mathrm{sys}$ is the systemic velocity of the source, assumed to be $4.25$ km s$^{-1}$ for V883 Ori \citep{Tobin2023}. The frequency line width $\sigma_\nu$ is converted from the velocity line width $\sigma_v$ as $\sigma_\nu = \nu_0 \sigma_v / c$, where $\nu_0$ is the rest frequency of each transition listed in Table \ref{tab:transitions}. In practice, the full width of half maximum (FWHM) of the velocity $\Delta V_\mathrm{FWHM}$ is used as a parameter instead of $\sigma_\nu$ or $\sigma_v$. The optical depth at the line center ($\tau_\mathrm{0, i}$) is calculated as
\begin{equation}
    \tau_{0, i} = \frac{c^2A_\mathrm{u}N_\mathrm{u}}{8\pi\nu_0^2\sqrt{2\pi}\sigma_\nu}\Bigg(\exp\left(\frac{h\nu_0}{k_\mathrm{B}T_\mathrm{ex}}\right) - 1\Bigg),
\end{equation}
\begin{equation}
    \frac{N_\mathrm{u}}{N} = \frac{g_\mathrm{u}}{Q(T_\mathrm{ex})}\exp\left(-\frac{E_\mathrm{u}}{k_\mathrm{B}T_\mathrm{ex}}\right),
\end{equation}
where $A_\mathrm{ul}$ is the Einstein A coefficient for spontaneous emission, $g_\mathrm{u}$ is the upper state degeneracy, $N$ is the molecular column density, $Q$ is the partition function of the molecule, and $E_\mathrm{u}$ is the upper state energy of the transition. While this suit of formulation is similar to that of the eXtended CASA Line Analysis Software Suite (XCLASS; \citealt{Moller2017}), which is used in a previous work on the V883 Ori disk \citep{Lee2019}, we used an independent implementation by ourselves.

The model intensity $I_\nu$ (in unit of Jy sr$^{-1}$) is then integrated over a solid angle of the emitting region $\Omega$, which is common for all transitions and species, to obtain the spectra of flux density (in unit of Jy). Finally, the model flux density spectra are convolved with a Lorentz function $f(\nu)$ with a width of $\gamma$,
\begin{equation}
    f(\nu) = \frac{1}{\pi\gamma} \frac{\gamma^2}{\gamma^2 + \nu^2},
\end{equation}
to take line broadening in the spectra corrected for Keplerian rotation. As demonstrated in Figure \ref{fig:}, the spectra corrected for Keplerian rotation still deviates from a simple Gaussian due to (1) the finite spatial resolution or beam smearing, which cannot fully resolve the highest velocity component of Keplerian rotation, (2) potential emission from the elevated disk surface of the back side and front side of the disk, which have different projected velocities, and (3) spectral Hanning smoothing, which effectively reduces the spectral resolution. In our data, (1) and (2) could be dominant causes of the deviation, since the spatial resolution is not so high to fully resolve the disk emission and the COM emission could originate from the warm disk surface in addition to the midplane, as already shown in the Band 6 data (M. Leemker et al. in preparation). Figure \ref{fig} shows that the convolution with the Lorentz function well approximates the deviation from Gaussian while conserving the velocity-integrated flux density, which is relevant for column density derivation. A similar method has been employed in \citet{Cataldi2021}, where they used a Gaussian convolution to take the line broadening into account (see also \citealt{Bergner2021} and \citealt{Guzman2021}). 

In total, we consider n free parameters $\hat{\theta} = \{\Omega, T_\mathrm{ex}\}$... to fit the observed spectra. The fit 

\section{Spectroscopic data for molecules}





%% For this sample we use BibTeX plus aasjournals.bst to generate the
%% the bibliography. The sample631.bib file was populated from ADS. To
%% get the citations to show in the compiled file do the following:
%%
%% pdflatex sample631.tex
%% bibtext sample631
%% pdflatex sample631.tex
%% pdflatex sample631.tex

\bibliography{sample631}{}
\bibliographystyle{aasjournal}

%% This command is needed to show the entire author+affiliation list when
%% the collaboration and author truncation commands are used.  It has to
%% go at the end of the manuscript.
%\allauthors

%% Include this line if you are using the \added, \replaced, \deleted
%% commands to see a summary list of all changes at the end of the article.
%\listofchanges

\end{document}

% End of file `sample631.tex'.
