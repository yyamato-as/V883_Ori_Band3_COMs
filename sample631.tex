%% Beginning of file 'sample631.tex'
%%
%% Modified 2021 March
%%
%% This is a sample manuscript marked up using the
%% AASTeX v6.31 LaTeX 2e macros.
%%
%% AASTeX is now based on Alexey Vikhlinin's emulateapj.cls 
%% (Copyright 2000-2015).  See the classfile for details.

%% AASTeX requires revtex4-1.cls and other external packages such as
%% latexsym, graphicx, amssymb, longtable, and epsf.  Note that as of 
%% Oct 2020, APS now uses revtex4.2e for its journals but remember that 
%% AASTeX v6+ still uses v4.1. All of these external packages should 
%% already be present in the modern TeX distributions but not always.
%% For example, revtex4.1 seems to be missing in the linux version of
%% TexLive 2020. One should be able to get all packages from www.ctan.org.
%% In particular, revtex v4.1 can be found at 
%% https://www.ctan.org/pkg/revtex4-1.

%% The first piece of markup in an AASTeX v6.x document is the \documentclass
%% command. LaTeX will ignore any data that comes before this command. The 
%% documentclass can take an optional argument to modify the output style.
%% The command below calls the preprint style which will produce a tightly 
%% typeset, one-column, single-spaced document.  It is the default and thus
%% does not need to be explicitly stated.
%%
%% using aastex version 6.3
\documentclass[twocolumn, twocolappendix, astrosymb, times]{aastex631}


%% The default is a single spaced, 10 point font, single spaced article.
%% There are 5 other style options available via an optional argument. They
%% can be invoked like this:
%%
%% \documentclass[arguments]{aastex631}
%% 
%% where the layout options are:
%%
%%  twocolumn   : two text columns, 10 point font, single spaced article.
%%                This is the most compact and represent the final published
%%                derived PDF copy of the accepted manuscript from the publisher
%%  manuscript  : one text column, 12 point font, double spaced article.
%%  preprint    : one text column, 12 point font, single spaced article.  
%%  preprint2   : two text columns, 12 point font, single spaced article.
%%  modern      : a stylish, single text column, 12 point font, article with
%% 		  wider left and right margins. This uses the Daniel
%% 		  Foreman-Mackey and David Hogg design.
%%  RNAAS       : Supresses an abstract. Originally for RNAAS manuscripts 
%%                but now that abstracts are required this is obsolete for
%%                AAS Journals. Authors might need it for other reasons. DO NOT
%%                use \begin{abstract} and \end{abstract} with this style.
%%
%% Note that you can submit to the AAS Journals in any of these 6 styles.
%%
%% There are other optional arguments one can invoke to allow other stylistic
%% actions. The available options are:
%%
%%   astrosymb    : Loads Astrosymb font and define \astrocommands. 
%%   tighten      : Makes baselineskip slightly smaller, only works with 
%%                  the twocolumn substyle.
%%   times        : uses times font instead of the default
%%   linenumbers  : turn on lineno package.
%%   trackchanges : required to see the revision mark up and print its output
%%   longauthor   : Do not use the more compressed footnote style (default) for 
%%                  the author/collaboration/affiliations. Instead print all
%%                  affiliation information after each name. Creates a much 
%%                  longer author list but may be desirable for short 
%%                  author papers.
%% twocolappendix : make 2 column appendix.
%%   anonymous    : Do not show the authors, affiliations and acknowledgments 
%%                  for dual anonymous review.
%%
%% these can be used in any combination, e.g.
%%
%% \documentclass[twocolumn,linenumbers,trackchanges]{aastex631}
%%
%% AASTeX v6.* now includes \hyperref support. While we have built in specific
%% defaults into the classfile you can manually override them with the
%% \hypersetup command. For example,
%%
%% \hypersetup{linkcolor=red,citecolor=green,filecolor=cyan,urlcolor=magenta}
%%
%% will change the color of the internal links to red, the links to the
%% bibliography to green, the file links to cyan, and the external links to
%% magenta. Additional information on \hyperref options can be found here:
%% https://www.tug.org/applications/hyperref/manual.html#x1-40003
%%
%% Note that in v6.3 "bookmarks" has been changed to "true" in hyperref
%% to improve the accessibility of the compiled pdf file.
%%
%% If you want to create your own macros, you can do so
%% using \newcommand. Your macros should appear before
%% the \begin{document} command.
%%
\usepackage{xspace}
\usepackage{amsmath}
\usepackage{txfonts}
\usepackage{multirow}
\newcommand{\vdag}{(v)^\dagger}
\newcommand\aastex{AAS\TeX}
\newcommand\latex{La\TeX}
\newcommand{\hcotp}{HCO$_2^+$}
\newcommand{\methanol}{CH$_3$OH\xspace}
\newcommand{\acetaldehyde}{CH$_3$CHO\xspace}
\newcommand{\methylformate}{CH$_3$OCHO\xspace}
\newcommand{\dimethylether}{CH$_3$OCH$_3$\xspace}
\newcommand{\acetone}{CH$_3$COCH$_3$\xspace}
\newcommand{\ethyleneoxide}{$c$-C$_2$H$_4$O\xspace}
\newcommand{\propenal}{$t$-C$_2$H$_3$CHO\xspace}
\newcommand{\propanal}{$s$-C$_2$H$_5$CHO\xspace}

%% Reintroduced the \received and \accepted commands from AASTeX v5.2
%\received{March 1, 2021}
%\revised{April 1, 2021}
%\accepted{\today}

%% Command to document which AAS Journal the manuscript was submitted to.
%% Adds "Submitted to " the argument.
%\submitjournal{PSJ}

%% For manuscript that include authors in collaborations, AASTeX v6.31
%% builds on the \collaboration command to allow greater freedom to 
%% keep the traditional author+affiliation information but only show
%% subsets. The \collaboration command now must appear AFTER the group
%% of authors in the collaboration and it takes TWO arguments. The last
%% is still the collaboration identifier. The text given in this
%% argument is what will be shown in the manuscript. The first argument
%% is the number of author above the \collaboration command to show with
%% the collaboration text. If there are authors that are not part of any
%% collaboration the \nocollaboration command is used. This command takes
%% one argument which is also the number of authors above to show. A
%% dashed line is shown to indicate no collaboration. This example manuscript
%% shows how these commands work to display specific set of authors 
%% on the front page.
%%
%% For manuscript without any need to use \collaboration the 
%% \AuthorCollaborationLimit command from v6.2 can still be used to 
%% show a subset of authors.
%
%\AuthorCollaborationLimit=2
%
%% will only show Schwarz & Muench on the front page of the manuscript
%% (assuming the \collaboration and \nocollaboration commands are
%% commented out).
%%
%% Note that all of the author will be shown in the published article.
%% This feature is meant to be used prior to acceptance to make the
%% front end of a long author article more manageable. Please do not use
%% this functionality for manuscripts with less than 20 authors. Conversely,
%% please do use this when the number of authors exceeds 40.
%%
%% Use \allauthors at the manuscript end to show the full author list.
%% This command should only be used with \AuthorCollaborationLimit is used.

%% The following command can be used to set the latex table counters.  It
%% is needed in this document because it uses a mix of latex tabular and
%% AASTeX deluxetables.  In general it should not be needed.
%\setcounter{table}{1}

%%%%%%%%%%%%%%%%%%%%%%%%%%%%%%%%%%%%%%%%%%%%%%%%%%%%%%%%%%%%%%%%%%%%%%%%%%%%%%%%
%%
%% The following section outlines numerous optional output that
%% can be displayed in the front matter or as running meta-data.
%%
%% If you wish, you may supply running head information, although
%% this information may be modified by the editorial offices.
\shorttitle{V883 Ori Band 3}
\shortauthors{Yamato et al.}
%%
%% You can add a light gray and diagonal water-mark to the first page 
%% with this command:
%% \watermark{text}
%% where "text", e.g. DRAFT, is the text to appear.  If the text is 
%% long you can control the water-mark size with:
%% \setwatermarkfontsize{dimension}
%% where dimension is any recognized LaTeX dimension, e.g. pt, in, etc.
%%
%%%%%%%%%%%%%%%%%%%%%%%%%%%%%%%%%%%%%%%%%%%%%%%%%%%%%%%%%%%%%%%%%%%%%%%%%%%%%%%%
\graphicspath{{./}{figures/}}
%% This is the end of the preamble.  Indicate the beginning of the
%% manuscript itself with \begin{document}.

\begin{document}

\title{Chemistry of Complex Organic Molecules in the V883 Ori Disk Revealed by ALMA Band 3 Observations}

%% LaTeX will automatically break titles if they run longer than
%% one line. However, you may use \\ to force a line break if
%% you desire. In v6.31 you can include a footnote in the title.

%% A significant change from earlier AASTEX versions is in the structure for 
%% calling author and affiliations. The change was necessary to implement 
%% auto-indexing of affiliations which prior was a manual process that could 
%% easily be tedious in large author manuscripts.
%%
%% The \author command is the same as before except it now takes an optional
%% argument which is the 16 digit ORCID. The syntax is:
%% \author[xxxx-xxxx-xxxx-xxxx]{Author Name}
%%
%% This will hyperlink the author name to the author's ORCID page. Note that
%% during compilation, LaTeX will do some limited checking of the format of
%% the ID to make sure it is valid. If the "orcid-ID.png" image file is 
%% present or in the LaTeX pathway, the OrcID icon will appear next to
%% the authors name.
%%
%% Use \affiliation for affiliation information. The old \affil is now aliased
%% to \affiliation. AASTeX v6.31 will automatically index these in the header.
%% When a duplicate is found its index will be the same as its previous entry.
%%
%% Note that \altaffilmark and \altaffiltext have been removed and thus 
%% can not be used to document secondary affiliations. If they are used latex
%% will issue a specific error message and quit. Please use multiple 
%% \affiliation calls for to document more than one affiliation.
%%
%% The new \altaffiliation can be used to indicate some secondary information
%% such as fellowships. This command produces a non-numeric footnote that is
%% set away from the numeric \affiliation footnotes.  NOTE that if an
%% \altaffiliation command is used it must come BEFORE the \affiliation call,
%% right after the \author command, in order to place the footnotes in
%% the proper location.
%%
%% Use \email to set provide email addresses. Each \email will appear on its
%% own line so you can put multiple email address in one \email call. A new
%% \correspondingauthor command is available in V6.31 to identify the
%% corresponding author of the manuscript. It is the author's responsibility
%% to make sure this name is also in the author list.
%%
%% While authors can be grouped inside the same \author and \affiliation
%% commands it is better to have a single author for each. This allows for
%% one to exploit all the new benefits and should make book-keeping easier.
%%
%% If done correctly the peer review system will be able to
%% automatically put the author and affiliation information from the manuscript
%% and save the corresponding author the trouble of entering it by hand.

%\correspondingauthor{August Muench}
%\email{greg.schwarz@aas.org, gus.muench@aas.org}

\author[0000-0003-4099-6941]{Yoshihide Yamato}
\affiliation{Department of Astronomy, Graduate School of Science, The University of Tokyo, 7-3-1 Hongo, Bunkyo-ku, Tokyo 113-0033, Japan}

\author[0000-0003-2493-912X]{Shota Notsu}
\affiliation{Department of Astronomy, Graduate School of Science, The University of Tokyo, 7-3-1 Hongo, Bunkyo-ku, Tokyo 113-0033, Japan}
\affiliation{Star and Planet Formation Laboratory, RIKEN Cluster for Pioneering Research, 2-1 Hirosawa, Wako, Saitama 351-0198, Japan}

\author[0000-0003-3283-6884]{Yuri Aikawa}
\affiliation{Star and Planet Formation Laboratory, RIKEN Cluster for Pioneering Research, 2-1 Hirosawa, Wako, Saitama 351-0198, Japan}

\author[0000-0002-3297-4497]{Nami Sakai}
\affiliation{Star and Planet Formation Laboratory, RIKEN Cluster for Pioneering Research, 2-1 Hirosawa, Wako, Saitama 351-0198, Japan}

\author[0000-0003-3655-5270]{Yuki Okoda}
\affiliation{Star and Planet Formation Laboratory, RIKEN Cluster for Pioneering Research, 2-1 Hirosawa, Wako, Saitama 351-0198, Japan}

\author[0000-0002-7058-7682]{Hideko Nomura}
\affiliation{Division of Science, National Astronomical Observatory of Japan, 2-21-1 Osawa, Mitaka, Tokyo 181-8588, Japan}

%% Note that the \and command from previous versions of AASTeX is now
%% depreciated in this version as it is no longer necessary. AASTeX 
%% automatically takes care of all commas and "and"s between authors names.

%% AASTeX 6.31 has the new \collaboration and \nocollaboration commands to
%% provide the collaboration status of a group of authors. These commands 
%% can be used either before or after the list of corresponding authors. The
%% argument for \collaboration is the collaboration identifier. Authors are
%% encouraged to surround collaboration identifiers with ()s. The 
%% \nocollaboration command takes no argument and exists to indicate that
%% the nearby authors are not part of surrounding collaborations.

%% Mark off the abstract in the ``abstract'' environment. 
\begin{abstract}
Complex organic molecules (COMs) in protoplanetary disks are key to understanding the origin of volatiles in comets in our solar system, yet the chemistry of COMs in protoplanetary disks remains poorly understood.
% because COMs are mainly hidden in the cold ice mantles. 
Here we present Atacama Large Millimeter/submillimeter Array (ALMA) Band 3 observations of the disk around the young outbursting star V883 Ori, where the COMs sublimate from ices and are thus observable thanks to the warm condition of the disk. We have robustly identified eleven oxygen-bearing COMs including $^{13}$C-isotopologues in the disk-integrated spectra, five of which are the first detection in this disk. The radial distributions of the COM emission, revealed by the detailed analyses of the line profiles, show the inner emission cavity, similar to the previous observations in Band 6 and Band 7. We found that the COMs abundance ratios with respect to methanol are significantly higher than those in the warm protostellar envelopes of IRAS~16293-2422 and similar to the ratios in the solar system comet 67P/Churyumov-Gerasimenko, suggesting the efficient (re-)formation of COMs in protoplanetary disks. We also constrained the $^{12}$C/$^{13}$C and D/H ratios of COMs in protoplanetary disks for the first time. The $^{12}$C/$^{13}$C ratios of acetaldehyde, methyl formate, and dimethyl ether are consistently lower ($\sim$\,20--30) than the canonical ratio in the interstellar medium ($\sim$\,69), indicating the efficient $^{13}$C-fractionation of CO. The D/H ratios of methanol and methyl formate are slightly lower than the typical values in protostellar envelopes, possibly pointing to the destruction and reformation of COMs in disks. We also discuss the implications for nitrogen and sulfur chemistry in protoplanetary disks.  

\end{abstract}


%% Keywords should appear after the \end{abstract} command. 
%% The AAS Journals now uses Unified Astronomy Thesaurus concepts:
%% https://astrothesaurus.org
%% You will be asked to selected these concepts during the submission process
%% but this old "keyword" functionality is maintained in case authors want
%% to include these concepts in their preprints.
%% \keywords{Classical Novae (251) --- Ultraviolet astronomy(1736) --- History of astronomy(1868) --- Interdisciplinary astronomy(804)}

%% From the front matter, we move on to the body of the paper.
%% Sections are demarcated by \section and \subsection, respectively.
%% Observe the use of the LaTeX \label
%% command after the \subsection to give a symbolic KEY to the
%% subsection for cross-referencing in a \ref command.
%% You can use LaTeX's \ref and \label commands to keep track of
%% cross-references to sections, equations, tables, and figures.
%% That way, if you change the order of any elements, LaTeX will
%% automatically renumber them.
%%
%% We recommend that authors also use the natbib \citep
%% and \citet commands to identify citations.  The citations are
%% tied to the reference list via symbolic KEYs. The KEY corresponds
%% to the KEY in the \bibitem in the reference list below. 

\section{Introduction} \label{sec:intro}
Our Solar System formed from its protoplanetary disk, or the proto-solar disk, a byproduct of the Sun's formation via the gravitational collapse of the parent molecular cloud. Observations have found many volatile ices in solar system comets, which should contain information about the chemical composition of the solar nebula, suggesting that the proto-solar disk was rich in volatiles such as water and organics. Organic molecules are of particular interest because they could be the precursors of the prebiotic molecules that may have given rise to life on Earth. \citep{Ceccarelli2023}.

The origin of the cometary water and organics has been extensively discussed but remains controversial. The most commonly discussed possibility is that the cometary ices are, at least in part, inherited from the interstellar medium (ISM) through the formation of the Solar System. 
% For example, the high D/H ratio of water observed in a protoplanetary disk, which shows a similar value to those in protostellar envelopes and comets, strongly suggests the inheritance of water from the ISM rather than the chemical reset (e.g., destruction and reformation of water) in the disk \citep{Tobin2023}. 
For example, the higher D$_2$O/HDO ratio than the HDO/H$_2$O ratio in both the warm inner envelopes of protostars \citep{Coutens2014, Jensen2021} and a solar system comet \citep{Altwegg2017}
% , which is naturally explained by the bulk water formation in cloud-formation stage followed by accelerated deuteration in prestellar stage \citep{Furuya2016}, 
strongly suggests the inheritance of water from the ISM rather than the chemical reset (e.g., destruction and reformation).
Organic molecules have also been detected in both protostellar envelopes and solar system comets \citep[e.g.,][]{Jorgensen2016, Rubin2019}. In the warm inner envelopes heated by the central protostar, organic molecules are thermally desorbed at a typical sublimation temperature of $\sim100$\,K, allowing us to observe molecules sublimated from ices. Among these molecules, carbon-bearing molecules with six or more atoms are empirically referred to complex organic molecules (COMs). The composition of COMs in protostellar envelopes is compared with that in comets, from which the possible inheritance of COMs from the prestellar and protostellar phases is discussed \citep[e.g.,][]{Drozdovskaya2019}.

Observations of COMs in protoplanetary disks are rather sparse compared to those in protostellar envelopes, since in the bulk of typical disks around T Tauri stars, COMs are mainly locked into ices due to its colder nature ($<$\,100~K).  Relatively simple organic molecules such as \methanol, CH$_3$CN, and HCOOH, which are non-thermally desorbed and/or formed via the gas-phase chemistry in the outer cold region, are detected in a handful of disks \citep{Walsh2016, Oberg2015, Bergner2018, Loomis2018, Ilee2021, Favre2018}, but these observations do not directly trace the bulk composition of COMs hidden in the ice mantles. Recently, high-sensitivity observations with Atacama Large Millimeter/submillimeter Array (ALMA) have just begun to detect the emission of thermally desorbed COMs in warm disks around Herbig Ae/Be stars \citep{vanderMarel2021, Booth2021_CH3OH, Brunken2022, Booth2023_HD169142}, but the detections are still limited to a few numbers of disks and molecular species. Since protoplanetary disks could be potential sites for the chemical reset from the ISM \citep[e.g.,][]{Walsh2014, Furuya2014, Eistrup2016}, constraining the composition and distribution of COMs in disks is crucial to fully trace the chemical evolution during star and planet formation.

Recent observations of the disk around the young outbursting star, V883 Ori, have opened a new window to COMs in protoplanetary disks. V883 Ori is a low-mass ($M_\star = 1.3\,M_\odot$; \citealt{Cieza2016}) FU Orionis type object located in the L1641 cluster in the Orion molecular clouds \citep[$d\approx400$\,pc;][]{Strom1993} with a well-developed Keplerian-rotating disk \citep{Cieza2016}, which shows a large-amplitude outburst in the optical ($\Delta m_\mathrm{V} > 4$\,mag) with an increased luminosity of $\sim200\,L_\odot$ \citep{Furlan2016}. This high luminosity warms the disk and shifts the sublimation front (or snowline) of water and COMs \citep{Tobin2023}, allowing for observing thermally desorbed molecules in the disk. In addition, because the duration of the outburst ($\sim10^2$\,yr) is shorter than the timescale of the typical gas-phase chemical reaction (several $10^4$\,yr; e.g., \citealt{Nomura2009}), it is possible to observe the fresh sublimates without significant gas-phase chemical changes and probe the composition of disk ices almost directly. 
% Indeed, the location of water snowline has recently been estimated to be approximately $\sim80$\,au, which is a far large radius compared to the typical radius of snowline in T Tauri disks ($\sim1$--5\,au), by the direct detection of gas phase water \citep{Tobin2023}. Since water sublimation temperature ($\sim100$--150\,K) is similar to or higher than that of COMs \citep[e.g.,][]{Furuya2014}, COMs are also expected ro be detected.
\citet{vantHoff2018} detected \methanol emission for the first time in this disk with high-resolution ($\sim0\farcs14$) ALMA Band 7 observations, followed by the additional detection of several COMs such as \methylformate, \acetaldehyde, and CH$_3$CN in ALMA Band 7 at $\sim0\farcs1$ resolution \citep{Lee2019}. These COM emissions all show the velocity structure consistent with the Keplerian rotation traced by the C$^{18}$O emission \citep{Cieza2016, vantHoff2018}, proving that the COM emission originates from the rotationally supported disk. The COM emission spreads over 40--120 au radii with an inner emission cavity ($\lesssim40$\,au radius), similar to the spatial distribution of the water emission \citep{Tobin2023}. \citet{Lee2019} compared the relative abundances of a few COMs with respect to \methanol, the simplest COM, with those in the warm envelopes of the Class 0 protostar IRAS~16293-2422~B \citep[hereafter IRAS~16293B; e.g.,][]{Jorgensen2016} and the solar system comet 67P/Churyumov-Gerasimenko \citep[hereafter 67P/C-G; e.g.,][]{Altwegg2019}. They found that the abundances of COMs in the V883 Ori disk are generally higher than those in IRAS~16293B (except for CH$_3$CN) and similar to those in 67P/C-G, suggesting potential chemical evolution in protoplanetary disks.

However, there are several limitations to these observations. First, in ALMA Band 7, i.e., the sub-mm wavelengths, the intense dust continuum emission obscures the molecular line emission in the innermost region ($\lesssim40$\,au radius), resulting in an apparent inner cavity in the molecular line emission \citep{vantHoff2018, Lee2019, Tobin2023}. This prevents us from studying the chemical compositions there, i.e., the comet- and terrestrial-planet-forming zone, which can be directly compared with measurements in comets in our solar system. Second, the limited number of unambiguously detected species \citep{Lee2019} makes it difficult to derive a general and concrete picture of chemical evolution in protoplanetary disks. Third, previous observations did not measure the isotopic ratios (e.g., $^{12}$C/$^{13}$C and D/H) of COMs \citep[except for the D/H ratio of \methanol;][]{Lee2019, Lee2023}, the most sensitive chemical fingerprint for tracing the formation conditions and thermal history.

In this paper, we present the new observations of COMs toward the V883 Ori disk in ALMA Band 3, where the dust continuum emission is expected to be fainter than in the sub-mm wavelength due to the lower opacity of dust grains at longer wavelengths. Observational and imaging details are described in Section \ref{sec:observation}, followed by a detailed analysis of the disk-integrated spectra in Section \ref{sec:analysis_result}, where we report the first detection of several species including $^{13}$C- and D-isotopologues. We discuss the implications of the present observations for the physical and chemical structure of the V883~Ori disk, as well as for the chemical composition and isotopic chemistry of COMs in protoplanetary disks in Section \ref{sec:discussion}. We finally summarize the present work in Section \ref{sec:summary}.
% Complex organic molecules (COMs) are the precursors of pre-biotic molecules such as amino acids and and sugars. The COMs in the interstellar medium (ISM) can be delivered to planetary systems via various processes of star and planet formation and build the chemical inventory of planets, which is represented by the present Solar System. Studying COMs in star- and planet-forming regions is thus key to understanding the origin of chemical complexity in Solar System \citep[e.g.,][]{Ceccarelli2023}.  

% One of the promising way to approach the origin of the chemical complexity is to observationally characterize the organic chemistry in low-mass star-forming regions and compare it to the Solar System measurements. In the warm inner envelopes around young protostars, COMs have been ubiquitously detected in high-resolution observations with Atacama Large Millimeter/submillimeter Array (ALMA). \citep{}


\section{Observations} \label{sec:observation}

\subsection{ALMA Band 3 Observations}
V883 Ori was observed in Band 3 during ALMA Cycle 8 (project code: 2021.1.00357.S, PI: S. Notsu). The observations were carried out in a total of four executions, one with a compact antenna configuration and the other three with an extended antenna configuration. The observation dates, number of antennas, on-source integration time, precipitable water vapor (PWV), baseline coverage, angular resolution, maximum recoverable scales (MRS), and calibrator information are listed in Table \ref{tab:observations}. 

The correlator setup included eleven spectral windows (SPWs) in Frequency Division Mode (FDM), one of which was dedicated to continuum acquisition with a wide bandwidth of 937.5\,MHz for an accurate determination of the continuum level. The frequency resolution of the continuum SPW was 0.488\,MHz, resulting in a velocity resolution of $\sim$1.5\,km \,s$^{-1}$. The continuum SPW included many spectral lines of COMs. Other ten SPWs targeted specific spectral lines with narrower bandwidths of 58.59\,MHz or 117.19\,MHz, which also covered many COMs lines. The frequency resolution of these SPWs was 0.141\,MHz, resulting in a velocity resolution of $\sim$0.4--0.5\,km\,s$^{-1}$. 
% For all SPWs, the spectral averaging factor was set to 1, and thus the native channel width was half the spectral resolution. 
The detailed properties of the SPWs are summarized in Table \ref{tab:cube_properties}. We note that two SPWs (8 and 9) were partially overlapped, and only the wider SPW 9 is used for analysis. We also note that the primary target of this observation was two protonated carbon dioxide (\hcotp) lines, which will be presented in future work (S. Notsu et al. in prep.). This paper focuses on a large number of COMs lines covered by these SPWs, including the continuum SPW.


\subsection{Calibration and Imaging}
Initial calibrations were performed by the ALMA staff using the standard ALMA calibration pipeline version 2021.2.0.128. Subsequent self-calibration and imaging were performed using the Common Astronomy Software Applications \citep[CASA;][]{CASA} version 6.2.1.7. The pipeline-calibrated data were first imaged without any deconvolution (i.e., dirty imaging) or continuum subtraction to accurately specify the line-free channels by visually inspecting the dirty image cubes. These line-free channels are averaged to obtain the continuum visibilities. Self-calibration was then performed using the continuum visibilities with the \texttt{gaincal} and \texttt{applycal} tasks. A round of each phase-only and phase-plus-amplitude self-calibration was first performed on the data with a compact antenna configuration. The extended antenna configuration data were then concatenated and self-calibrated together. Two rounds of phase-only self-calibration and one round of phase-plus-amplitude self-calibration were performed on the combined data. The solutions were then applied to the spectral line visibilities.
% , which were then continuum-subtracted using the CASA task \texttt{uvcontsub}. 

Each of the SPWs was imaged using the \texttt{tclean} task \citep{Hogbom1974} with Briggs weighting (\texttt{robust} $=$ 0.5). The \texttt{tclean} task was run in parallel using \texttt{mpicasa} implementation. As a specific weighting scheme, \texttt{briggsbwtaper} was used with independent weight densities for each channel (i.e., \texttt{perchanweightdensity = True}) to achieve more uniform sensitivity across channels. For each SPW, a common restoring beam across all channels is used. A few channels at the edge of the SPW are removed during the imaging process due to the large deviation in beam size caused by a slight difference in the frequency coverage between the data with compact and extended antenna configurations\footnote{see \url{https://casadocs.readthedocs.io/en/stable/notebooks/memo-series.html\#Correcting-bad-common-beam}.}. All SPWs were cleaned down to 3\,$\times$\,RMS with the native channel widths. An automatic masking algorithm with \texttt{automultithresh} was also used to generate the CLEAN mask. The typical beam size and the RMS noise level were $\sim0\farcs3$--$0\farcs4$ and 0.6--1\,mJy\,beam$^{-1}$, respectively. The RMS noise level was measured on the images without primary beam correction. The beam sizes and RMS noise levels for each SPW are listed in Table \ref{tab:cube_properties}. Finally, the continuum component is subtracted from these image cubes using the \texttt{imcontsub} task to produce the spectral-line-only image cubes. The continuum subtraction on the image plane instead of the visibility plane (e.g., with the \texttt{uvcontsub} task) mitigates as much as possible the Jorsater \& van Moorsel (JvM) effect \citep{JvM, Czekala2021} on the line emission, which is critical for the flux scale (see Appendix \ref{appendix:JvM_effect}). Throughout this paper, we use the image cubes that were continuum-subtracted on the image plane, unless otherwise noted.


% \subsection{JvM Correction}\label{subsec:JvM_correction}
% The restored images with the default CLEANing process were further processed to account for the Jorsater \& van Moorsel \citep[][hereafter JvM]{JvM} effect: the JvM correction. Due to the inconsistency between the units of the residual image (in Jy per dirty beam) and the model image (in Jy per CLEAN beam), the flux scale in the CLEANed image can be incorrect, particularly for the faint emission. We followed the approach described in \citet{Czekala2021}, where the correction is made by rescaling the residual image by the ratio of the CLEAN beam and dirty beam volumes ($\epsilon$). The JvM $\epsilon$ is $\approx$ 0.29 for all SPWs as listed in Table \ref{tab:cube_properties}. The small $\epsilon$ values indicate that the effect is severe, as expected for the dataset combined with different antenna configurations \citep{Czekala2021}. While the JvM correction recovers the correct flux scale on the CLEANed image, which is critical for determining molecular column densities and temperatures, \citet{Casassus2022} cautioned that the JvM correction may exaggerate the S/N of the emission by artificially reducing the noise level. We thus show the spectra extracted from the JvM-uncorrected images as well in Appendix \ref{}, to ensure that the detections of molecular line emission are statistically significant. 
% After the JvM correction, the primary beam correction was also applied to the JvM-corrected image cubes. Throughout this paper, we use the JvM-corrected, primary-beam-corrected images, unless otherwise stated. 






\begin{deluxetable*}{ccccccccc}
\label{tab:observations}
\tablecaption{Observational Details}
\tablehead{\colhead{Date} & \colhead{\# of Ant.} & \colhead{On-source Int.} & \colhead{PWV}  & \colhead{Baseline} & \colhead{Ang. Res.} & \colhead{MRS} & \colhead{Bandpass/Amplitude Cal.} & \colhead{Phase Cal.} \\
\colhead{} & \colhead{} & \colhead{(min)} & \colhead{(mm)} & \colhead{(m)} & \colhead{($\arcsec$)} & \colhead{($\arcsec$)} & \colhead{} & \colhead{}}
\startdata
2021 Nov. 21 & 44 & 43 & 2.2 & 41--3396 & 0.3 & 5.2 & J0423-0120 & J0541-0541 \\
2021 Nov. 21 & 44 & 43 & 2.0 & 41--3396 & 0.3 & 5.2 & J0538-4405 & J0541-0541 \\
2021 Nov. 22 & 43 & 43 & 3.8 & 41--3396 & 0.3 & 4.7 & J0423-0120 & J0541-0541 \\
2022 Jan. 20 & 41 & 47 & 3.6 & 14--740  & 1.5 & 17.4 & J0423-0120 & J0541-0541
\enddata
\end{deluxetable*}

% \begin{deluxetable*}{cccccccccc}
% \label{tab:cube_properties}
% \tablecaption{Properties of Image Cubes}
% \tablehead{\colhead{SPW} & \colhead{Cent. Freq.} & \colhead{\# of Chan.} & \colhead{Bandwidth} & \multicolumn{2}{c}{Channel Width}  & \colhead{Vel. Res.} & \colhead{Beam Size (P.A.)} & \colhead{RMS} & \colhead{JvM $\epsilon$} \\
% \colhead{} & \colhead{(GHz)} & \colhead{} & \colhead{(MHz)} & \colhead{(MHz)} & \colhead{(km\,s$^{-1}$)} & \colhead{(km\,s$^{-1}$)} & \colhead{} & \colhead{(mJy\,beam$^{-1}$}) & \colhead{}}
% \startdata
% 0 & 85.160960 & 478 & 58.229 & 0.122 & 0.43 & 0.43 & 0$\farcs$42$\times$0$\farcs$31 (-75$\arcdeg$) & 1.1 & 0.29 \\
% 1 & 85.530202 & 478 & 58.229 & 0.122 & 0.43 & 0.43 & 0$\farcs$42$\times$0$\farcs$31 (-75$\arcdeg$) & 1.0 & 0.29 \\
% 2 & 85.851713 & 478 & 58.229 & 0.122 & 0.43 & 0.43 & 0$\farcs$42$\times$0$\farcs$31 (-75$\arcdeg$) & 0.99 & 0.28 \\
% 3 & 85.924957 & 478 & 58.229 & 0.122 & 0.43 & 0.43 & 0$\farcs$41$\times$0$\farcs$31 (-75$\arcdeg$) & 0.99 & 0.28 \\
% 4 & 86.669481 & 478 & 58.229 & 0.122 & 0.42 & 0.42 & 0$\farcs$41$\times$0$\farcs$31 (-75$\arcdeg$) & 0.94 & 0.29 \\
% 5 & 86.752980 & 958 & 116.824 & 0.122 & 0.42 & 0.42 & 0$\farcs$41$\times$0$\farcs$31 (-75$\arcdeg$) & 0.94 & 0.29 \\
% 6 & 87.327244 & 478 & 58.229 & 0.122 & 0.42 & 0.42 & 0$\farcs$41$\times$0$\farcs$31 (-75$\arcdeg$) & 0.98 & 0.29 \\
% 7 & 96.490854 & 478 & 58.229 & 0.122 & 0.38 & 0.38 & 0$\farcs$37$\times$0$\farcs$28 (-75$\arcdeg$) & 1.0 & 0.28 \\
% 8 & 96.743210 & 478 & 58.229 & 0.122 & 0.38 & 0.38 & 0$\farcs$37$\times$0$\farcs$27 (-75$\arcdeg$) & 0.99 & 0.30 \\
% 9 & 96.754197 & 958 & 116.824 & 0.122 & 0.38 & 0.37 & 0$\farcs$37$\times$0$\farcs$27 (-75$\arcdeg$) & 0.99 & 0.30 \\
% 10 & 97.979676 & 3839 & 937.036 & 0.244 & 0.75 & 1.5 & 0$\farcs$36$\times$0$\farcs$26 (-73$\arcdeg$) & 0.59 & 0.33
% \enddata
% \end{deluxetable*}

\begin{deluxetable*}{cCCCCCCCC}
\label{tab:cube_properties}
\tablecaption{Properties of Image Cubes}
\tablehead{\colhead{SPW} & \colhead{Cent. Freq.} & \colhead{\# of Chan.} & \colhead{Bandwidth} & \multicolumn{2}{c}{Channel Width}  & \colhead{Vel. Res.} & \colhead{Beam Size (P.A.)} & \colhead{RMS} \\
\colhead{} & \colhead{(GHz)} & \colhead{} & \colhead{(MHz)} & \colhead{(MHz)} & \colhead{(km\,s$^{-1}$)} & \colhead{(km\,s$^{-1}$)} & \colhead{} & \colhead{(mJy\,beam$^{-1}$})}
\startdata
0 & 85.160960 & 478 & 58.229 & 0.122 & 0.43 & 0.51 & 0$\farcs$42$\times$0$\farcs$31\,(-75$\arcdeg$) & 1.1 \\
1 & 85.530202 & 478 & 58.229 & 0.122 & 0.43 & 0.50 & 0$\farcs$42$\times$0$\farcs$31\, (-75$\arcdeg$) & 1.0 \\
2 & 85.851713 & 478 & 58.229 & 0.122 & 0.43 & 0.50 & 0$\farcs$42$\times$0$\farcs$31\, (-75$\arcdeg$) & 0.99 \\
3 & 85.924957 & 478 & 58.229 & 0.122 & 0.43 & 0.50 & 0$\farcs$41$\times$0$\farcs$31\, (-75$\arcdeg$) & 0.99 \\
4 & 86.669481 & 478 & 58.229 & 0.122 & 0.42 & 0.50 & 0$\farcs$41$\times$0$\farcs$31\, (-75$\arcdeg$) & 0.94 \\
5 & 86.752980 & 958 & 116.824 & 0.122 & 0.42 & 0.50 & 0$\farcs$41$\times$0$\farcs$31\, (-75$\arcdeg$) & 0.94 \\
6 & 87.327244 & 478 & 58.229 & 0.122 & 0.42 & 0.49 & 0$\farcs$41$\times$0$\farcs$31\, (-75$\arcdeg$) & 0.98 \\
7 & 96.490854 & 478 & 58.229 & 0.122 & 0.38 & 0.45 & 0$\farcs$37$\times$0$\farcs$28\, (-75$\arcdeg$) & 1.0 \\
8 & 96.743210 & 478 & 58.229 & 0.122 & 0.38 & 0.45 & 0$\farcs$37$\times$0$\farcs$27\, (-75$\arcdeg$) & 0.99 \\
9 & 96.754197 & 958 & 116.824 & 0.122 & 0.38 & 0.45 & 0$\farcs$37$\times$0$\farcs$27\, (-75$\arcdeg$) & 0.99 \\
10 & 97.979676 & 3839 & 937.036 & 0.244 & 0.75 & 1.5 & 0$\farcs$36$\times$0$\farcs$26\, (-73$\arcdeg$) & 0.59
\enddata
\end{deluxetable*}


\section{Data Analysis and Result}\label{sec:analysis_result}
\subsection{Spectrum Extraction and Line Identification}\label{subsec:line_identification}
Figure \ref{fig:spectra_gallery} shows the spectra integrated over 1\farcs2 aperture and corrected for the line broadening due to the Keplerian rotation of the disk. The original disk-integrated spectra are shown in Figures \ref{fig:disk-integrated_spectra0}--\ref{fig:disk-integrated_spectra3} in Appendix \ref{appendix:spectra}. To obtain the spectra corrected for the Keplerian line broadening, we used the \texttt{integrated\_spectrum()} function implemented in the Python package \texttt{GoFish} \citep{GoFish}. This method first deprojects the disk and then aligns the Dopper-shifted spectra at each position within the disk to a common velocity (the systemic velocity of the source) to recover the single peak spectra for each transition, facilitating the identification of the blended transitions. In this procedure, we assumed a position angle (32\arcdeg) and inclination angle (38.3\arcdeg) of the disk, a central stellar mass of $1.29\,M_\odot$, and a distance of 400\,pc, based on previous works \citep{Cieza2016, Tobin2023}. We also assumed that the emission originates from the midplane of the disk, and did not consider the vertical extent of the emission. The uncertainties of the spectra were calculated per channel basis within the \texttt{integrated\_spectrum()} function taking into account the spectral decorrelation \citep{Yen2016}.

As shown in Figure \ref{fig:spectra_gallery}, numerous spectral features are detected in the disk-integrated spectra. Since many of the COM transitions exist in narrow frequency ranges and overlap between different molecular species, visual identification of the lines is challenging. Alternatively, we fitted a synthetic spectrum to the observed spectra to identify each of these features. The details of the synthetic spectral model are described in Appendix \ref{appendix:spectral_model}. Briefly, the model used a common emitting region size, a common excitation temperature, and a common line width under the local thermodynamic equilibrium (LTE) conditions. We varied the emitting region size, excitation temperature, line width, and the column density of each species to fit the observed spectra, and test if they could be reproduced with the model. To construct the model, the spectroscopic data of the molecular lines are taken from the Jet Propulsion Spectroscopy \citep[JPL;][]{JPL} and the Cologne Database for Molecular Spectroscopy \citep[CDMS;][]{CDMS1, CDMS2, CDMS3}, as detailed in Appendix \ref{appendix:spectroscopic_data}. Only the transitions with an Einstein A coefficient for spontaneous emission of $\geq10^{-8}$\,s$^{-1}$ and an upper state energy of $\leq1000$\,K are considered in the model.

Following \citet{Jorgensen2020}, we set two criteria for the identification of molecular species. First, the synthetic spectrum of multiple (more than one) transitions of a given species accounts for the observed spectra including at least one detected transitions that is not blended with transitions of other species. Second, no undetected transitions are over-predicted by the synthetic spectrum. Species that fulfill the second criterion but with a single (unblended) detected transition are considered as tentative identifications. Based on these criteria, we robustly identified 11 species and their isotopologues: \methanol, \acetaldehyde, \methylformate, \dimethylether, \ethyleneoxide, \acetone, \propenal, \propanal, $^{13}$CH$_3$OCHO, CH$_3$O$^{13}$CHO, and CH$_3^{13}$CHO. We also tentatively identified 5 species and isotopologues: $^{13}$CH$_3$OCH$_3$, CH$_2$DOH, CH$_2$DCHO, CH$_3$CDO, and C$_2$H$_3$CN. In addition, one transition of each sulfur-bearing molecule (OCS and SO$_2$) was detected. The detected transitions are listed in Table \ref{tab:transitions} in Appendix \ref{appendix:transitions}.



% The identification of species is based on the number of transitions detected; if multiple transitions of the species are detected at $>3\sigma$ in the stacked spectra, we regard that species as a firm identification. If only one transition is detected at $>3\sigma$ in the stacked spectra, it is considered as a tentative identification. 

\begin{figure*}
\epsscale{1.15}
\plotone{V883_Ori_stacked_spectra_model_v2.pdf}
\caption{Spectra corrected for Keplerian rotation toward the V883 Ori disk (gray). The model spectra for each species are shown in colored lines. }
\label{fig:spectra_gallery}
\end{figure*}

\begin{figure*}
\epsscale{1.15}
\plotone{V883_Ori_mom0_gallery.pdf}
\caption{Velocity-integrated intensity maps of the molecular line emission in the V883 Ori disk. The velocity range for integration are $\pm3.5$ km s$^{-1}$ with respect to the source systemic velocity $v_\mathrm{sys}=4.25$ km s$^{-1}$ \citep{Tobin2023} including \dimethylether and \acetone, where multiple blended transitions are integrated together. The molecular species, transitions, and upper state energy levels are indicated in the upper-left corner of each panel. The black contours start from 5$\sigma$ with steps of 2.5$\sigma$, where $\sigma$ are the noise level of each map measured on the emission-free region. The synthesized beam and a scale bar of 80 au are shown in the lower left and right corner of each panel, respectively.}
\label{fig:moment_zero_gallery}
\end{figure*}


% The uncertainties of the original spectra were calculated in each channel as $\sqrt{\mathrm{\Omega_\mathrm{aperture}/\Omega_\mathrm{beam}}} \times \sigma$, where $\Omega_\mathrm{aperture}$, $\Omega_\mathrm{beam}$ and $\sigma$ are the aperture area, the beam area, and the image RMS, respectively. 

% Figure \ref{fig}(a) shows the continuum image of the V883 Ori disk at $\sim$ 3\,mm. The continuum emission was imaged with a Briggs weighting (robust $=0.5$). The beam size and RMS noise level of the continuum image were 0.4 x 0.3 and xx mJy beam-1, respectively. The continuum emission associated with the V883 Ori disk is marginally spatially resolved, and its extent is in agreement with the previous observations with higher angular resolution \citep{Cieza2016}. 

% Figure \ref{fig}(b)--(d) shows the velocity-integrated intensity maps (contour) and the velocity centroid maps (color) of three representative molecular lines (\methanol, \acetaldehyde, \methylformate).

% Figure \ref{fig:} shows the disk-averaged spectra toward the V883 Ori disk. These spectra were extracted from the central   

\subsection{Spatial Distributions}
\subsubsection{Velocity-integrated Intensity Maps}
Figure \ref{fig:moment_zero_gallery} shows the velocity-integrated intensity map (zeroth moment map) of the selected transitions without significant blending with other species. These maps are created using \texttt{bettermoments} \citep{bettermoments} by integrating over the velocity range of $\pm3.5$ km s$^{-1}$ with respect to the source systemic velocity ($v_\mathrm{sys} = 4.25$ km s$^{-1}$; \citealt{Tobin2023}) without any masking. For \dimethylether and \acetone, multiple blended transitions of each species are integrated together. For these molecules, the integration range is the combination of the default integration range ($\pm3.5$ km s$^{-1}$) for the multiple blended transitions. The molecular line emission is all confined to the disk region ($\sim0\farcs3$ or $\sim120$ au radius) and is marginally spatially resolved. While the spatial extent of the emission is consistent with previous ALMA observations in Bands 6 and 7 \citep{vantHoff2018, Lee2019, Tobin2023} considering the difference in the beam size, the central emission cavities seen in the Band 6 and 7 data are not detected in these maps at the current spatial resolution of $\sim0\farcs3$.
% We will discuss the spatial distributions of COM emission in more detail in Section \ref{subsubsec:spatial_dist}.

\subsubsection{Line Profile Analysis}\label{subsubsec:line_profile_analysis}
As the disk emission is only marginally spatially resolved, it is difficult to directly infer the distribution of the emission in the inner region, particularly the emission cavity seen in the sub-mm line observations in Bands 6 and 7 \citep{vantHoff2018, Lee2019, Tobin2023}. Alternatively, we can indirectly probe the emission in the inner region, which is not spatially resolved in the velocity-integrated intensity maps (Figure \ref{fig:moment_zero_gallery}), from the line profile by assuming gas kinematics. 
% If we assume that the gas in the disk follows the Keplerian rotation, 
For the line emission of gas in a Keplerian-rotating disk, the high-velocity components with respect to the systemic velocity, or the line wings, purely contain information about the emission from the inner region. More specifically, the maximum velocity of Keplerian rotation as a function of disk radius ($v_\mathrm{max}(r)$) can be written as 
\begin{equation}\label{eq:Keplerian_rotation}
    v_\mathrm{max}(r) = \sqrt{\frac{GM_\star}{r}}\sin i,
\end{equation}
where $G$ is the gravitational constant, $M_\star$ is the central stellar mass, and $i$ is the inclination of the disk. The V883 Ori disk has been revealed to be a Keplerian-rotating disk by the higher-resolution observations in Bands 6 and 7 \citep{Cieza2016, Lee2019}. To confirm this in the present Band 3 data, we examined the line profiles of three bright transitions of \methanol, \acetaldehyde, and \methylformate (Figure \ref{fig:line_profile}). These line profiles are all double-peaked, consistent with the Keplerian rotation. We then compared these profiles with the maximum velocity of Keplerian rotation at certain radii (Equation \ref{eq:Keplerian_rotation}) in Figure \ref{fig:line_profile}. As seen in the line wings of each line profile, no significant emission is detected in the velocity channels corresponding to $\lesssim$ 40 au radius. The channel maps (Figures \ref{fig:channelmap_CH3OH}, \ref{fig:channelmap_CH3CHO}, and \ref{fig:channelmap_CH3OCHO}) also show no significant emission at $\lesssim$ 40 au radius. 


\begin{figure*}
\epsscale{1.15}
\plotone{line_profile_comparison.pdf}
\caption{Line profiles of \methanol $2_{-1,1}$ -- $1_{-1,0}$ E $v_t=0$, \acetaldehyde $5_{2,3}$ -- $4_{2,2}$ E $v_t=0$, and \methylformate $7_{6,1}$ -- $6_{6,0}$ E $v_t=0$. The vertical gray  dotted line marks the systemic velocity (4.25 km s$^{-1}$). The vertical blue dashed lines indicate the corresponding disk radii at each velocity channel based on the maximum velocity of Keplerian rotation in Equation (\ref{eq:Keplerian_rotation}). The horizontal gray dashed line indicates the zero-flux level. There are no significant emission at the velocity channels which corresponds to inside $\sim$ 40\,au radius.}
\label{fig:line_profile}
\end{figure*}

To quantitatively assess the contribution of the emission at each radius, we used a forward modeling approach to reconstruct the radial intensity profiles from the line profiles. Using the relation in Equation (\ref{eq:Keplerian_rotation}), the radial intensity profile (as a function of radius) can be constructed from the line profile (as a function of velocity) as demonstrated in \citet{Bosman2021}. We used an approach similar to that of \citet{Bosman2021} to construct the radial intensity profiles of the aforementioned three bright transitions from their line profiles. Details of the modeling can be found in Appendix \ref{appendix:line_profile_analysis_method}. Figure \ref{fig:radial_profile_from_line_profile} shows the reconstructed radial intensity profiles of these three transitions. All three transitions show the depression in the innermost region with a radial peak at $\sim$ 50--60 au, which is consistent with the higher-resolution Band 6 profile \citep{Tobin2023}. This suggests that the bulk of the observed Band 3 emission can be explained by emission from the 40--80 au region, and that the inner emission cavity likely exists even in the Band 3 emission as well. 
% We note that, however, the uncertainty of the intensity profile is quite large. 

\begin{figure}
% \epsscale{1.15}
\plotone{radialProfile_from_lineProfile_Band6Overlay.pdf}
\caption{Reconstructed radial intensity profiles of \methanol (blue), \acetaldehyde (orange), and \methylformate (green) transitions. The horizontal gray dashed line indicates the zero-intensity level. The gray dashed curve shows the radial intensity profile of \methanol transition in Band 6 created by deprojection and azimuthal averaging on the higher-resolution ($\sim$ 0\farcs1) velocity-integrated intensity maps \citep{Tobin2023}. The Band 6 intensity profile is normalized by its radial peak being matched to the radial peak of the Band 3 \methanol profile.}
\label{fig:radial_profile_from_line_profile}
\end{figure}


\subsection{Column Density Retrieval}\label{subsec:spectral_fit}
To derive the disk-averaged column density of each species, we first employ a simultaneous fit of an LTE spectral model to the extracted spectra, taking into account the spectral blending and line optical depth. The same spectral model is used as for line identification (see Appendix \ref{appendix:spectral_model} for details), where all (tentatively) identified species as in Section \ref{subsec:line_identification} are considered. We excluded the optically thick ($\tau > 0.2$) transitions from the fit to minimize the optical depth effect; the selection is based on the optical depth at the line center in the best-fit model for line identification (Section \ref{subsec:line_identification}). Specifically, we excluded the transitions of abundant species (e.g., \methanol, \methylformate, and \acetaldehyde) with a large Einstein A coefficient and a relatively low upper state energy. \textcolor{blue}{The excluded transitions are indicated in Table \ref{tab:transitions} in Appendix \ref{appendix:transitions}}


% The emitting region area is fixed based on the previous Band 6 observations \citep{Tobin2023}. 
Since the emission is only marginally spatially resolved, we assumed that the emitting region in Band 3 is the same as that of \methanol in Band 6 \citep{Tobin2023}. The inner and outer radius of the emitting area are fixed to 0\farcs1 and 0\farcs3, respectively. 
% This facilitates the direct comparison of the column density of COMs. 
In addition to the (tentatively) identified transitions in Section \ref{subsec:line_identification}, we included several undetected species in the spectral fit: $^{13}$CH$_3$CHO, CH$_2$DOCHO, and CH$_3$OCDO. In total, we consider 24 free parameters: excitation temperature ($T_\mathrm{ex}$), line width ($\Delta V_\mathrm{FWHM}$), column density of 21 species, and an additional parameter $\gamma$ for line broadening (see Appendix \ref{appendix:spectral_model}). 
% In addition, we properly consider the flux calibration uncertainty and the data correlation between spectral channels originating from the Hanning smoothing applied in the correlator and the procedure to correct for the line broadening due to the Kepelerian rotation (Section \ref{subsec:line_identification}). The detailed procedure are described in Appendices. 
We used the affine-invariant Markov Chain Monte Carlo (MCMC) algorithm implemented in the \texttt{emcee} Python package \citep{emcee} to explore the parameter spaces. We run 100 walkers for 15000 steps, and the initial 12000 steps are discarded as burn-in. Figure \ref{fig:spectral_fit_demo} demonstrates the unique ability of the simultaneous spectral fit to fully exploit the transitions for column density derivation even with a severe spectral blending. 

The results of the fits are summarized in Table \ref{tab:specfit_result}. The derived excitation temperature is $106.7_{-3.8}^{+4.3}$ K, similar to the typical sublimation temperature of COMs ($\sim100$\,K), indicating the thermal sublimation of molecules. The column densities of COMs are $\sim$10$^{15}$--10$^{18}$ cm$^{-2}$, broadly in agreement with previous estimates using Band 7 data \citep{Lee2019}. 
% While both fits obtained similarly high excitation temperatures of $\sim100$\,K indicative of ice sublimation, the column density of molecules are significantly different between the two fits. In the fit with $R_\mathrm{s}$ being a free parameter, the emitting radius is smaller than that of in Band 6 (0\farcs3) by a factor of $\sim2$, and the column density of molecules are thus higher than that in the fit with a fixed emitting radius by factors of $\sim$ 4--8.

\begin{figure*}
\epsscale{1.15}
\plotone{spectral_fit_demo_v2.pdf}
\caption{Demonstration of the simultaneous spectral fit for a selected frequency range in SPW 7. The observed spectra (gray) is well reproduced by the model (black) composed of multiple blended transitions of different species, \methanol, \acetaldehyde, \acetone, \ethyleneoxide, and CH$_3^{13}$CHO (dashed colored lines).}
\label{fig:spectral_fit_demo}
\end{figure*}

% \begin{deluxetable}{lcc}
% \tablecaption{Results of the spectral fits}
% \tablehead{\colhead{Parameter} & \colhead{$R_\mathrm{s} = 0\farcs3$} & \colhead{$R_\mathrm{s}$ Free}}
% \startdata
% $R_\mathrm{s}$ ($^{\prime\prime}$) & 0.3 (fixed) & $0.148_{-0.004}^{+0.006}$ \\
% $T_\mathrm{ex}$ (K) & $100.4_{-2.8}^{+2.7}$ & $98.9_{-2.0}^{+2.0}$ \\
% $\Delta V_\mathrm{FWHM}$ (km s$^{-1}$) & $0.48_{-0.01}^{+0.01}$ & $1.20_{-0.09}^{+0.08}$ \\
% $\gamma$ (km s$^{-1}$) & $1.38_{-0.02}^{+0.02}$ & $1.15_{-0.03}^{+0.04}$ \\
% $N$ (CH$_3$OH) (cm$^{-2}$) & $2.91_{-0.24}^{+0.29} \times 10^{18}$ & $2.19_{-0.26}^{+0.27} \times 10^{19}$ \\
% $N$ (CH$_3$OCHO) (cm$^{-2}$) & $5.36_{-0.15}^{+0.15} \times 10^{17}$ & $4.16_{-0.34}^{+0.32} \times 10^{18}$ \\
% $N$ (CH$_3$CHO) (cm$^{-2}$) & $3.74_{-0.18}^{+0.21} \times 10^{17}$ & $2.55_{-0.21}^{+0.19} \times 10^{18}$ \\
% $N$ (CH$_3$OCH$_3$) (cm$^{-2}$) & $5.13_{-0.30}^{+0.31} \times 10^{17}$ & $2.35_{-0.25}^{+0.24} \times 10^{18}$ \\
% $N$ (CH$_3$COCH$_3$) (cm$^{-2}$) & $3.23_{-0.25}^{+0.26} \times 10^{16}$ & $1.33_{-0.14}^{+0.15} \times 10^{17}$ \\
% $N$ ($c$-C$_2$H$_4$O) (cm$^{-2}$) & $3.91_{-0.28}^{+0.31} \times 10^{16}$ & $1.83_{-0.24}^{+0.26} \times 10^{17}$ \\
% $N$ ($t$-C$_2$H$_3$CHO) (cm$^{-2}$) & $4.34_{-0.19}^{+0.20} \times 10^{15}$ & $2.15_{-0.19}^{+0.18} \times 10^{16}$ \\
% $N$ ($s$-C$_2$H$_5$CHO) (cm$^{-2}$) & $1.75_{-0.16}^{+0.16} \times 10^{16}$ & $6.31_{-0.80}^{+0.79} \times 10^{16}$ \\
% $N$ (CH$_2$DOH) (cm$^{-2}$) & $2.11_{-0.23}^{+0.24} \times 10^{16}$ & $8.58_{-1.15}^{+1.18} \times 10^{16}$ \\
% $N$ (CH$_3$O$^{13}$CHO) (cm$^{-2}$) & $2.99_{-0.23}^{+0.23} \times 10^{16}$ & $1.16_{-0.12}^{+0.13} \times 10^{17}$ \\
% $N$ (CH$_3^{13}$CHO) (cm$^{-2}$) & $7.70_{-1.10}^{+1.14} \times 10^{16}$ & $4.09_{-0.65}^{+0.69} \times 10^{17}$ \\
% $N$ (CH$_2$DCHO) (cm$^{-2}$) & $1.65_{-0.33}^{+0.35} \times 10^{16}$ & $9.40_{-2.03}^{+2.24} \times 10^{16}$ \\
% $N$ (CH$_3$CDO) (cm$^{-2}$) & $6.61_{-0.75}^{+0.77} \times 10^{15}$ & $2.64_{-0.40}^{+0.42} \times 10^{16}$ \\
% $N$ (C$_2$H$_3$CN) (cm$^{-2}$) & $1.92_{-0.39}^{+0.41} \times 10^{15}$ & $7.96_{-1.87}^{+2.11} \times 10^{15}$ \\
% $N$ (SO$_2$) (cm$^{-2}$) & $1.70_{-0.30}^{+0.31} \times 10^{16}$ & $1.35_{-0.28}^{+0.31} \times 10^{17}$
% \enddata
% \tablenotetext{\dagger}{Uncertainties are the 16th and 84th percentile of the posterior distributions, where only the statistical uncertainty is included.}
% \label{tab:specfit_result}
% \end{deluxetable}

\begin{deluxetable}{lccc}
\tablecaption{Column Density of Molecules}
\tablehead{\colhead{Molecule} & \colhead{$N$ (cm$^{-2}$)}}
\startdata
CH$_3$OH & $3.9_{-0.6}^{+0.7} \times 10^{18}$ \\ 
CH$_3$CHO & $6.5_{-0.5}^{+0.5} \times 10^{17}$ \\ 
CH$_3$OCHO & $7.4_{-0.3}^{+0.3} \times 10^{17}$ \\ 
CH$_3$OCH$_3$ & $5.7_{-0.5}^{+0.5} \times 10^{17}$ \\ 
CH$_3$COCH$_3$ & $4.4_{-0.5}^{+0.5} \times 10^{16}$ \\ 
$c$-C$_2$H$_4$O & $3.5_{-0.4}^{+0.5} \times 10^{16}$ \\ 
$t$-C$_2$H$_3$CHO & $5.0_{-0.3}^{+0.3} \times 10^{15}$ \\ 
$s$-C$_2$H$_5$CHO & $1.8_{-0.3}^{+0.3} \times 10^{16}$ \\ 
CH$_2$DOH & $2.2_{-0.4}^{+0.4} \times 10^{16}$ \\ 
CH$_3$CDO & $5.5_{-1.3}^{+1.3} \times 10^{15}$ \\ 
CH$_2$DCHO & $2.8_{-0.8}^{+0.8} \times 10^{16}$ \\ 
CH$_3^{13}$CHO & $2.3_{-0.8}^{+0.8} \times 10^{16}$ \\ 
$^{13}$CH$_3$CHO & $< 1.5 \times 10^{17}$ \\ 
CH$_3$O$^{13}$CHO & $3.2_{-0.3}^{+0.4} \times 10^{16}$ \\ 
$^{13}$CH$_3$OCHO & $3.2_{-0.3}^{+0.3} \times 10^{16}$ \\ 
CH$_2$DOCHO & $< 1.2 \times 10^{16}$ \\ 
CH$_3$OCDO & $< 1.4 \times 10^{16}$ \\ 
$^{13}$CH$_3$OCH$_3$ & $2.3_{-0.6}^{+0.6} \times 10^{16}$ \\ 
SO$_2$ & $3.0_{-0.7}^{+0.7} \times 10^{16}$ \\ 
OCS & $5.5_{-1.3}^{+2.0} \times 10^{16}$ \\ 
C$_2$H$_3$CN & $2.2_{-0.9}^{+0.9} \times 10^{15}$ \\ 
\enddata
\tablenotetext{\dagger}{The excitation temperature and line width, which are assumed to be common for all molecules, are estimated to be $106.8_{-3.9}^{+4.2}$\,K and $0.71_{-0.05}^{+0.07}$\,km\,s$^{-1}$, respectively.}
\tablenotetext{\ddagger}{Uncertainties are the 16th and 84th percentile of the posterior distributions, where only the statistical uncertainty is included. The upper limits are determined based on the 99.7th percentile of the posterior distributions.}
\label{tab:specfit_result}
\end{deluxetable}



% \subsubsection{Integrated Intensity Analysis}
% % The spectral analysis described in the previous section uses the spectra aligned for Keplerian rotation. As the emission is only marginally spatially resolved, this procedure of the spectral alignment might have some additional uncertainty 
% To validate the results of the spectral analysis described in Section \ref{subsubsec:spectral_fit}, we also complementarily employed the analysis of the velocity-integrated intensities \citep[e.g.,][]{Goldsmith1999} for three species (\methanol, \methylformate, and \acetaldehyde) whose transitions without line blending are detected in a sufficient number for this analysis. Following \citet{Goldsmith1999}, the column density of molecules in the upper state of each transition under the assumption of optically thin emission, $N_\mathrm{u}^{\mathrm{thin}}$, can be related to the disk-integrated flux density $S_\nu$,
% \begin{equation}
%     N_\mathrm{u}^\mathrm{thin} = \frac{4\pi S_\nu \Delta v}{A_\mathrm{ul}\Omega hc},
% \end{equation}
% where $A_\mathrm{ul}$ is the Einstein A coefficient, $\Delta v$ is the line width, and $\Omega$ is the solid angle of the emitting region. In the case where the optical depth is not negligible ($\tau \nll 1$), the upper state column density $N_\mathrm{u}^\mathrm{thin}$ needs to be scaled by an optical depth correction factor,
% \begin{equation}\label{eq:tau_correction_factor}
%     C_\tau = \frac{\tau}{1 - e^{-\tau}},
% \end{equation}
% to obtain the correct upper state column density, i.e., $N_\mathrm{u} = N_\mathrm{u}^\mathrm{thin} C_\tau$ \citep{Goldsmith1999}. The upper state column density $N_\mathrm{u}$ is then related to the total column density $N$ by the Boltzmann equation,
% \begin{equation}
%     \frac{N_\mathrm{u}}{g_\mathrm{u}} = \frac{N}{Q(T_\mathrm{ex})}e^{-E_\mathrm{u} / k_\mathrm{B} T_\mathrm{ex}},
% \end{equation}
% where $g_\mathrm{u}$ is the upper state degeneracy of each transition, $Q$ is the partition function of the molecule, $T_\mathrm{ex}$ is the excitation temperature, and $E_\mathrm{u}$ is the upper state energy of each transition. Taking the logarithm of this equation yields the conventional rotation diagram as in \citet{Goldsmith1999},
% \begin{equation}
%     \ln\frac{N_\mathrm{u}}{g_\mathrm{u}} = \ln N - \ln Q(T_\mathrm{ex}) - \frac{E_\mathrm{u}}{k_\mathrm{B}T_\mathrm{ex}},
% \end{equation}
% or, expressing the optical depth correction factor explicitly,
% \begin{equation}\label{eq:rotation_diagram}
%     \ln\frac{N_\mathrm{u}^\mathrm{thin}}{g_\mathrm{u}} + \ln C_\tau = \ln N - \ln Q(T_\mathrm{ex}) - \frac{E_\mathrm{u}}{k_\mathrm{B}T_\mathrm{ex}}.
% \end{equation}
% The optical depth $\tau$ can be related back to the upper state column density of each transition by 
% \begin{equation}\label{eq:optical_depth}
%     \tau = \frac{c^3A_\mathrm{ul}N_\mathrm{u}}{8\pi\nu^3\Delta V_\mathrm{FWHM}}\left(e^{h\nu/k_\mathrm{B}T_\mathrm{ex}} - 1\right),
% \end{equation}
% which makes it possible to write down a likelihood function $\mathcal{L}(N, T_\mathrm{ex})$ from Equation (\ref{eq:rotation_diagram}) to optimize these parameters. 

% To conduct this analysis, we first correctly measured the velocity-integrated flux density ($S_\nu\Delta v$) of each transition by applying a Keplerian mask to the image cubes. Since even the Keplerian masking could not resolve the spectral blending due to the limited spatial/spectral resolution, blended transitions were excluded from the measurement except for two blended \methanol transitions $21_{6,16}$ -- $22_{5,17}$ A $v_t=0$ and $21_{6,15}$ -- $22_{5,18}$ A $v_t=0$, each of whose velocity-integrated flux density is measured by dividing the sum of these transitions by two as the intrinsic strength of these transitions are same (see Table \ref{tab:flux_density}). These transitions are critical for this analysis as the number of optically thin \methanol transitions are insufficient (one) if these transitions were excluded. To construct the Keplerian mask, we used the aforementioned disk parameters (see Section \ref{subsec:line_identification}) and an inner radius and outer radius of 0\farcs05 and 0\farcs3, respectively. The Keplerian mask is convolved with the observing beam, and then used to extract the spectrum. We confirmed that the beam-convolved Keplerian mask is apparently encompassing the observed emission. The extracted spectra for each transition are shown in Figure \ref{fig:CH3OH_spectra}, \ref{fig:CH3OCHO_spectra}, and \ref{fig:CH3CHO_spectra} in Appendix \ref{appendix:spectra}. The uncertainty of the spectra is calculated at per-channel basis as $\sqrt{\Omega_\mathrm{mask}/\Omega_\mathrm{beam}}\times\sigma_\mathrm{RMS}$, where $\Omega_\mathrm{mask}$ is the solid angle of the Keplerian mask in each channel, $\Omega_\mathrm{beam}$ is the solid angle of the beam, and $\sigma_\mathrm{RMS}$ is the channel RMS reported in Table \ref{tab:cube_properties}. The velocity-integrated flux densities were measured by integrating the spectra along the velocity axis, i.e., $S_\nu\Delta v = \sum_i S_{\nu, i}\,\Delta v_\mathrm{chan}$, where $i$ is the index for each channel and $\Delta v_\mathrm{chan}$ is the channel width. The uncertainties of the flux densities were calculated as $\sqrt{\Delta v_\mathrm{res}/\Delta v_\mathrm{chan}}\,\times\sqrt{\sum_i (dS_{\nu, i})^2}\,\Delta v_\mathrm{chan}$, where $\Delta v_\mathrm{res}$ and $dS_{\nu, i}$ are the velocity resolution (Table \ref{tab:cube_properties}) and the uncertainty of the spectra, respectively. The additional factor of $\sqrt{\Delta v_\mathrm{chan}/\Delta v_\mathrm{res}}$ accounts for the correlation between channels due to the narrower channel width than the velocity resolution. The measured flux densities are reported in Table \ref{tab:flux_density} with spectroscopic data of each transition.

% For the solid angle of emitting region $\Omega$ and the line width $\Delta V_\mathrm{FWHM}$ in Equation (\ref{eq:optical_depth}), we assumed the same values in the spectral analysis. The inner and outer radii of the emitting region size are $0\farcs1$ and $0\farcs3$, respectively, and the line width is fixed to the best-fit value of the spectral analysis (0.7 km s$^{-1}$).

% % We first tried to fit using all transitions listed in Table \ref{tab:flux_density}, from which we excluded highly optically thick transitions ($\tau \gg 1$) as they could trace the different temperature regions (e.g., vertical temperature gradient) and were not fitted well even with the optical depth correction factor (Equation \ref{eq:tau_correction_factor}). 
% The same transitions as those in the spectral analysis (Section \ref{subsubsec:spectral_fit}) are excluded from the fit as well for the same reason (optical depth). The excluded transitions are marked in Table \ref{tab:flux_density}. We sampled the posterior probability distributions of $N$ and $T_\mathrm{ex}$ using the \texttt{emcee} package \citep{emcee}. For both fits, we run 32 walkers for 1000 steps, and the initial 500 steps are discarded as burn-in. The $N_\mathrm{u}/g_\mathrm{u}$ values of the observed transitions are plotted against $E_\mathrm{u}$ in Figure \ref{fig:rotdiag_0.3arcsec} and \ref{fig:rotdiag_0.15arcsec} with models generated from ($N, T_\mathrm{ex}$) pairs randomly selected from the posterior distributions.


% \begin{deluxetable*}{lLLCCC}
% \tablecaption{Spectroscopic Data and Flux Density of Unblended Transitions}
% \tablehead{\colhead{Transition} & \colhead{$\nu_0$} & \colhead{log$_{10}A_\mathrm{ul}$} & \colhead{$g_\mathrm{u}$} & \colhead{$E_\mathrm{u}$} & \colhead{$S_\nu\Delta v^\mathrm{a}$} \\
% \colhead{} & \colhead{(GHz)} & \colhead{(s$^{-1}$)} & \colhead{} & \colhead{(K)} & \colhead{(mJy km s$^{-1}$)}}
% \startdata
% \hline 
% \multicolumn{6}{c}{\methanol} \\
% \hline 
% $2_{0,2}$ -- $1_{0,1}$ A $v_t=1$ & 96.513686 & -5.4709 & 20.0 & 430 & 40.2 \pm 3.4 \\
% $2_{0,2}$ -- $1_{0,1}$ E $v_t=0$ & 96.744545 & -5.4676 & 20.0 & 20 & 150 \pm 3.3 \\
% $2_{-1,1}$ -- $1_{-1,0}$ E $v_t=0$ & 96.755501 & -5.581 & 20.0 & 28 & 135 \pm 3.3 \\
% $2_{1,1}$ -- $1_{1,0}$ A $v_t=0$ & 97.582798 & -5.5807 & 20.0 & 21 & 155 \pm 2.9 \\
% $21_{6,16}$ -- $22_{5,17}$ A $v_t=0$$^\dagger$ & 97.677684 & -5.8404 & 172.0 & 729 & 18 \pm 1.9$^\dagger$ \\
% $21_{6,15}$ -- $22_{5,18}$ A $v_t=0$$^\dagger$ & 97.678803 & -5.8403 & 172.0 & 729 & 18 \pm 1.9$^\dagger$ \\
% \hline 
% \multicolumn{6}{c}{\methylformate} \\
% \hline 
% $7_{6,1}$ -- $6_{6,0}$ E $v_t=1$ & 85.157135 & -5.6248 & 30.0 & 228 & 22.8 \pm 3.6 \\
% $7_{4,3}$ -- $6_{4,2}$ E $v_t=1$ & 85.506219 & -5.2142 & 30.0 & 214 & 34.5 \pm 3.4 \\
% $7_{5,3}$ -- $6_{5,2}$ E $v_t=1$ & 85.55338 & -5.3527 & 30.0 & 220 & 26.7 \pm 3.4 \\
% $7_{6,1}$ -- $6_{6,0}$ E $v_t=0$ & 85.919209 & -5.6138 & 30.0 & 40 & 36.3 \pm 3.3 \\
% $7_{4,4}$ -- $7_{3,5}$ E $v_t=0$ & 96.507882 & -6.1542 & 30.0 & 27 & 28.6 \pm 3.4 \\
% $8_{4,5}$ -- $8_{3,6}$ A $v_t=0$ & 96.709259 & -5.9454 & 34.0 & 31 & 33 \pm 3.3 \\
% $7_{4,3}$ -- $7_{3,5}$ E $v_t=0$ & 96.776715 & -6.4523 & 30.0 & 27 & 16.7 \pm 3.3 \\
% $5_{4,2}$ -- $5_{3,3}$ A $v_t=0$ & 96.794121 & -6.1026 & 22.0 & 19 & 27.2 \pm 3.3 \\
% $8_{5,3}$ -- $7_{5,2}$ E $v_t=1$ & 97.577303 & -5.0823 & 34.0 & 225 & 34.2 \pm 2.9 \\
% $8_{3,6}$ -- $7_{3,5}$ A $v_t=1$ & 97.597161 & -4.9359 & 34.0 & 214 & 43.4 \pm 2.9 \\
% $8_{4,5}$ -- $7_{4,4}$ A $v_t=1$ & 97.661401 & -4.9937 & 34.0 & 219 & 42.2 \pm 2.9 \\
% $8_{6,3}$ -- $7_{6,2}$ E $v_t=1$ & 97.738738 & -5.2243 & 34.0 & 232 & 33.7 \pm 3 \\
% $8_{4,4}$ -- $7_{4,3}$ A $v_t=1$ & 97.752885 & -4.9924 & 34.0 & 219 & 38.8 \pm 2.9 \\
% $8_{5,4}$ -- $7_{5,3}$ E $v_t=1$ & 97.885663 & -5.0787 & 34.0 & 224 & 28.5 \pm 3 \\
% $8_{4,4}$ -- $7_{4,3}$ E $v_t=1$ & 97.897118 & -4.9875 & 34.0 & 219 & 35.4 \pm 3.1 \\
% $8_{7,1}$ -- $7_{7,0}$ E $v_t=0$ & 98.182336 & -5.4902 & 34.0 & 53 & 51.8 \pm 2.9 \\
% $8_{6,2}$ -- $7_{6,1}$ E $v_t=0$ & 98.270501 & -5.218 & 34.0 & 45 & 71.3 \pm 3 \\
% $8_{5,3}$ -- $7_{5,2}$ A $v_t=0$ & 98.435802 & -5.0719 & 34.0 & 37 & 82 \pm 3 \\
% $8_{4,5}$ -- $7_{4,3}$ E $v_t=0$ & 98.443186 & -6.4657 & 34.0 & 31 & 23.1 \pm 2.9 \\
% \hline 
% \multicolumn{6}{c}{\acetaldehyde} \\
% \hline 
% $9_{1,8}$ -- $9_{0,9}$ E $v_t=1$ & 85.947624 & -5.559 & 38.0 & 249 & 19.7 \pm 3.4 \\
% $5_{2,3}$ -- $4_{2,2}$ E $v_t=0$ & 96.475524 & -4.6168 & 22.0 & 23 & 96.5 \pm 3.4 \\
% $11_{4,8}$ -- $12_{3,10}$ E $v_t=0$ & 96.48895 & -6.1541 & 46.0 & 97 & 33.5 \pm 3.3 \\
% $7_{0,7}$ -- $6_{1,6}$ A $v_t=0$ & 96.765371 & -5.5556 & 30.0 & 25 & 60.1 \pm 3.3 \\
% $5_{2,4}$ -- $4_{2,3}$ E $v_t=1$ & 96.800291 & -4.613 & 22.0 & 226 & 65.8 \pm 3.3 \\
% $5_{3,2}$ -- $4_{3,1}$ E $v_t=2$ & 97.612131 & -4.9808 & 22.0 & 419 & 17.9 \pm 2.9 \\
% $19_{2,18}$ -- $18_{3,15}$ E $v_t=0$ & 97.796104 & -6.1248 & 78.0 & 183 & 26.8 \pm 2.9 \\
% $10_{4,7}$ -- $11_{3,8}$ A $v_t=1$ & 97.941422 & -6.1583 & 42.0 & 290 & 8.53 \pm 3 \\
% $21_{3,18}$ -- $20_{4,17}$ A $v_t=0$ & 98.20169 & -6.01 & 86.0 & 235 & 25.6 \pm 3 \\
% $6_{3,3}$ -- $7_{2,5}$ E $v_t=0$ & 98.368631 & -6.245 & 26.0 & 39 & 31.5 \pm 2.9
% \enddata
% \tablenotetext{\dagger}{Blended each other. The reported flux density is the sum of these transitions' divided by two.}
% \tablenotetext{\ddagger}{Excluded from rotation diagram analysis due to the high optical depth.}
% \tablenotetext{\mathrm{a}}{Reported errors only account for the statistical uncertainty.}
% \label{tab:flux_density}
% \end{deluxetable*}



% \begin{figure*}
% \epsscale{1.15}
% \plotone{rotation_diagram_size0.3arcsec.pdf}
% \caption{Rotation diagram of \methanol, \methylformate, and \acetaldehyde with an emitting radius of $R_\mathrm{s} = 0\farcs3$ and a line width of $\Delta V_\mathrm{FWHM} = 0.5$ km s$^{-1}$. The black points denotes the observed values with the correction for optical depth. The individual blue lines indicates the models generated from the samples randomly selected from the posterior distributions. The median values of the posterior distributions of $N$ and $T_\mathrm{ex}$ with associated uncertainties (16th and 84th percentiles of the posterior distributions) are shown in the lower-left corner. These uncertainties only account for the statistical uncertainty.}
% \label{fig:rotdiag_0.3arcsec}
% \end{figure*}

% % \begin{figure*}
% % \epsscale{1.15}
% % \plotone{rotation_diagram_size0.15arcsec.pdf}
% % \caption{Same as Figure \ref{fig:rotdiag_0.3arcsec}, but with an emitting radius of $R_\mathrm{s} = 0\farcs15$ and a line width of $\Delta V_\mathrm{FWHM} = 1.2$ km s$^{-1}$.}
% % \label{fig:rotdiag_0.15arcsec}
% % \end{figure*}

% The rotation diagram and the estimated column density and excitation temperature for each fit are shown in Figure \ref{fig:rotdiag_0.3arcsec} and \ref{fig:rotdiag_0.15arcsec}. For all fits, the column densities are consistent with the values derived from the spectral analysis (Section \ref{subsubsec:spectral_fit}), confirming that these methods are robust. The excitation temperatures are also consistent ($\sim100\,\mathrm{K}$) within the uncertainties among these molecules, supporting the assumption of a common excitation temperature in the spectral analysis (Section \ref{subsubsec:spectral_fit}) at least for \methanol, \methylformate, and \acetaldehyde. 



\section{Discussion} \label{sec:discussion}
\subsection{Comparison of the spatial distributions with the sub-mm observations}
% \subsubsection{Spatial Distributions}\label{subsubsec:spatial_dist}
The spatial distributions of the COM emission in Band 3, where the dust continuum emission would be fainter than in the sub-mm regime, are essential information to infer the origin of the inner cavity seen in the sub-mm observations. If the central region where the emission cavity is seen in Band 6/7 is filled with COM emission in Band 3, it would be evidence that the COM emission is reduced in the sub-mm observations due to the intense dust continuum emission. 

We found that while the spatial distributions of the COM emission in Band 3 are consistently centrally-peaked in velocity-integrated intensity maps (Figure \ref{fig:moment_zero_gallery}), the reconstructed radial intensity profiles (Figure \ref{fig:radial_profile_from_line_profile}) show the inner depression of the emission. This indicates that the inner emission cavity is smeared out by the beam on the velocity-integrated intensity maps. We confirmed that the inner emission cavity of a \methanol line in Band 6\footnote{Taken from the Harvard Dataverse repository (\url{https://dataverse.harvard.edu/dataset.xhtml?persistentId=doi:10.7910/DVN/MDQJEU)})} \citep{Tobin2023} is smeared out when the image is smoothed to the same spatial resolution as that of the presented Band 3 data.
% We confirmed that the reconstructed radial intensity profiles (Figure \ref{fig:radial_profile_from_line_profile}) and the radial intensity profile of \methanol in Band 6\footnote{Created by deprojecting and azimuthally averaging the velocity-integrated intensity map taken from the Harvard Dataverse repository (\url{https://dataverse.harvard.edu/dataset.xhtml?persistentId=doi:10.7910/DVN/MDQJEU)})} \citep{Tobin2023} are both centrally peaked and consistent with those created from the velocity-integrated intensity maps (Figure \ref{fig:moment_zero_gallery}) when they are smoothed to the same spatial resolution as that of the presented Band 3 data. 
Although the uncertainties of the reconstructed radial intensity profiles are quite large, the profiles all have radial peaks at $\sim$ 50--60\,au, which is consistent with the higher-resolution Band 6 profile (Figure \ref{fig:radial_profile_from_line_profile}). Therefore, it is likely that the emission observed in Band 3 mainly traces the components outside the emission cavity ($\gtrsim 40\,$ au, i.e., the same component as that traced by the sub-mm observations). 

We note, however, that there are some additional systematic uncertainties in the reconstructed radial intensity profiles (Figure \ref{fig:radial_profile_from_line_profile}). In reality, the molecular line emission could originate from the disk surface \citep[or the warm molecular layer,][]{Aikawa2002, Law2021, Law2022, Law2023, PanequeCarreno2023}, where the Keplerian velocity is slightly smaller than that in the midplane. Indeed, the COM emission in the V883 Ori disk also shows evidence of the elevated emission surfaces in the higher-resolution Band 6 data (M. Leemker et al. in prep.). This invalidates the assumption made in the reconstruction of the radial intensity profiles that the emission comes entirely from the disk midplane. Therefore, while the depression of the emission in the innermost region is robust, the actual emission cavity sizes in Band 3 emission are rather uncertain. In addition, transitions with different optical depths or upper state energies, which could trace the different disk heights, may have different emission distributions. 
% To fully reveal the emission distribution and COM chemistry in the innermost region of the V883 Ori disk, we need higher-resolution observations in Band 3 and even longer wavelengths (e.g., Band 1, Karl G. Jansky Very Large Array (VLA), and next generation Very Large Array (ngVLA)) and/or the detailed combined analysis of the line and continuum emission.
% Furthermore, there could be a deviation from the Keplerian rotation in the innermost region, which again complicates the reconstruction process. 

% which is different from the sub-mm observations \citep{vantHoff2018, Lee2019, Tobin2023}. While this may indicate that the Band 3 observations traces the emission from the cavity region in sub-mm, the spatial resolution of the presented Band 3 observations is insufficient to conclude so. We compare the radial intensity profile of \methanol emission along the disk major axis (P.A. $= 32\arcdeg$) in Band 3 and Band 6 in Figure \ref{fig:radial_profile_CH3OH} (The Band 6 data is based on \citealt{Tobin2023}\footnote{Taken from the Harvard Dataverse repository \url{https://dataverse.harvard.edu/dataset.xhtml?persistentId=doi:10.7910/DVN/MDQJEU}}). The Band 6 intensity profile is created from the higher resolution data ($\sim0\farcs1$ beam) than the Band 3 data. The Band 6 data was smoothed to the same beam size as that in Band 3 using the CASA task \texttt{imsmooth} to facilitate the direct comparison to the Band 3 data. The direction of the disk major axis is almost same as the direction of the minor axis of the Band 3 beam, which has the maximum potential to spatially resolve the COM emission. As seen in Figure \ref{fig:radial_profile_CH3OH}, even the Band 6 data which shows the inner cavity with the original resolutions \citep{Tobin2023} shows a centrally-peaked profile with the Band 3 beam, which is quite similar to that of the Band 3 data. These similar profiles indicate that the COM emission in both Band 6 and Band 3 traces the same component, or at least there are no predominant component from the cavity region in the Band 3 data.

% We also examined the channel map of the COM emission in Band 3 (Figure \ref{fig:channelmap_CH3OH}, \ref{fig:channelmap_CH3OCHO}, and \ref{fig:channelmap_CH3CHO} in Appendix \ref{appendix:channel_maps}) to search for the potential signals from the cavity region. Assuming that the COM gas follows the Keplerian rotation, the high-velocity components purely traces the emission from the inner region, allowing for the direct inspection of more inner region than that explored by the velocity-integrated maps. The channel maps of the brightest transitions of \methanol, \methylformate, and \acetaldehyde, which should be most sensitive to the innermost region with a given observing sensitivity, are shown as representatives. The channel map shows that the emission is mainly originating from the velocity range of $v_\mathrm{sys} \pm 3.5$\,km\,s$^{-1}$, which corresponds to the radius of $\gtrsim 40$ au on the assumption that the emission completely follows the Keplerian rotation, although faint emission ($\sim3\sigma$) are seen in a few channels which correspond to 20--30 au. 

% In summary, it is likely that the emission observed in the Band 3 observations mainly traces the components outside the cavity ($\gtrsim 40\,$ au, i.e., the same component as that traced by the sub-mm observations) with a potential small additional contribution from more inner region. 


% \begin{figure}
% % \epsscale{1.15}
% \plotone{radial_profile_CH3OH.pdf}
% \caption{Normalized radial intensity profiles of the \methanol emission along the disk major axis (P.A. $= 32\arcdeg$) in Band 3 (blue) and Band 6 (orange). The intensity profile of the Band 6 data is created from the velocity-integrated intensity map smoothed to the same beam size as that of the Band 3 data and shown by the dark-orange line. The original beam size of the Band 6 data is $0\farcs098\times0\farcs078$, and the original intensity profile is shown by the light-orange line. These profiles are normalized by the radial peaks. The color-shaded region indicates the 1$\sigma$ scatter in each radial bin. The beam profile of the Band 3 data along the disk major axis is shown in the black dotted line. The horizontal gray dashed line marks the zero-intensity level.}
% \label{fig:radial_profile_CH3OH}
% \end{figure}





% \subsubsection{Column Densities}
% Disk-averaged column density is an alternative probe of the emission cavity, even without directly spatially resolving the cavity region. If the Band 3 observations trace more inner region in addition to the component observed in sub-mm (i.e., outside the cavity), the disk-averaged column densities of COMs based on the Band 3 data should be systematically larger than those based on sub-mm data. With the presented Band 3 observations, we found no systematic increases in column densities of COMs compared to those based on the sub-mm observations (Table \ref{tab:specfit_result}). This naturally indicates that the Band 3 observations traces the same emission component with that traced by the sub-mm observations, consistent with the implication from the spatial distributions (Section \ref{subsubsec:spatial_dist}). 

% \textcolor{red}{Ask Jeong-Eun for the latest column density estimate from Band 6}

% \textcolor{red}{Note for potential uncertainty in the column density? Missing flux etc.}

\subsection{The physical and chemical structure of the innermost region of the V883 Ori disk}\label{subsec:physicochemical_structure}
Our line profile analyses suggest that the COM line emission in Band 3 also shows the inner emission cavity at $\lesssim$ 40 au radius. Here we discuss the possible origins of the emission cavity and the associated physical and chemical structures of the V883 Ori disk. 


% \subsubsection{Bright dust continuum emission hiding the molecular lines}
The first possible explanation for the emission cavity of the COM emission in Band 3 is that the dust continuum emission is still too intense even in Band 3 to detect the line emission in the innermost disk. In the sub-mm wavelengths, the dust continuum emission at $\lesssim$ 40 au radius has been shown to be optically thick ($\tau \gtrsim 2$) by the intra-band analysis of the ALMA Band 6 observations with a spatial resolution of $\sim0\farcs04$ \citep{Cieza2016}. Since the dust opacity is generally smaller in lower frequencies, the dust continuum emission is expected to be more optically thin in the longer wavelength regime. 

However, the factors that affect the line intensity are not only the dust optical depth, but also the temperature of the dust emitting layer. Consider a simple slab model for the vertical structure of a disk. If the line-emitting layer is well separated from the dust-emitting layer (near the midplane) and is closer to the observer \citep[see e.g.,][]{Bosman2021}, the observed line intensity after the continuum subtraction is expressed as 
\begin{align}\label{equation:line_intensity_separated}
    I_\mathrm{L-C} &= B_\nu(T_\mathrm{gas})(1 - e^{-\tau_\mathrm{line}}) + I_{\nu, \mathrm{dust}}e^{-\tau_\mathrm{line}} - I_{\nu, \mathrm{dust}} \notag \\
    &= (B_\nu(T_\mathrm{gas}) - I_{\nu, \mathrm{dust}})(1 - e^{-\tau_\mathrm{line}}),
\end{align}
where $B_\nu$ is the Planck function for blackbody radiation, $T_\mathrm{gas}$ is the gas temperature representing the temperature of the line emitting region, $\tau_\mathrm{line}$ is the optical depth of the line emission at the line center, and $I_{\nu, \mathrm{dust}} = \chi B_\nu(T_\mathrm{dust})(1 - e^{-\tau_\mathrm{dust}})$ is the intensity of the dust continuum emission. Here $\chi$ is the intensity reduction factor due to the scattering effect \citep[e.g.,][]{Bosman2021}, $T_\mathrm{dust}$ is the dust temperature, and $\tau_\mathrm{dust}$ is the optical depth of the dust continuum emission. Equation (\ref{equation:line_intensity_separated}) is an extension of the situation discussed in \citet{Bosman2021}, where the dust continuum emission is optically thick (i.e., $1 - e^{-\tau_\mathrm{dust}} \sim 1$) and the temperature of the line- and dust-emitting region is the same (i.e., $T_\mathrm{gas} = T_\mathrm{dust}$, vertically isothermal disk).
% If the dust continuum emission is optically thick (i.e., $1 - e^{-\tau_\mathrm{dust}} \sim 1$) and the temperature of the line and dust emitting region is the same (i.e., $T_\mathrm{gas} = T_\mathrm{dust}$, vertically isothermal disk), Equation (\ref{equation:line_intensity_separated}) will be reduced to the case discussed in \citet{Bosman2021}.

In archetypal disks where the temperature structure is mainly determined by the passive irradiation from the central star, the temperature in the line-emitting region ($T_\mathrm{gas}$) is usually much higher than the temperature in the disk midplane, from which the dust emission is mostly originating ($T_\mathrm{dust}$), allowing us to observe the line emission originating from the disk surface. However, in the case of the V883 Ori disk, where the disk temperature is expected to be elevated due to the accretion outburst, viscous heating may be efficient in the midplane, which could lead to a higher temperature in the disk midplane ($T_\mathrm{dust}$) higher. Indeed, \citet{Lee2019} suggest that the intensity depression of the COM emission can be reproduced by the presence of the additional heating in the midplane. Although the intensity of the dust emission also depends on the scattering effect ($\chi$), this warmer temperature in the midplane makes it even more difficult to detect the line emission originating from its emitting layer closer to the observer, even if the optical depth of the dust continuum emission is not very high, 

While the emission surface analysis of the Band 6 data (M. Leemker et al. in prep.) indeed suggests that the line-emitting layer is separated and elevated from the dust emitting layer in the disk midplane, this analysis is based on the emission from the outer region ($\gtrsim$ 40 au radius).
The bulk of the line emission in the innermost region ($\lesssim$ 40 au radius) may be well mixed with the dust emission and originate from the disk midplane. In this case, the observed line intensity after the continuum subtraction will completely disappear if the line opacity is negligible compared to the dust opacity (see \citealt{Bosman2021}, Equation (4)). 

% \textcolor{red}{Discussion on the difference between HDO and CH3OH? Since water is formed in gas-phase while CH3OH is not, the similarity of the HDO and CH3OH profiles may further support the dust absorption origin. But this can be done only by Band 6 data...}

% \textcolor{red}{Should we show the comparison of dust continuum brightness temperature in Band 3 and Band 6 at higher resolution?}

Another possible explanation for the inner emission cavity is that the molecules at $\lesssim40$ au radius are destroyed by the strong UV/X-ray radiation from the central protostar. The UV radiation can destroy COMs via photodissociation \citep[e.g.,][]{Garrod2006, Oberg2009}. This UV effect will be generally effective only in low-density regions (e.g., disk surfaces) due to the 
extinction of dust grains.
On the other hand, X-rays can penetrate into higher-density regions than UV radiation.
Recently, \citet{Notsu2021} modeled the effect of the X-ray radiation from the central protostar on the chemistry of the protostellar envelopes, and found that the gas-phase fractional abundances of \methanol and other COMs within their snowlines decrease as the X-ray luminosity of the central protostar increases. In the presence of the strong X-ray radiation, COMs are mainly destroyed by the X-ray-induced photodissociation \citep[e.g.,][]{Garrod2006, Taquet2016, Notsu2021}. 
% While the efficiency of the destruction of the molecules could be lower in higher-density disks than the lower-density envelopes \citep{Notsu2021}, a similar mechanism may work in the innermost region of the V883 Ori disk, where the X-ray/UV luminosity can also be increased due to the outburst of the central star, in particular in the low-density disk surface.
In the \citet{Notsu2021} model, \methanol is destroyed in $\sim\,10^3$\,yr when the X-ray ionization rate is $\sim10^{-13}$\,s$^{-1}$ in the infalling envelope. Here we use the ionization rate rather than the flux, since the former is the direct input parameter for the chemical reaction network models. While the destruction timescale may weakly depend on the gas density, which could be different between the inner envelopes and the $\lesssim40$\,au radius region of the V883~Ori disk, \methanol and other COMs could be destroyed in $\sim100$ yr if the X-ray ionization rate is $\gtrsim 10^{-12}$\,s$^{-1}$.
We note that \citet{Kuhn2019} reported that X-ray luminosities and plasma temperature of X-ray radiation in FU Ori type stars tend to be larger than in typical non-bursting YSOs, based on the comparison with a sample of low-mass stars in the Orion Nebula Cluster.
In addition, cosmic-rays accelerated near the central star (in strongly magnetized shocks along the outflow or in accretion shocks near the stellar surface) could penetrate into the higher-density disk midplane and cause the destruction of COMs via cosmic-ray-induced photodissociation \citep[e.g.,][]{Padovani2020, Cabedo2023}.

%\textcolor{red}{Any comments/additional discussions? Are there any references that should be cited here? Feel free to edit the sentences in the paragraph just above... $\to$ Notsu-san}

In summary, the inner cavity of the COM emission in Band 3 can be explained by the absorption of the molecular line emission by the bright dust continuum emission, and/or by the chemical destruction of the molecules. Our ALMA Band 3 observations suggest that the observations at longer wavelengths, where the dust continuum emission should be fainter, with ALMA Band 1, Karl G. Jansky Very Large Array (VLA), and the future next generation Vary Large Array (ngVLA), are essential to directly probe the chemistry of the innermost region of the V883 Ori disk.

\subsection{Chemical abundance ratios of COMs}\label{subsec:abundance_ratio}
% FU Ors are unique targets for astrochemistry, since we can observe fresh sublimates which are hidden in ice in the quiescent phase of the disk. 
The V883 Ori disk is a unique and ideal target for disk chemistry since we can observe fresh sublimates and probe the chemical composition of COMs which is hidden in typical colder disks. Here we discuss the chemical composition of the COMs detected in our observations and their implications for the chemical evolution during star and planet formation. 
Figure \ref{fig:ratio_comparison} compares the column density ratios of COMs with respect to \methanol in the V883 Ori disk with those in different evolutionary stages, including the warm inner envelopes of the Class 0 protobinary IRAS 16293-2422 A/B \citep[][see also \citealt{Drozdovskaya2019}]{Lykke2017, Jorgensen2018, Manigand2020, Manigand2021}, the protoplanetary disk around the Herbig Ae star Oph-IRS~48 \citep{Brunken2022}, and the solar system comet 67P/C-G \citep[][see also \citealt{Drozdovskaya2019}]{Rubin2019, Schuhmann2019}. IRAS 16293-2422 is a famous ``hot corino'' source, extensively studied in the ALMA Protostellar Interferometric Line Survey \citep[PILS;][]{Jorgensen2016}, where we can observe COM-rich warm gas sublimated from ices. The Oph-IRS~48 disk is the only disk so far to show emission of multiple COM species. A localized COM emission offset from the disk center has been observed, which is interpreted to be sublimated from the icy dust mantle stirred up to the warm disk surface \citep{vanderMarel2021, Brunken2022}. The comet 67P/C-G is also a well-known object whose chemical composition has recently been extensively studied by in-situ measurements in the Rosetta mission \citep[e.g.,][]{Altwegg2019}. We found higher abundance ratios of COMs in the V883 Ori disk compared to those in the warm envelopes of IRAS 16293-2422 by a factor of $\gtrsim$ 5--10, indicating that the abundance of chemically complex species is enhanced in protoplanetary disks. The COM abundance ratios in the V883 Ori disk are similar to those in another disk Oph-IRS~48 and in comet 67P/C-G, except for one species, \methylformate, which shows a lower abundance ratios in the comet compared to the disks. These results suggest that the more complex species than \methanol efficiently form during the evolution from the protostellar envelope to the protoplanetary disks, and that the chemical composition of disks can be inherited to comets without drastic chemical reprocessing. 

In general, the formation of COMs are thought to occur mainly via reactions on dust grain surfaces \citep[][and references therein]{Herbst2009}. In the cold temperature region ($\lesssim 10$\,K), the \methanol ice forms via efficient hydrogenation of the CO ice \citep[e.g.,][]{Tielens1982, Watanabe2002, Watanabe2003}. Therefore, the \methanol ice forms in cold molecular clouds and could be delivered to protoplanetary disks. On the other hand, on the warmer (30--50\,K) dust grain surfaces, more complex molecules such as \acetaldehyde, \methylformate, \dimethylether, and \acetone form via diffusion and reactions of heavier radicals (e.g., CH$_3$, CH$_2$OH, and CH$_3$O) formed via the UV/X-ray/cosmic-ray-induced photolysis and hydrogen abstraction of simpler molecules including \methanol \citep[e.g.,][]{Garrod2006, Garrod2013, Oberg2009, Walsh2014, Furuya2014, Notsu2021, Notsu2022}. For example, theoretical and experimental studies \citep[e.g.,][]{Garrod2006, Chuang2016} have shown that \acetaldehyde, \methylformate, \dimethylether, and \acetone are mainly formed via radical-radical reactions,
\begin{align}
    \mathrm{HCO} + \mathrm{CH_3} &\longrightarrow \mathrm{CH_3CHO}, \label{eq:r-r_reaction1} \\ 
    \mathrm{HCO} + \mathrm{CH_3O} &\longrightarrow \mathrm{CH_3OCHO}, \label{eq:r-r_reaction2} \\
    \mathrm{CH_3} + \mathrm{CH_3O} &\longrightarrow \mathrm{CH_3OCH_3}, \label{eq:r-r_reaction3} \\
    \mathrm{CH_3} + \mathrm{CH_3CO} &\longrightarrow \mathrm{CH_3COCH_3}. \label{eq:r-r_reaction4}
\end{align}
These reactions may contribute to the efficient formation of these species in protoplanetary disks \citep{Walsh2014, Furuya2014}, which may be the origin of abundant complex species in the V883 Ori disk (Figure \ref{fig:ratio_comparison}).

% In addition to these grain-surface reactions, some gas-phase reactions are also considered to contribute to the formation/destruction of complex molecules \citep[e.g.,][]{Garrod2006, Nomura2009, Taquet2016, Aikawa2020}. 

% Since the duration timescale of the FU Ori outburst is $\sim10^1$--$10^2$ yr, which is quite shorter than the typical timescale of gas-phase reactions \citep[$\sim10^4$ yr; e.g.,][]{Nomura2009}, and the outburst of V883 Ori seems to have began before 1888 \citep{Pickering1890}, the observed COM abundance ratios would directly reflect the ice composition, i.e., flesh sublimates without chemical composition being altered by gas-phase synthesis. Therefore, formation of complex molecules during the delivery from the envelope to the disk rather than the gas-phase alternation during the outburst would be responsible for the observed difference in the chemical composition of COMs. Indeed, the typical timescale of grain surface reactions ($\sim10^5$--$10^6$ yr) are comparable to the typical lifetime of Class I/II disks ($\sim$ 1 Myr), which makes it possible to efficiently form complex molecules.

The present observations also detected several isomeric pairs, i.e., \acetaldehyde and \ethyleneoxide pair and \acetone and \propanal pair. The abundance ratios between isomers are key to constraining the formation pathways. We found a \acetaldehyde/\ethyleneoxide ratio of $9.2_{-1.1}^{+1.2}$ and a \acetone/\propanal ratio of $2.2_{-0.4}^{+0.4}$ in the V883 Ori disk. These ratios are consistent with the values in IRAS 16293-2422 B \citep{Lykke2017} within an order of magnitude, suggesting that the isomers are also similarly formed.  

\textcolor{red}{add a comparison to Jeong et al. after the discussion at RIKEN}
Jeong et al. also observed COMs and shows the results consistent with our analysis...

% \textcolor{red}{[COMPARISON TO THE OLD BAND 7 DATA. SHOULD WE UPDATE THIS TO THE NEW VALUE PRESENTED IN JEONG ET AL.?] \citet{Lee2019} also detected several species presented in Figure \ref{fig:ratio_comparison} in ALMA Band 7 observations. Figure \ref{fig:ratio_comparison_Lee19} compared the abundance ratios of \acetaldehyde, \methylformate, \acetone, and \ethyleneoxide with respect to \methanol in our observations to those presented in \citet{Lee2019}. While the abundances ratios of \acetone and \ethyleneoxide are consistent between the two measurements, those of \acetaldehyde and \methylformate measured in our observations are higher than in \citet{Lee2019} by a factor of 2--8. This discrepancy may be due to the difference in the way to estimate the column densities: \citet{Lee2019} employed a similar but independent spectral fit to estimate the abundance ratios from the disk-integrated spectra, and assumed a circular emitting area of 0\farcs6 diameter without any inner cavities. This emitting area is different from our assumption of a ring-like emitting area between 0\farcs1 and 0\farcs3 radii (see Section \ref{subsec:spectral_fit}), which could result in a different line optical depth. \citet{Lee2019} also assumed a canonical $^{12}$C/$^{13}$C ratio of 60 to derive the \methanol column density using $^{13}$\methanol as the major isotopologue transitions are optically thick. The isotope ratio of $^{12}$C/$^{13}$C could, however, deviate from 60 (see Section \ref{subsubsec:12C/13C}).} 

% the actual spatial distributions in Band 7, which show inner emission cavities of $\sim0\farcs1$ radius, and

% Among the detected COMs, \propenal is the least abundant molecules. 





\begin{figure}
\epsscale{1.15}
\plotone{comparison_to_other_sources_v2}
\caption{Comparison of the COM abundance ratios with respect to \methanol among different evolutionary stages including the V883 Ori disk measured in the present work. The data point with a down arrow indicate an upper limit. We compiled the literature values for the protostellar envelopes of IRAS 16293-2422 A/B \citep{Lykke2017, Jorgensen2018, Manigand2020, Manigand2021}, the comet 67P/C-G \citep{Rubin2019, Schuhmann2019}, and the IRS 48 disk \citep{Brunken2022}. Note that while all the cometary values are limited to be upper limits because cometary measurement (mass spectra) cannot distinguish the isomeric molecules with the same mass (e.g., \acetaldehyde and \ethyleneoxide pair and \acetone and \propanal pair), the upper limit of \acetaldehyde in the IRS 48 disk is due to non-detection.}
\label{fig:ratio_comparison}
\end{figure}

% \begin{figure}
% \epsscale{1.15}
% \plotone{comparison_Lee19}
% \caption{\textcolor{red}{[COMPARISON TO THE OLD BAND 7 DATA. MAYBE WE CAN REMOVE THIS FIGURE.]Comparison of the abundance ratios of \acetaldehyde, \methylformate, \acetone, and \ethyleneoxide with respect to \methanol to those presented in the previous Band 7 observations \citep{Lee2019}. Note that the \ethyleneoxide is tentatively identified in \citet{Lee2019}.}}
% \label{fig:ratio_comparison_Lee19}
% \end{figure}


\subsection{Isotopic ratios of COMs}
The present observations also detected several isotopologues, including D and $^{13}$C, of some abundant molecules. Here we discuss the implications of these detections and the $^{12}$C/$^{13}$C and D/H ratios for the isotopic chemistry in protoplanetary disks. 

\subsubsection{$^{12}$C/$^{13}$C ratio}\label{subsubsec:12C/13C}
% The present observations detected $^{13}$C-isotopologues for \acetaldehyde, \methylformate, and \dimethylether. While \dimethylether has only one isomer of $^{13}$C-isotopologues as this molecule has two equivalent CH$_3$- functional groups, \acetaldehyde and \methylformate have two isomers of $^{13}$C isotopologues for each as they have two different functional groups which contain C atoms (i.e., CH$_3$- and CHO- functional group). For \methylformate and \dimethylether, all the isomers (i.e., $^{13}$CH$_3$OCHO, CH$_3$O$^{13}$CHO, and $^{13}$CH$_3$OCH$_3$) are detected. For \acetaldehyde, CH$_3^{13}$CHO are detected but $^{13}$CH$_3$CHO is non-detection, from which we constrain the upper limit of the column density. For the most abundant COMs, \methanol, $^{13}$C isotopologues are not covered in our spectral setup, and therefore no constraints on the $^{12}$C/$^{13}$C ratio of \methanol have been obtained. \textcolor{blue}{mention Lee et al. in prep. for low CH3OH/13CH3OH ratio?}
% Two $^{13}$C isotopologues of \acetaldehyde and \methylformate have been detected. No transitions of the $^{13}$C isotopologue of \methanol, $^{13}$\methanol, were covered in our spectral setup, but it has been detected in Band 6 observations \citep{Lee2019}. Since \citet{Lee2019} uses $^{13}$\methanol to derive the column density of normal \methanol due to the optical thickness of the \methanol transitions observed assuming the ISM $^{12}$C/$^{13}$C ratio of 69, no reliable estimates of $^{12}$\methanol/$^{13}$\methanol have been obtained. On the other hand, the transitions of main isotopologues of \acetaldehyde and \methylformate covered in the present observations include optically thin ones, and the $^{13}$C isotopologues of them are also detected, which enables us to deduce reliable estimates of $^{12}$C/$^{13}$C ratios of COMs for the first time.  

Figure \ref{fig:12C13C} shows the probability density distributions of the $^{12}$C/$^{13}$C ratios of \acetaldehyde, \methylformate, and \dimethylether. The measured $^{12}$C/$^{13}$C ratios are also summarized in Table \ref{tab:12C13C_summary} together with the value in the warm envelopes of IRAS 16293-2422.  The measured $^{12}$C/$^{13}$C ratios are all $\sim$20--30\footnote{We note that the statistical correction has not been applied for $^{13}$\dimethylether, which apparently contains equivalent two carbon atoms in the CH$_3$- functional groups. However, these two carbon atoms are actually not equivalent based on the formation pathway of \dimethylether (Equation \ref{eq:r-r_reaction3}); the formation of \dimethylether (and therefore the inclusion of $^{13}$C into \dimethylether) happens via the reaction between non-equivalent reactants (CH$_3$ and CH$_3$O) unlike e.g., the equivalent hydrogen addition to CO which forms CH$_3$OH and its deuterated isotopologues. We here simply compare the $^{12}$C/$^{13}$C of \dimethylether without statistical correction to other species, although the actual correction factor depends on the $^{12}$C/$^{13}$C ratios of reactants (i.e., CH$_3$ and CH$_3$O), which can be varied among them.}, which is significantly lower than the elemental abundance ratio of $^{12}$C/$^{13}$C in the local ISM ($\sim$ 69; \citealt{Wilson1999}), although the \acetaldehyde/$^{13}$CH$_3$CHO ratio has a probability at higher values as well due to the non-detection. Interestingly, a similarly low $^{12}$C/$^{13}$C ratio ($21\pm5$) of CO has been reported in the 70--110 au region of the protoplanetary disk around TW Hya by \citet{Yoshida2022_12CO13CO} \citep[see also][]{Zhang2017}. They suggest that the gas-phase isotope-exchange reaction, 
\begin{equation}\label{eq:12C13C_reaction}
    ^{13}\mathrm{C}^+ + ^{12}\mathrm{CO} \to {}^{12}\mathrm{C}^+ + {}^{13}\mathrm{CO} + \Delta E,
\end{equation}
($\Delta E \approx 35$\,K; \citealt{Langer1984, Furuya2011}), with the help of the high gas-phase C/O ratio ($> 1$), would lead to such a low $^{12}$CO/$^{13}$CO ratio in the warm molecular layer. The same mechanism would naturally explain the observed low $^{12}$C/$^{13}$C ratio in COMs if the $^{13}$C-rich CO is incorporated into the ice on dust grains in the disk midplane via vertical mixing and freeze-out \citep{Furuya2022}, from which the COMs such as \acetaldehyde and \methylformate are synthesized via hydrogenation and radical-radical reactions (Equations (\ref{eq:r-r_reaction1})--(\ref{eq:r-r_reaction4})) on dust grain surfaces.

An alternative mechanism that could alter the $^{12}$CO/$^{13}$CO ratio is the difference in the binding energies between the two isotopologues. \citet{Smith2015} proposed that the slightly higher binding energy of $^{13}$CO than that of $^{12}$CO ($840\pm4$\,K and $835\pm5$\,K on $^{12}$CO ice; \citealt{Smith2021}) could explain the high $^{12}$CO/$^{13}$CO ratios in the gas phase ($\sim85$--165) observed toward several low-mass young stellar objects. The difference in binding energies leads to the sublimation of $^{12}$CO at a slightly lower temperature than $^{13}$CO, making the gas-phase CO $^{13}$C-poor (and CO ice $^{13}$C-rich). However, this fractionation mechanism should work only in a very narrow temperature range. Also, the difference in binding energy between $^{12}$CO and $^{13}$CO is rather uncertain. It is therefore speculative that this mechanism contributes to the $^{13}$C fractionation. In addition, the gas and dust dynamics in disks (e.g., vertical turbulent mixing) is a critical unknown factor for this mechanism to work for the $^{13}$C fractionation of CO and consequently of COMs \citep[e.g.,][]{Yoshida2022_12CO13CO}.


In the warm envelope around the Class 0 protostar IRAS 16293-2422 B, similar lower values of the $^{12}$C/$^{13}$C ratio ($\sim$ 30) have been observed in some COMs (CH$_2$(OH)CHO (glycolaldehyde), \dimethylether, and possibly \methylformate) \citep[see Table \ref{tab:12C13C_summary};][]{Jorgensen2016, Jorgensen2018}, but for IRAS 16293-2422 A no evidence for $^{13}$C fractionation in COMs has been observed \citep[Table \ref{tab:12C13C_summary};][]{Manigand2020}. CO (and thus COMs) can be enriched in $^{13}$C via the exchange reaction (Equation \ref{eq:12C13C_reaction}) only when C/O ratio is greater than unity (or more specifically, CO is not the dominant C reservoir). In a rotationally supported disk, the C/O ratio could exceed unity due to the decoupling of the ice-coated grains from the gas. The low $^{12}$C/$^{13}$C ratio of some COMs in IRAS 16293-2422 B may indicate that similar dust-gas decoupling already occurs in the protostellar envelope \citep[e.g.,][]{Koga2022}. 
We note that the elemental abundance ratios of C/O and C/H in the gas phase become lower with increasing ionization rate in the disk ($\gtrsim10^{-17}$ s$^{-1}$; \citealt{Eistrup2016, Schwarz2018, Notsu2020}), which may inhibit the increase of $^{13}$CO abundance via the exchange reaction (Equation \ref{eq:12C13C_reaction}) \citep{Woods2009}. Alternatively, it may be possible that the destruction of carbonaceous grains enhances the C/H abundance and C/O ratio in protostellar disks and envelopes \citep[e.g.,][]{Wei2019, vantHoff2020}, although it is unclear whether the destruction of carbonaceous grains efficiently occurs in the outer cold region where CO and COMs form ($\lesssim100$ K).

In the comet 67P/C-G, only the \methanol/$^{13}$CH$_3$OH ratio has been measured ($91\pm10$), and it shows no $^{13}$C fractionation (Table \ref{tab:12C13C_summary}; \citealt{Altwegg2020_isotopes}). The $^{12}$C/$^{13}$C ratios of simpler molecular species such as CO, CO$_2$, and H$_2$CO have also been measured, but no $^{13}$C fractionation has been observed \citep{Hassig2017, Rubin2017} except for H$_2$CO ($40\pm14$; \citealt{Altwegg2020_isotopes}). This trend is in contrast to the observed $^{13}$C fractionation in COMs in the V883 Ori disk. Although the origin of this difference is unclear, a speculative explanation could be the formation environment of the comet 67P/C-G; it may be formed from the dust grains with non-$^{13}$C-rich ices, which can be produced from lower gas-phase C/O ratios ($< 1$). \citet{Yoshida2022_12CO13CO} reported a radial variation of the gas-phase $^{12}$CO/$^{13}$CO ratio in the disk around TW~Hya, where the outer region ($>130$ au) shows a higher $^{12}$CO/$^{13}$CO ratio ($>84$) than that in the inner region (70--110 au; $21\pm5$). If a similar radial variance in the $^{12}$CO/$^{13}$CO ratio existed in the proto-solar disk, the $^{13}$C fractionation in comets depends on the formation location within the disk.



% With the fact that the COMs are likely to be synthesized efficiently in disks as well (Section \ref{subsec:abundance_ratio}), this indicates that the carbon isotope fractionation are continuously at work during star and planet formation processes. 

% It is also interesting to investigate if there are any differences in $^{12}$C/$^{13}$C ratio between different functional groups, i.e., the CHO- and CH$_3$- branches. If the $^{12}$C/$^{13}$C is different 

% \textcolor{red}{future prospects: comparison of CH3-group and CHO-group $\to$ constrain formation pathway? Now trying to deduce the upper limit of 13CH3CHO}


\begin{figure*}
% \epsscale{1.15}
\plotone{12C13C}
\caption{Kernel density estimate (KDE) of the posterior probability density distributions of column density ratios of $^{12}$C- to $^{13}$C-isotopologues for \acetaldehyde, \methylformate, and \dimethylether. Two isomers of $^{13}$C-isotopologues are observed for \acetaldehyde and \methylformate. The probability density is normalized by the peak being unity and offset for visual clarity. The ISM value of 69 is marked by the vertical dashed line.}
\label{fig:12C13C}
\end{figure*}

\begin{deluxetable*}{lcccc}
\tablecaption{$^{12}$C/$^{13}$C Ratios of COMs}
\tablehead{\colhead{} & \colhead{V883 Ori (this work)$^\dagger$} & \colhead{IRAS~16293A\tablenotemark{a}} & \colhead{IRAS~16293B\tablenotemark{b}} & \colhead{67P/C-G\tablenotemark{c}}}
\startdata
Methanol & & & \\
\quad \methanol/$^{13}$CH$_3$OH & --- & $65\pm27$ & --- & $91\pm10$ \\
Acetaldehyde & & & \\
\quad \acetaldehyde/CH$_3^{13}$CHO & 29$_{-8}^{+16}$ & --- & 67$^{\ddagger}$ & --- \\
\quad \acetaldehyde/$^{13}$CH$_3$CHO & $> 4.2$ & --- & 67$^{\ddagger}$ & ---\\
Methyl Formate & & & \\
\quad \methylformate/CH$_3$O$^{13}$CHO & 23$_{-3}^{+3}$ & $75\pm32$ & (41)$^{\ast}$ & --- \\
\quad \methylformate/$^{13}$CH$_3$OCHO & 23$_{-2}^{+3}$ & --- & --- & --- \\
Dimethyl Ether & & & \\
\quad \dimethylether/$^{13}$CH$_3$OCH$_3$ & $25_{-6}^{+10}$ & $86\pm18$ & 17 & --- \\
Glycolaldehyde & & & \\
\quad CH$_2$OHCHO/CH$_2$OH$^{13}$CHO & --- & --- & 27$^{\ddagger}$ & --- \\
\quad CH$_2$OHCHO/$^{13}$CH$_2$OHCHO & --- & --- & 27$^{\ddagger}$ & --- \\
\enddata
\tablenotetext{\dagger}{For the measurement in the V883 Ori disk, uncertainties are the 16th and 84th percentile of the posterior distributions. The 0.3rd percentile (corresponding to $3\sigma$) of the posterior distributions are adopted for the lower limit of \acetaldehyde/$^{13}$CH$_3$CHO.}
\tablenotetext{\ddagger}{Derived from the fit assuming the same column density for these isomeric pairs.}
\tablenotetext{\ast}{Tentative measurement using tentatively detected transitions.}
\tablenotetext{a}{\citet{Manigand2020}.}
\tablenotetext{b}{\citet{Jorgensen2016} and \citet{Jorgensen2018}.}
\tablenotetext{c}{\citet{Altwegg2020_isotopes}}
\label{tab:12C13C_summary}
\end{deluxetable*}

\subsubsection{D/H ratio}
Deuterium fractionation (an enhancement of D/H ratios in molecules) is also a key to understanding the formation environment and thermal history of planetary medium \citep[e.g.,][]{Ceccarelli2014, Nomura2023_PPVII}. The D/H ratios of COMs, in particular \methanol, has been measured in warm protostellar envelopes \citep[e.g.,][]{Jorgensen2018, Drozdovskaya2021}. Even multiply-deuterated species have been detected toward IRAS 16293-2422 protobinary \citep[e.g.,][]{Manigand2019, Richard2021, Drozdovskaya2022}, and the similarity of the D enrichment between the protostellar envelopes and the solar system comets suggests the inheritance of interstellar molecules to comets \citep{Drozdovskaya2021}. In protoplanetary disks, while the D/H ratios of several simpler molecules such as HCN, HCO$^{+}$, and N$_2$H$^{+}$ have been measured \citep[e.g.,][see also \citealt{Aikawa2022}]{Cataldi2021}, no measurements of COM deuteration exist so far. 


The deuterium fractionation is initiated by the deuteration of $\mathrm{H_3^+}$ with the gas-phase ion-molecule reaction, 
\begin{equation}\label{eq:deuteration}
    \mathrm{H}_3^+ + \mathrm{HD} \rightarrow \mathrm{H_2D^+} + \mathrm{H_2} + \Delta E.
\end{equation}
This reaction is exothermic ($\Delta E \approx 230$\,K; e.g., \citealt{Millar1989}), and therefore at the low temperature ($\lesssim 30$\,K) the backward reaction is suppressed, enhancing the $\mathrm{H_2D^+}$ abundance in the gas phase. In dense cold regions such as prestellar cores, freeze-out of CO, a main reactant with $\mathrm{H_2D^+}$, further enhances the fractionation \citep[e.g.,][]{Roberts2000}. Dissociative recombination of $\mathrm{H_2D^+}$ enhances the abundance of D atoms, which subsequently causes the deuteration of molecules formed on dust grain surfaces, including COMs. In the warmer conditions ($\gtrsim 30$\,K), the fractionation is more moderate, since the backward reaction of Equation (\ref{eq:deuteration}) is not severely suppressed. Molecular D/H ratio thus can be used to probe the formation environment of molecules including COMs. For example, some molecules showing relatively high D/H ratios ($\gtrsim$ several \%) in the warm protostellar envelopes are considered to be formed in cold prestellar cores \citep[e.g.,][]{Jorgensen2018, Yamato2022}

We have observed multiple deuterated species of COMs, \methanol, \acetaldehyde, and \methylformate, for the first time in protoplanetary disks, while the deuterated \methanol, CH$_2$DOH, has also been detected in the previous Band 7 observations \citep{Lee2019}. For deuterated \acetaldehyde, two different isomers (CH$_2$DCHO and CH$_3$CDO) have been tentatively detected. Several transitions of two isomers of deuterated \methylformate (CH$_3$OCDO and CH$_2$DOCHO) are covered but not detected, from which we constrain the upper limit on their column densities. Figure \ref{fig:DH} shows the probability density distributions of D/H ratios of these molecules. The measured D/H ratios of CH$_2$DOH, CH$_3$CDO, and CH$_2$DCHO are $0.0057_{-0.0015}^{+0.0018}$, $0.0084_{-0.0019}^{+0.0021}$, and $0.043_{-0.012}^{+0.012}$, respectively. For \acetaldehyde, these two different D/H (column density) ratios are settled to similar values if we apply the statistical correction (i.e., divide CH$_2$DCHO/\acetaldehyde ratio by a factor of three). For \methylformate, upper limits of $\lesssim0.02$ for both CH$_2$DOCHO/\methylformate and CH$_3$OCDO/\methylformate ratios are obtained.

% Furthermore, the present observations covered several transitions of deuterated \methylformate, CH$_2$DOCHO and CH$_3$OCDO. 
% Non-detection of these lines provided an upper limit of $\lesssim0.02$ for both CH$_2$DOCHO/\methylformate and CH$_3$OCDO/\methylformate ratios. This upper limit indicates that the deuteration of \methylformate is also lower compared to that in IRAS 16293-2422 B ($\sim0.06$; \citealt{Jorgensen2018}), which may also point to the formation in warmer regions.

The CH$_2$DOH/\methanol ratio is slightly lower than the typical values observed in the warm protostellar envelopes ($\sim$~0.01--0.1; see Figure 2 in \citealt{Drozdovskaya2021}).  The lower value may point to the formation at warmer condition than in prestellar cores, for example, in protoplanetary disks. Although methanol is considered to form on the surface of cold dust grains in cold prestellar cores, and be delivered to protoplanetary disks, it can be destroyed by the UV/X-ray radiation and/or energetic particles such as cosmic rays from the central star (\citealt{Notsu2021}, see also Section \ref{subsec:physicochemical_structure}) and reform on the dust grain surfaces in disks. As the D/H ratios are very sensitive to the formation temperature, the slightly lower D/H ratio of \methanol in the V883 Ori disk may be explained by the reformation of \methanol on warmer dust grain surfaces. The upper limits on the D/H ratios of \methylformate are also lower by a factor of $\sim3$ compared to those in IRAS 16293-2422 B ($\sim0.06$; \citealt{Jorgensen2018}), which may also point to the (re-)formation in warmer conditions.

In the comet 67P/C-G, D/H ratio of \methanol (not the column density ratio, but the elemental D/H ratio after the statistical correction) has been measured to be 0.71--6.6~\% \citep{Drozdovskaya2021}, which reduced to a CH$_2$DOH/CH$_3$OH ratio (0.021--0.20) similar to the values in the warm protostellar envelopes, pointing to the prestellar origins of \methanol in the comet 67P/C-G \citep{Drozdovskaya2021}. The CH$_2$DOH/\methanol ratio in the V883 Ori disk ($0.0057_{-0.0015}^{+0.0018}$) is smaller than the value in the comet 67P/C-G by a factor of $\sim$ 3--4. This might suggest the different origins of \methanol in the comet 67P/C-G and the V883 Ori disk; while \methanol in the V883 Ori disk may have experienced the destruction and reformation on the lukewarm ($\sim$~30--50 K) dust grain surfaces in the inner region of the disk in its quiescent phase, the comet 67P/C-G might be formed in the cold ($\sim$~10 K) outer region of the proto-solar disk, where molecules are directly inherited from the natal envelope without significant chemical alternation. In addition, it may also be possible that the efficiency of \methanol destruction varies between the V883 Ori disk in its quiescent phase and the proto-solar disk, potentially due to the different X-ray/UV/cosmic-ray fluxes from the central star \citep{Notsu2021}.

\textbf{We note that the CH$_2$DOH/\methanol ratio derived in the present work is lower than the value reported in \citet{Lee2019} by a factor of $\sim$~3.5. Although the origin of this discrepancy is unclear, the different assumptions in the spectral fits as discussed in Section \ref{subsubsec:12C/13C} and the different upper state energies of covered transitions may affect the results. In addition, the spectroscopic information (i.e., rest frequencies and intrinsic strengths of transitions) may be unreliable for CH$_2$DOH: the observed transitions are all $b$-type transitions, which potentially include larger uncertainties compared to $a$-type transitions \citep{Pearson2012, Watanabe2021, Ohno2022}. }

The deuterated \acetaldehyde has only been detected toward the protostar IRAS 16293-2422 B \citep{Coudert2019}, and the CH$_3$CDO/\acetaldehyde and CH$_2$DCHO/\acetaldehyde ratios are measured to be $\sim0.047$ and $\sim0.014$, respectively \citep{Manigand2020}. These values are comparable to what we found in the V883 Ori disk, but the trend between the two isomers are different. After the statistical correction, the deuteration of CHO- functional group is significantly higher than that of CH$_3$- functional group in IRAS 16293-2422 B \citep{Manigand2020}, while the deuteration of these functional groups are similar in the V883 Ori disk . It should be noted, however, that other COMs do not show such differential deuteration among functional groups in IRAS 16293-2422 B \citep{Jorgensen2018}. If the differential deuteration in two functional groups of \acetaldehyde is real, it may imply that the formation pathways of \acetaldehyde in protostellar envelopes and protoplanetary disks are different (e.g., successive hydrogenation and radical-radical reactions). 

We finally note that the measured D/H ratios here are based on the tentative identification of deuterated isotopologues, and therefore further observations of sufficient number of transitions are needed to confirm our results. 




\begin{figure*}
% \epsscale{1.15}
\plotone{DH}
\caption{The KDE of the posterior probability density distributions of column density ratios of deuterated to normal isotopologues for \methanol (blue), \acetaldehyde (orange), and \methylformate (green). For \acetaldehyde and \methylformate, two different isomers of deuterated species (CH$_3$CDO and CH$_2$DCHO) have been observed. The probability density is normalized by the peak being unity and offset for visual clarity. The gray-shaded region indicates the range of typical values for CH$_2$DOH/\methanol ratios measured in warm protostellar envelopes. Note that the statistical corrections have not been applied for CH$_2$DOCHO and CH$_2$DCHO which contain three equivalent hydrogen atoms in their CH$_3$- functional groups.}
\label{fig:DH}
\end{figure*}

\subsection{Deficiency of Nitrogen-bearing COMs}
While the number of detected oxygen-bearing COMs in the V883 Ori disk is comparable to that in protostellar envelopes, nitrogen-bearing COMs seem to be deficient in the V883 Ori disk compared to those in protostellar envelopes. In the present observations, 
%While many species of oxygen-bearing COMs have been detected in the present observations, 
only C$_2$H$_3$CN has been tentatively detected as a nitrogen-bearing COM. Our spectral setup covers the transitions of other nitrogen-bearing COMs, such as NH$_2$CHO and C$_2$H$_5$CN, but they are not detected. In Band 7 observations, CH$_3$CN is the only nitrogen-bearing COMs that has been detected \citep{Lee2019}. Even for CH$_3$CN, its column density is lower than those of oxygen-bearing COMs \citep{Lee2019}. The spectral setup of the present observations did not cover intense transitions of CH$_3$CN, and therefore it is not detected. 
% Other nitrogen-bearing COMs are also not detected or not covered in the present spectral setup. 
%In summary, the nitrogen-bearing COMs are apparently deficient in the V883 Ori disk compared to the oxygen-bearing COMs.

Segregation of oxygen-bearing COMs and nitrogen-bearing COMs, i.e. significant abundance variation between them, have been observed in protostellar envelopes.
Recently, \citet{vantHoff2020} suggested that the sublimation of refractory carbonaceous grains at a high temperature ($\sim300$--500\,K, depending on the gas density) could lead to an enhancement of carbon- and nitrogen-bearing molecules compared to oxygen-bearing molecules in the inner hot region of protostellar envelopes, as the refractories contain more carbon and nitrogen than volatile ices referring to cometary abundances \citep{Rice2018, Rubin2019}. Furthermore, recently-reported evidence of the abundant ammonium salt in comet 67P/C-G \citep{Altwegg2020, Poch2020} provides an additional nitrogen source in the inner hot region, since the sublimation temperature of ammonium salt is comparable to that of carbonaceous grains \citep[$\sim200$--250\,K; e.g.,][]{Noble2013, Bergner2016}. \citet{Nazari2023} reported a systematic enhancement of the abundance of nitrogen-bearing COMs (CH$_3$CN and C$_2$H$_3$CN) in the inner ``hot'' ($\gtrsim300$\,K) region compared to the ``warm'' ($\sim100$--300\,K) region using the ALMA surveys toward 37 high-mass protostars, providing the first evidence of the carbonaceous grains (and potentially ammonium salt) sublimation.

%The sublimation of carbonaceous grains and ammonium salt may explain the apparent deficiency of the nitrogen-bearing COMs in the V883 Ori disk. 
As discussed in Section \ref{subsec:physicochemical_structure}, the present Band 3 observations (and previous sub-mm observations as well) mainly trace the emission from the intermediate region of the disk ($\sim40$--80\,au) and do not probe the innermost region ($\lesssim40$\,au). Therefore, one of the possible explanations for the deficiency of nitrogen-bearing COMs is that the nitrogen-bearing COMs are abundant in the innermost hot region ($\lesssim40$\,au) but they are hidden by the intense dust emission in ALMA observations (or beam-diluted due to their smaller emitting areas). To fully reveal the nitrogen content in the V883 Ori disk, high-resolution observations at longer wavelengths (ALMA Band 1, VLA, and ngVLA) are needed to avoid the dust absorption. It is also essential to determine the temperature structure of the disk to fully characterize the nitrogen chemistry in the V883 Ori disk.

\subsection{Implications for Sulfur Chemistry}
Sulfur-bearing molecules have been routinely observed in protostellar environments and solar system comets, and their potential chemical links have been discussed \citep[e.g.,][]{Drozdovskaya2018}. On the other hand, detection of gas-phase sulfur-bearing molecules are still sparse in Class II disks except for CS and its isotopologue \citep[][and references therein]{LeGal2021}. Recent observations have just began to report the (tentative) detection of sulfur-bearing molecules such as SO, SO$_2$, and SiS in a few warm disks, which might be originated from the thermal desorption and/or the accretion shock toward the protoplanets \citep{Booth2021, Booth2023, Law2023_SiS}. 

In the V883 Ori disk, \citet{Lee2019} reported the detection of CH$_3$SH and tentative detection of SO$_2$. The present observations have confirmed SO$_2$ and newly detected OCS.
Since the emission distributions are similar to that of other oxygen-bearing COMs in the V883 Ori disk (see Figure \ref{fig:moment_zero_gallery} for OCS), the SO$_2$ and OCS detected in the present observations likely originate from the thermal sublimation as with the other molecules. \textbf{The SO$_2$/OCS ratios are measured to be $0.56_{-0.20}^{+0.21}$ in the V883 Ori disk. In molecular clouds, SO$_2$ and OCS ices are ubiquitously detected by infrared observations \citep[e.g.,][]{Boogert2015, McClure2023}. These molecules are also detected in the warm envelopes of IRAS~16293-2422 \citep{Drozdovskaya2018} and the comet 67P/C-G \citep{Calmonte2016}. The SO$_2$/OCS ratios are $\sim$~0.2--0.5 in molecular cloud ice and in IRAS~16293-2422, consistent with the value in the V883 Ori disk, suggesting that the sulfur-bearing ices in protoplanetary disks are inherited from the ISM. On the other hand, the SO$_2$/OCS ratio is higher in comet 67P/C-G ($\sim$~280) than these values by three orders of magnitude, which may indicate the potential chemical evolution from the ISM or different formation temperatures between ices in molecular clouds and comet 67P/C-G as discussed in \citet{Drozdovskaya2018}.}



% Sulfur-bearing molecules such as CS and SO have also been used to infer the gas-phase C/O ratio in protoplanetary disks. \textcolor{red}{can we say something about C/O ratio from OCS/SO2 ratio?}




\section{Summary} \label{sec:summary}
We presented the ALMA Band 3 observations of COMs in the V883 Ori disk at an angular resolution of 0\farcs3--0\farcs4. We analyzed the disk-integrated spectra in detail to obtain the spatial distribution of the COM emission and the column densities of COMs. Our major findings are summarized as follows:
\begin{enumerate}
    \item[1.] We robustly identified eleven oxygen-bearing COMs (including isotopologues) in the disk-integrated spectra, where \dimethylether, \propenal, \propanal, $^{13}$\methylformate, and CH$_3$O$^{13}$CHO are the first detection in the V883 Ori disk. We also tentatively identified five COMs (including isotopologues) and detected two sulfur-bearing molecules, OCS and SO$_2$. Nitrogen-bearing COMs are not detected except for a tentative detection of C$_2$H$_3$CN.
    \item[2.] While the velocity-integrated intensity maps show the centrally-peaked emission morphology, the detailed analyses of the line profiles revealed the inner emission cavity ($\sim40$\,au radius), similar to the previous sub-mm observations. This indicates that the COM emissions are suppressed even in Band 3 where the dust continuum emission is fainter, possibly due to the higher dust temperature in the midplane caused by the viscous accretion heating. In addition, the destruction of COMs in the cavity region by the strong UV/X-ray/cosmic-ray radiations from the central outbursting protostar may explain the inner emission cavity of COMs.  Our ALMA Band 3 observations suggest that the observations in longer wavelengths, where the dust continuum emission should be fainter, with ALMA Band 1, Jansky Very Large Array (VLA), and future ngVLA, are essential to directly probe the chemistry of the innermost region of the V883 Ori disk.
    
    \item[3.] We found that the column density ratios of complex molecules with respect to \methanol are significantly higher than those in the warm protostellar envelopes of IRAS 16293-2422, and similar to the values measured in comet 67P/C-G. This may indicate that the formation of complex molecules occurs in protoplanetary disks, which can cause the chemical evolution en route to planetary systems.
    
    \item[4.] We characterized the $^{13}$C-fractionation pattern of COMs in protoplanetary disks for the first time. The $^{12}$C/$^{13}$C ratios of \acetaldehyde, \methylformate, and \dimethylether consistently show a lower value ($\sim$\,20--30) compared to the canonical ISM ratios ($\sim$\,69). These COMs could be formed from CO enriched in $^{13}$C due to the exchange reaction with $^{13}$C$^+$ in an environment characterized with a high gas-phase C/O ratios.
    % which indicates the $^{13}$C-fractionation of CO in gas phase caused by the dust-gas decoupling is incorporated into COMs in the ice mantles. 
    % We also compared the $^{12}$C/$^{13}$C ratios of COMs in the protostellar envelopes of IRAS 16293-2422 and comet 67P/C-G and discussed the potential origins of their different $^{13}$C-fractionation patterns.
    \item[5.]  We also measured the D/H ratios of multiple COMs in protoplanetary disks for the first time. The D/H ratios of \methanol and \methylformate shows lower values ($\sim$\,0.006 and $\lesssim$\,0.02, respectively) compared to those in protostellar envelopes, implying the (re-)formation of these molecules in warmer dust grain surfaces in protoplanetary disks. 
    % The D/H ratios of \methanol are also lower than that in comet 67P/C-G, potentially suggesting that the \methanol ices in the V883 Ori disk have experienced a significant destruction and reformation on the warmer dust surface in the disk while \methanol in comet 67P/C-G is directly inherited from the natal envelope in the outer region of the proto-solar disk. 
    % In addition, the destruction efficiency of \methanol in the disk of outbursting V883 Ori may be different from that in the proto-solar disk, which could cause the different degree of inheritance.
    % \item[6.] We discussed the deficiency of nitrogen-bearing COMs in the V883 Ori disk in terms of the segregation of oxygen-bearing COMs and nitrogen-bearing COMs. It may be possible that the nitrogen-bearing COMs are hidden in the inner hot region of the V883 Ori disk, where the destruction of carbonaceous grains and the potential sublimation of ammonium salts may provide the additional feedstock of the nitrogen-bearing COMs.   
\end{enumerate}


\begin{acknowledgments}
We thank Tomohiro~Yoshida and Gianni~Cataldi for the fruitful discussion on the technical aspects of the present work. We also thank Kiyoaki~Doi for the discussion on the dust continuum emission of the V883 Ori disk in various ALMA Bands. 
% plus Jeong-Eun?
Y.Y. is financially supported by Grant-in-Aid for the Japan
Society for the Promotion of Science (JSPS) Fellows (KAKENHI Grant Number JP23KJ0636) and International Graduate Program for Excellence in Earth-Space Science (IGPEES) of the University of Tokyo.
S.N. and Y.O. are supported by RIKEN Special Postdoctoral Researcher Program (Fellowships).
S.N.~is also grateful for support from Grants-in-Aid for JSPS (Japan Society for the Promotion of Science) Fellows Grant Number JP23KJ0329, and MEXT/JSPS Grants-in-Aid for Scientific Research (KAKENHI) Grant Numbers JP20K22376, JP23K13155, and JP23H05441.
Y.A. is supported by MEXT/JSPS Grants-in-Aid for Scientific Research (KAKENHI) Grant Numbers JP20H05847 and JP21H04495.
N.S. is supported by MEXT/JSPS Grants-in-Aid for Scientific Research (KAKENHI) Grant Numbers JP20H05845 and JP20H00182.
Y.O. is supported by MEXT/JSPS Grants-in-Aid for Scientific Research (KAKENHI) Grant Number JP22K20390.
H.N. is supported by MEXT/JSPS Grants-in-Aid for Scientific Research (KAKENHI) Grant Numbers JP18H05441 and JP19K03910.
This paper makes use of the following ALMA data: ADS/JAO.ALMA\#2021.1.00357.S, ADS/JAO.ALMA\#2021.1.00186.S. % need to verify if this data acknowlegdemnet is needed
ALMA is a partnership of ESO (representing its member states), NSF (USA) and NINS (Japan), together with NRC (Canada), MOST and ASIAA (Taiwan), and KASI (Republic of Korea), in cooperation with the Republic of Chile. The Joint ALMA Observatory is operated by ESO, AUI/NRAO and NAOJ. 
% The authors thank the ALMA staff for their excellent support. % Do we really need this?
%%

%%
\end{acknowledgments}

%% To help institutions obtain information on the effectiveness of their 
%% telescopes the AAS Journals has created a group of keywords for telescope 
%% facilities.
%
%% Following the acknowledgments section, use the following syntax and the
%% \facility{} or \facilities{} macros to list the keywords of facilities used 
%% in the research for the paper.  Each keyword is check against the master 
%% list during copy editing.  Individual instruments can be provided in 
%% parentheses, after the keyword, but they are not verified.

\vspace{5mm}
\facilities{ALMA}

%% Similar to \facility{}, there is the optional \software command to allow 
%% authors a place to specify which programs were used during the creation of 
%% the manuscript. Authors should list each code and include either a
%% citation or url to the code inside ()s when available.

\software{GoFish}

%% Appendix material should be preceded with a single \appendix command.
%% There should be a \section command for each appendix. Mark appendix
%% subsections with the same markup you use in the main body of the paper.

%% Each Appendix (indicated with \section) will be lettered A, B, C, etc.
%% The equation counter will reset when it encounters the \appendix
%% command and will number appendix equations (A1), (A2), etc. The
%% Figure and Table counter will not reset.

\appendix
% \restartappendixnumbering
\section{Channel Maps of the COM emission}\label{appendix:channel_maps}
Figure \ref{fig:channelmap_CH3OH}, \ref{fig:channelmap_CH3CHO}, and \ref{fig:channelmap_CH3OCHO} show the channel maps of the bright, unblended transitions of \methanol, \acetaldehyde, and \methylformate, respectively, which are used for the line profile analysis in Section \ref{subsubsec:line_profile_analysis}. The comparison to the corresponding Keplerian radius (i.e., $r = GM_\star\sin^2 i / (v - v_\mathrm{sys})^2$) at each velocity channel is made.

\begin{figure*}
% \epsscale{1.15}
\plotone{channelmap_CH3OH.pdf}
\caption{Channel maps of the \methanol $2_{-1,1}$ -- $1_{-1,0}$ E $v_t=0$ transition. The black contours mark the [$-$3, 3, 5, 7, ...]$\,\times\,\sigma$ levels. The dashed contours indicate the negative values. The numbers in the upper-right and upper-left corners indicate the velocity of the channel and its corresponding Keplerian radius ($r = GM_\star\sin^2i/(v - v_\mathrm{sys})^2$), respectively. The red cross in each panel indicate the position of the disk center. The beam is shown in the lower-left corner of the lower-left panel.}
\label{fig:channelmap_CH3OH}
\end{figure*}

\begin{figure*}
% \epsscale{1.15}
\plotone{channelmap_CH3CHO.pdf}
\caption{Same as Figure \ref{fig:channelmap_CH3OH}, but for \acetaldehyde $5_{2,3}$ -- $4_{2,2}$ E $v_t=0$ transition.}
\label{fig:channelmap_CH3CHO}
\end{figure*}

\begin{figure*}
% \epsscale{1.15}
\plotone{channelmap_CH3OCHO.pdf}
\caption{Same as Figure \ref{fig:channelmap_CH3OH}, but for \methylformate $8_{5,3}$ -- $7_{5,2}$ A $v_t=0$ transition.}
\label{fig:channelmap_CH3OCHO}
\end{figure*}



\section{JvM effect and continuum subtraction}\label{appendix:JvM_effect}
Here we describe a brief summary of the Jorsater \& van Moorsel (JvM) effect \citep{JvM, Czekala2021}, and subsequently discuss the continuum subtraction methods, which could mitigate the JvM effect.

The JvM effect could result in incorrect measurements of the flux density on the interferometric images produced by the CLEAN deconvolution method, which was originally discussed in \citet{JvM} and extensively investigated in \citet{Czekala2021}. The CLEAN algorithm iteratively extracts the CLEAN components from the residual image, and this process will be terminated when the peak value of the residual image within the CLEAN mask reaches a user-defined threshold (typically 2--3 $\times$ the RMS level; see Figure 2 in \citet{Czekala2021} for a comprehensive summary of this procedure). The final image product of CLEAN is the sum of the CLEANed model image (convolved with a CLEAN beam determined by a Gaussian fit to the dirty beam) and the residual images, which have inconsistent units (Jansky per CLEAN beam and Jansky per dirty beam). Therefore, if faint emission below the CLEAN threshold remains on the residual image, the intensity scale in the final image can be incorrect (i.e., sum of values with two different units). The impact of this effect depends on the emission strengths, the CLEAN threshold, and the deviation of the dirty beam shape from the CLEAN beam shape (Gaussian): if a significant portion of the emission remains on the residual image and the deviation is large, the intensity scale in the final image will drastically modified from the correct intensity scale \citep{JvM, Czekala2021}.  

To correct for this unit inconsistency, \citet{Czekala2021} applied the ``JvM correction'' to the line emission in protoplanetary disks, where the residual image is scaled before the addition with the CLEANed model image by the ratio of the dirty beam area to the CLEAN beam area, forcing the unit of the residual image to be Jansky per CLEAN beam. However, \citet{Casassus2022} pointed out that the JvM correction can artificially manipulate the noise levels and exaggerate the signal-to-noise ratio of the emission, as the correction will scale the residual image that contains the noise typically by a factor smaller than unity. So far, no concrete solution for this problem has been drawn.

We applied the JvM correction to our data following \citet{Czekala2021}, which obtained a JvM $\epsilon$ (the ratio of dirty beam area to the CLEAN beam area) of $\sim0.28$, indicating that the effect is severe. This could result in a drastic underestimation of the uncertainties of physical quantities derived from the fits described in Section \ref{sec:analysis_result}. Another possible solution is CLEANing deeper (e.g., down to 0.5--1 $\times$ the RMS level) so that all the emission components are recovered, but this would be unrealistic as too deep CLEANing will not converge and it is practically unfeasible to CLEAN down to such deep levels particularly for the line image cubes with a lot of channels. Alternatively, we mitigate the JvM effect and recover correct flux scale as much as possible by employing the continuum subtraction on the image plane (with the CASA task \texttt{imcontsub}) rather than on the visibility plane (with the CASA task \texttt{uvcontsub}). The line emission in our data is almost co-spatial to the strong continuum emission, and therefore imaging without continuum subtraction can almost completely include the line emission into the CLEAN model, thanks to the strong line plus continuum intensity well above the CLEAN threshold. While the continuum component is still affected by the JvM effect, the line emission can be safely recovered by subsequent continuum subtraction on the image plane. Figure \ref{fig:JvM_spectrum} shows an example demonstrating that almost no spectral line component remain in the residual spectrum and the line emission is almost fully recovered in the CLEANed image. We note that this method is only applicable to the spectral line data associated with a strong continuum emission. If the continuum emission is faint in the emitting region of the spectral line, the spectral line component will still remain in the residual even with the continuum subtraction on the image plane. 

\begin{figure*}
\epsscale{1.15}
\plotone{JvM_effect_spectrum.pdf}
\caption{Line plus continuum spectra of SPW 2 averaged over 1\farcs2 aperture for the CLEAN image (dark gray), the CLEAN model (gray), and the residual (light gray). The CLEAN model spectrum (in unit of Jy pixel$^{-1}$ by default) is scaled up by a ratio of the CLEAN beam area to the pixel area for visual clarity. The spectral line emission is almost fully recovered into the CLEAN model and none of that remains in the residual spectrum. The horizontal dashed line indicates the zero-flux level.}
\label{fig:JvM_spectrum}
\end{figure*}


\section{Detected transitions}\label{appendix:transitions}
Table \ref{tab:transitions} lists the transitions detected in our observations for each molecular species, including the blended ones.

\startlongtable
\begin{deluxetable*}{ccCCCCc}
\tablecaption{Detected Transitions for Each Species\label{tab:transitions}}
\tablehead{\colhead{Species} & \colhead{Transition} & \colhead{$\nu_0$ (GHz)} & \colhead{log$_{10}A_\mathrm{ul}$ (s$^{-1}$)} & \colhead{$g_\mathrm{u}$} & \colhead{$E_\mathrm{u}$ (K)} & \colhead{SPW}}
\decimals
\startdata
\hline 
\multicolumn{7}{c}{Methanol (\methanol)} \\
\hline 
CH$_3$OH$^\ddagger$ & $2_{-1,2}$ -- $1_{-1,1}$ E $v_t=1$ & 96.492163 & -5.5963 & 20.0 & 298 & 7 \\
CH$_3$OH$^\ddagger$ & $2_{0,2}$ -- $1_{0,1}$ E $v_t=1$ & 96.493551 & -5.4706 & 20.0 & 307 & 7 \\
CH$_3$OH$^\ddagger$ & $2_{1,1}$ -- $1_{1,0}$ E $v_t=1$ & 96.501713 & -5.5967 & 20.0 & 420 & 7 \\
CH$_3$OH & $2_{0,2}$ -- $1_{0,1}$ A $v_t=1$ & 96.513686 & -5.4709 & 20.0 & 430 & 7 \\
CH$_3$OH$^\dagger$ & $2_{1,2}$ -- $1_{1,1}$ E $v_t=0$ & 96.739358 & -5.5923 & 20.0 & 12 & 9 \\
CH$_3$OH$^\dagger$ & $2_{0,2}$ -- $1_{0,1}$ A $v_t=0$ & 96.741371 & -5.4676 & 20.0 & 6 & 9 \\
CH$_3$OH & $2_{0,2}$ -- $1_{0,1}$ E $v_t=0$ & 96.744545 & -5.4676 & 20.0 & 20 & 9 \\
CH$_3$OH & $2_{-1,1}$ -- $1_{-1,0}$ E $v_t=0$ & 96.755501 & -5.581 & 20.0 & 28 & 9 \\
CH$_3$OH & $2_{1,1}$ -- $1_{1,0}$ A $v_t=0$ & 97.582798 & -5.5807 & 20.0 & 21 & 10 \\
CH$_3$OH$^\dagger$ & $21_{6,16}$ -- $22_{5,17}$ A $v_t=0$ & 97.677684 & -5.8404 & 172.0 & 729 & 10 \\
CH$_3$OH$^\dagger$ & $21_{6,15}$ -- $22_{5,18}$ A $v_t=0$ & 97.678803 & -5.8403 & 172.0 & 729 & 10 \\
\hline 
\multicolumn{7}{c}{Methanol (CH$_2$DOH)} \\
\hline 
CH$_2$DOH$^\ddagger$ & $2_{1,1}$ -- $2_{0,2}$ e0 & 86.668751 & -5.3322 & 5.0 & 10 & 4 \\
CH$_2$DOH$^\ddagger$ & $4_{1,3}$ -- $4_{0,4}$ o1 & 97.870192 & -5.3363 & 9.0 & 44 & 10 \\
CH$_2$DOH & $4_{0,4}$ -- $3_{1,3}$ e0 & 98.031213 & -5.616 & 9.0 & 21 & 10 \\
\hline 
\multicolumn{7}{c}{Methyl Formate (\methylformate)} \\
\hline 
CH$_3$OCHO & $7_{6,1}$ -- $6_{6,0}$ E $v_t=1$ & 85.157135 & -5.6248 & 30.0 & 228 & 3 \\
CH$_3$OCHO$^\dagger$ & $7_{5,3}$ -- $6_{5,2}$ A $v_t=1$ & 85.185466 & -5.3602 & 30.0 & 220 & 3 \\
CH$_3$OCHO$^\dagger$ & $7_{5,2}$ -- $6_{5,1}$ A $v_t=1$ & 85.186063 & -5.3602 & 30.0 & 220 & 3 \\
CH$_3$OCHO & $7_{4,3}$ -- $6_{4,2}$ E $v_t=1$ & 85.506219 & -5.2142 & 30.0 & 214 & 1 \\
CH$_3$OCHO & $7_{5,3}$ -- $6_{5,2}$ E $v_t=1$ & 85.55338 & -5.3527 & 30.0 & 220 & 1 \\
CH$_3$OCHO & $7_{6,1}$ -- $6_{6,0}$ E $v_t=0$ & 85.919209 & -5.6138 & 30.0 & 40 & 2 \\
CH$_3$OCHO$^\dagger$ & $7_{6,2}$ -- $6_{6,1}$ E $v_t=0$ & 85.926553 & -5.6137 & 30.0 & 40 & 2 \\
CH$_3$OCHO$^\dagger$ & $7_{6,2}$ -- $6_{6,1}$ A $v_t=0$ & 85.927227 & -5.6136 & 30.0 & 40 & 2 \\
CH$_3$OCHO$^\dagger$ & $7_{6,1}$ -- $6_{6,0}$ A $v_t=0$ & 85.927227 & -5.6136 & 30.0 & 40 & 2 \\
CH$_3$OCHO & $7_{4,4}$ -- $7_{3,5}$ E $v_t=0$ & 96.507882 & -6.1542 & 30.0 & 27 & 7 \\
CH$_3$OCHO & $8_{4,5}$ -- $8_{3,6}$ A $v_t=0$ & 96.709259 & -5.9454 & 34.0 & 31 & 9 \\
CH$_3$OCHO & $7_{4,3}$ -- $7_{3,5}$ E $v_t=0$ & 96.776715 & -6.4523 & 30.0 & 27 & 9 \\
CH$_3$OCHO & $5_{4,2}$ -- $5_{3,3}$ A $v_t=0$ & 96.794121 & -6.1026 & 22.0 & 19 & 9 \\
CH$_3$OCHO & $8_{5,3}$ -- $7_{5,2}$ E $v_t=1$ & 97.577303 & -5.0823 & 34.0 & 225 & 10 \\
CH$_3$OCHO & $8_{3,6}$ -- $7_{3,5}$ A $v_t=1$ & 97.597161 & -4.9359 & 34.0 & 214 & 10 \\
CH$_3$OCHO$^\dagger$ & $8_{7,2}$ -- $7_{7,1}$ E $v_t=1$ & 97.65127 & -5.4962 & 34.0 & 240 & 10 \\
CH$_3$OCHO$^\dagger$ & $10_{4,7}$ -- $10_{3,8}$ E $v_t=0$ & 97.65127 & -5.9015 & 42.0 & 43 & 10 \\
CH$_3$OCHO & $8_{4,5}$ -- $7_{4,4}$ A $v_t=1$ & 97.661401 & -4.9937 & 34.0 & 219 & 10 \\
CH$_3$OCHO$^\ddagger$ & $10_{4,7}$ -- $10_{3,8}$ A $v_t=0$ & 97.69426 & -5.8957 & 42.0 & 43 & 10 \\
CH$_3$OCHO$^\ddagger$ & $12_{1,11}$ -- $12_{0,12}$ A $v_t=1$ & 97.727054 & -6.2029 & 50.0 & 234 & 10 \\
CH$_3$OCHO & $8_{6,3}$ -- $7_{6,2}$ E $v_t=1$ & 97.738738 & -5.2243 & 34.0 & 232 & 10 \\
CH$_3$OCHO & $8_{4,4}$ -- $7_{4,3}$ A $v_t=1$ & 97.752885 & -4.9924 & 34.0 & 219 & 10 \\
CH$_3$OCHO$^\ddagger$ & $8_{4,4}$ -- $8_{3,5}$ E $v_t=1$ & 97.871147 & -5.967 & 34.0 & 219 & 10 \\
CH$_3$OCHO$^\ddagger$ & $10_{4,7}$ -- $10_{3,8}$ A $v_t=1$ & 97.878933 & -5.8854 & 42.0 & 230 & 10 \\
CH$_3$OCHO & $8_{5,4}$ -- $7_{5,3}$ E $v_t=1$ & 97.885663 & -5.0787 & 34.0 & 224 & 10 \\
CH$_3$OCHO & $8_{4,4}$ -- $7_{4,3}$ E $v_t=1$ & 97.897118 & -4.9875 & 34.0 & 219 & 10 \\
CH$_3$OCHO$^\ddagger$ & $21_{4,17}$ -- $21_{3,18}$ A $v_t=1$ & 98.066305 & -5.7833 & 86.0 & 337 & 10 \\
CH$_3$OCHO$^\ddagger$ & $8_{4,5}$ -- $7_{4,4}$ E $v_t=1$ & 98.176293 & -4.9853 & 34.0 & 218 & 10 \\
CH$_3$OCHO & $8_{7,1}$ -- $7_{7,0}$ E $v_t=0$ & 98.182336 & -5.4902 & 34.0 & 53 & 10 \\
CH$_3$OCHO$^\dagger$ & $8_{7,2}$ -- $7_{7,1}$ A $v_t=0$ & 98.190658 & -5.4899 & 34.0 & 53 & 10 \\
CH$_3$OCHO$^\dagger$ & $8_{7,1}$ -- $7_{7,0}$ A $v_t=0$ & 98.190658 & -5.4899 & 34.0 & 53 & 10 \\
CH$_3$OCHO$^\dagger$ & $8_{7,2}$ -- $7_{7,1}$ E $v_t=0$ & 98.19146 & -5.49 & 34.0 & 53 & 10 \\
CH$_3$OCHO & $8_{6,2}$ -- $7_{6,1}$ E $v_t=0$ & 98.270501 & -5.218 & 34.0 & 45 & 10 \\
CH$_3$OCHO$^\dagger$ & $8_{6,3}$ -- $7_{6,2}$ E $v_t=0$ & 98.278921 & -5.2178 & 34.0 & 45 & 10 \\
CH$_3$OCHO$^\dagger$ & $8_{6,3}$ -- $7_{6,2}$ A $v_t=0$ & 98.279762 & -5.2178 & 34.0 & 45 & 10 \\
CH$_3$OCHO$^\dagger$ & $8_{6,2}$ -- $7_{6,1}$ A $v_t=0$ & 98.279762 & -5.2178 & 34.0 & 45 & 10 \\
CH$_3$OCHO$^\dagger$ & $9_{0,9}$ -- $8_{1,8}$ A $v_t=1$ & 98.423165 & -5.701 & 38.0 & 212 & 10 \\
CH$_3$OCHO$^\dagger$ & $8_{5,3}$ -- $7_{5,2}$ E $v_t=0$ & 98.424207 & -5.0722 & 34.0 & 37 & 10 \\
CH$_3$OCHO$^\dagger$ & $8_{5,4}$ -- $7_{5,3}$ E $v_t=0$ & 98.431803 & -5.072 & 34.0 & 37 & 10 \\
CH$_3$OCHO$^\dagger$ & $8_{5,4}$ -- $7_{5,3}$ A $v_t=0$ & 98.43276 & -5.0719 & 34.0 & 37 & 10 \\
CH$_3$OCHO & $8_{5,3}$ -- $7_{5,2}$ A $v_t=0$ & 98.435802 & -5.0719 & 34.0 & 37 & 10 \\
CH$_3$OCHO & $8_{4,5}$ -- $7_{4,3}$ E $v_t=0$ & 98.443186 & -6.4657 & 34.0 & 31 & 10 \\
\hline 
\multicolumn{7}{c}{Methyl Formate (CH$_3$O$^{13}$CHO)} \\
\hline 
CH$_3$O$^{13}$CHO$^\dagger$ & $8_{6,2}$ -- $7_{6,1}$ E $v_t=0$ & 97.541168 & -5.2186 & 34.0 & 44 & 10 \\
CH$_3$O$^{13}$CHO$^\dagger$ & $8_{-6,3}$ -- $7_{-6,2}$ E $v_t=0$ & 97.549352 & -5.2185 & 34.0 & 44 & 10 \\
CH$_3$O$^{13}$CHO$^\dagger$ & $8_{6,3}$ -- $7_{6,2}$ A $v_t=0$ & 97.55014 & -5.2185 & 34.0 & 44 & 10 \\
CH$_3$O$^{13}$CHO$^\dagger$ & $8_{6,2}$ -- $7_{6,1}$ A $v_t=0$ & 97.550183 & -5.2185 & 34.0 & 44 & 10 \\
CH$_3$O$^{13}$CHO$^\ddagger$ & $8_{-5,4}$ -- $7_{-5,3}$ E $v_t=0$ & 97.702479 & -5.0727 & 34.0 & 37 & 10 \\
CH$_3$O$^{13}$CHO$^\ddagger$ & $8_{5,4}$ -- $7_{5,3}$ A $v_t=0$ & 97.703323 & -5.0726 & 34.0 & 37 & 10 \\
CH$_3$O$^{13}$CHO$^\dagger$ & $8_{5,3}$ -- $7_{5,2}$ A $v_t=0$ & 97.706423 & -5.0726 & 34.0 & 37 & 10 \\
CH$_3$O$^{13}$CHO & $8_{-3,6}$ -- $7_{-3,5}$ E $v_t=0$ & 97.874331 & -4.9221 & 34.0 & 27 & 10 \\
CH$_3$O$^{13}$CHO$^\ddagger$ & $8_{3,6}$ -- $7_{3,5}$ A $v_t=0$ & 97.878443 & -4.9215 & 34.0 & 27 & 10 \\
CH$_3$O$^{13}$CHO$^\ddagger$ & $8_{4,4}$ -- $7_{4,3}$ E $v_t=0$ & 98.019644 & -4.9931 & 34.0 & 31 & 10 \\
\hline 
\multicolumn{7}{c}{Acetaldehyde (\acetaldehyde)} \\
\hline 
CH$_3$CHO & $9_{1,8}$ -- $9_{0,9}$ E $v_t=1$ & 85.947624 & -5.559 & 38.0 & 249 & 2 \\
CH$_3$CHO$^\dagger$ & $12_{2,11}$ -- $12_{1,11}$ E $v_t=1$ & 87.303557 & -6.1944 & 50.0 & 285 & 6 \\
CH$_3$CHO$^\dagger$ & $11_{4,7}$ -- $12_{3,9}$ E $v_t=1$ & 87.303714 & -6.2602 & 46.0 & 303 & 6 \\
CH$_3$CHO & $5_{2,3}$ -- $4_{2,2}$ E $v_t=0$ & 96.475524 & -4.6168 & 22.0 & 23 & 7 \\
CH$_3$CHO & $11_{4,8}$ -- $12_{3,10}$ E $v_t=0$ & 96.48895 & -6.1541 & 46.0 & 97 & 7 \\
CH$_3$CHO$^\dagger$ & $5_{3,3}$ -- $4_{3,2}$ A $v_t=2$ & 96.716114 & -4.7374 & 22.0 & 420 & 9 \\
CH$_3$CHO$^\dagger$ & $5_{3,2}$ -- $4_{3,1}$ A $v_t=2$ & 96.717473 & -4.7374 & 22.0 & 420 & 9 \\
CH$_3$CHO$^\dagger$ & $5_{2,3}$ -- $4_{2,2}$ A $v_t=1$ & 96.718409 & -4.6093 & 22.0 & 228 & 9 \\
CH$_3$CHO & $7_{0,7}$ -- $6_{1,6}$ A $v_t=0$ & 96.765371 & -5.5556 & 30.0 & 25 & 9 \\
CH$_3$CHO & $5_{2,4}$ -- $4_{2,3}$ E $v_t=1$ & 96.800291 & -4.613 & 22.0 & 226 & 9 \\
CH$_3$CHO & $5_{3,2}$ -- $4_{3,1}$ E $v_t=2$ & 97.612131 & -4.9808 & 22.0 & 419 & 10 \\
CH$_3$CHO & $19_{2,18}$ -- $18_{3,15}$ E $v_t=0$ & 97.796104 & -6.1248 & 78.0 & 183 & 10 \\
CH$_3$CHO & $10_{4,7}$ -- $11_{3,8}$ A $v_t=1$ & 97.941422 & -6.1583 & 42.0 & 290 & 10 \\
CH$_3$CHO & $21_{3,18}$ -- $20_{4,17}$ A $v_t=0$ & 98.20169 & -6.01 & 86.0 & 235 & 10 \\
CH$_3$CHO & $6_{3,3}$ -- $7_{2,5}$ E $v_t=0$ & 98.368631 & -6.245 & 26.0 & 39 & 10 \\
\hline 
\multicolumn{7}{c}{Acetaldehyde (CH$_3^{13}$CHO)} \\
\hline 
CH$_3^{13}$CHO$^\ddagger$ & $5_{2,3}$ -- $4_{2,2}$ A $v_t=0$ & 96.494465 & -4.5971 & 11.0 & 22 & 7 \\
\hline 
\multicolumn{7}{c}{Acetaldehyde (CH$_3$CDO)} \\
\hline 
CH$_3$CDO & $5_{1,4}$ -- $4_{1,3}$ E $v_t=0$ & 97.81231 & -4.5235 & 11.0 & 15 & 10 \\
CH$_3$CDO & $5_{1,4}$ -- $4_{1,3}$ A $v_t=0$ & 97.828514 & -4.5234 & 11.0 & 15 & 10 \\
\hline 
\multicolumn{7}{c}{Acetaldehyde (CH$_2$DCHO)} \\
\hline 
CH$_2$DCHO & $11_{2,9}i$ -- $11_{1,10}i$ & 98.412832 & -5.2329 & 23.0 & 66 & 10 \\
\hline 
\multicolumn{7}{c}{Dimethyl Ether (\dimethylether)} \\
\hline 
CH$_3$OCH$_3$$^\dagger$ & $16_{3,14}$ -- $15_{4,11}$ AA & 97.990629 & -5.8984 & 330.0 & 136 & 10 \\
CH$_3$OCH$_3$$^\dagger$ & $16_{3,14}$ -- $15_{4,11}$ EE & 97.993382 & -5.8983 & 528.0 & 136 & 10 \\
CH$_3$OCH$_3$$^\dagger$ & $16_{3,14}$ -- $15_{4,11}$ EA & 97.996098 & -5.8984 & 132.0 & 136 & 10 \\
CH$_3$OCH$_3$$^\dagger$ & $16_{3,14}$ -- $15_{4,11}$ AE & 97.996174 & -5.8984 & 198.0 & 136 & 10 \\
\hline 
\multicolumn{7}{c}{Acetone (\acetone)} \\
\hline 
CH$_3$COCH$_3$$^\dagger$ & $17_{6,11}$ -- $17_{5,12}$ AE & 97.929123 & -4.7694 & 210.0 & 110 & 10 \\
CH$_3$COCH$_3$$^\dagger$ & $17_{6,11}$ -- $17_{5,12}$ & 97.929247 & -4.7694 & 140.0 & 110 & 10 \\
CH$_3$COCH$_3$$^\dagger$ & $17_{7,11}$ -- $17_{6,12}$ AE & 97.930235 & -4.7694 & 70.0 & 110 & 10 \\
CH$_3$COCH$_3$$^\dagger$ & $17_{7,11}$ -- $17_{6,12}$ & 97.930344 & -4.7693 & 140.0 & 110 & 10 \\
CH$_3$COCH$_3$$^\dagger$ & $17_{6,11}$ -- $17_{5,12}$ EE & 98.052399 & -4.7674 & 560.0 & 110 & 10 \\
CH$_3$COCH$_3$$^\dagger$ & $17_{7,11}$ -- $17_{6,12}$ EE & 98.053535 & -4.7674 & 560.0 & 110 & 10 \\
\hline 
\multicolumn{7}{c}{Ethylene Oxide (\ethyleneoxide)} \\
\hline 
$c$-C$_2$H$_4$O$^\ddagger$ & $11_{9,3}$ -- $11_{8,4}$ & 96.501033 & -5.0737 & 115.0 & 146 & 7 \\
$c$-C$_2$H$_4$O$^\ddagger$ & $12_{9,4}$ -- $12_{8,5}$ & 97.728742 & -4.9909 & 75.0 & 169 & 10 \\
\hline 
\multicolumn{7}{c}{Propenal (\propenal)} \\
\hline 
$t$-C$_2$H$_3$CHO$^\ddagger$ & $11_{2,10}$ -- $10_{2,9}$ & 97.815592 & -4.3286 & 23.0 & 36 & 10 \\
$t$-C$_2$H$_3$CHO$^\ddagger$ & $11_{8,3}$ -- $10_{8,2}$ & 97.947054 & -4.6391 & 23.0 & 159 & 10 \\
$t$-C$_2$H$_3$CHO$^\ddagger$ & $11_{8,4}$ -- $10_{8,3}$ & 97.947054 & -4.6391 & 23.0 & 159 & 10 \\
$t$-C$_2$H$_3$CHO$^\ddagger$ & $11_{7,5}$ -- $10_{7,4}$ & 97.947549 & -4.5377 & 23.0 & 129 & 10 \\
$t$-C$_2$H$_3$CHO$^\ddagger$ & $11_{7,4}$ -- $10_{7,3}$ & 97.94755 & -4.5377 & 23.0 & 129 & 10 \\
$t$-C$_2$H$_3$CHO$^\ddagger$ & $11_{9,2}$ -- $10_{9,1}$ & 97.948051 & -4.793 & 23.0 & 194 & 10 \\
$t$-C$_2$H$_3$CHO$^\ddagger$ & $11_{9,3}$ -- $10_{9,2}$ & 97.948051 & -4.793 & 23.0 & 194 & 10 \\
$t$-C$_2$H$_3$CHO$^\ddagger$ & $11_{10,1}$ -- $10_{10,0}$ & 97.950017 & -5.0727 & 23.0 & 233 & 10 \\
$t$-C$_2$H$_3$CHO$^\ddagger$ & $11_{10,2}$ -- $10_{10,1}$ & 97.950017 & -5.0727 & 23.0 & 233 & 10 \\
$t$-C$_2$H$_3$CHO$^\ddagger$ & $11_{6,5}$ -- $10_{6,4}$ & 97.950286 & -4.4656 & 23.0 & 102 & 10 \\
$t$-C$_2$H$_3$CHO$^\ddagger$ & $11_{6,6}$ -- $10_{6,5}$ & 97.950286 & -4.4656 & 23.0 & 102 & 10 \\
$t$-C$_2$H$_3$CHO$^\ddagger$ & $11_{5,7}$ -- $10_{5,6}$ & 97.957003 & -4.4126 & 23.0 & 79 & 10 \\
$t$-C$_2$H$_3$CHO$^\ddagger$ & $11_{5,6}$ -- $10_{5,5}$ & 97.957003 & -4.4126 & 23.0 & 79 & 10 \\
$t$-C$_2$H$_3$CHO$^\dagger$ & $11_{4,8}$ -- $10_{4,7}$ & 97.972119 & -4.3735 & 23.0 & 61 & 10 \\
$t$-C$_2$H$_3$CHO$^\dagger$ & $11_{4,7}$ -- $10_{4,6}$ & 97.972119 & -4.3735 & 23.0 & 61 & 10 \\
$t$-C$_2$H$_3$CHO & $11_{3,9}$ -- $10_{3,8}$ & 98.001321 & -4.3451 & 23.0 & 46 & 10 \\
$t$-C$_2$H$_3$CHO$^\ddagger$ & $11_{3,8}$ -- $10_{3,7}$ & 98.019044 & -4.3448 & 23.0 & 46 & 10 \\
\hline 
\multicolumn{7}{c}{Propanal (\propanal)} \\
\hline 
$s$-C$_2$H$_5$CHO$^\ddagger$ & $14_{4,11}$ -- $14_{3,12}$ E & 87.307042 & -5.1587 & 29.0 & 62 & 6 \\
$s$-C$_2$H$_5$CHO$^\ddagger$ & $14_{4,11}$ -- $14_{3,12}$ A & 87.307594 & -5.1587 & 29.0 & 62 & 6 \\
$s$-C$_2$H$_5$CHO$^\ddagger$ & $29_{6,23}$ -- $29_{5,24}$ E & 96.793208 & -4.8841 & 59.0 & 245 & 9 \\
$s$-C$_2$H$_5$CHO$^\ddagger$ & $29_{6,23}$ -- $29_{5,24}$ A & 96.794186 & -4.8841 & 59.0 & 245 & 9 \\
$s$-C$_2$H$_5$CHO$^\dagger$ & $12_{5,7}$ -- $12_{4,8}$ E & 98.378025 & -5.0192 & 25.0 & 53 & 10 \\
$s$-C$_2$H$_5$CHO$^\dagger$ & $12_{5,7}$ -- $12_{4,8}$ A & 98.378025 & -5.0191 & 25.0 & 53 & 10 \\
$s$-C$_2$H$_5$CHO$^\dagger$ & $29_{7,22}$ -- $29_{6,23}$ A & 98.39443 & -4.847 & 59.0 & 250 & 10 \\
$s$-C$_2$H$_5$CHO$^\dagger$ & $29_{7,22}$ -- $29_{6,23}$ E & 98.394692 & -4.847 & 59.0 & 250 & 10 \\
\hline 
\multicolumn{7}{c}{Others} \\
\hline 
C$_2$H$_3$CN & $9_{1,8}$ -- $8_{1,7}$ & 87.312812 & -4.2778 & 57.0 & 23 & 6 \\
SO$_2$$^\ddagger$ & $7_{3,5}$ -- $8_{2,6}$ & 97.702334 & -5.7413 & 15.0 & 47 & 10 \\
OCS & $7$ -- $6$ & 85.139103 & -5.7658 & 15.0 & 16 & 3
\enddata
\tablenotetext{\dagger}{Blended with the other transitions of the same species.}
\tablenotetext{\ddagger}{Blended with transitions of other species.}
\end{deluxetable*}





% \startlongtable
% \begin{deluxetable*}{ccCCCCc}
% \tablecaption{Detected Transitions for Each Species\label{tab:transitions}}
% \tablehead{\colhead{Species} & \colhead{Transition} & \colhead{$\nu_0$ (GHz)} & \colhead{log$_{10}A_\mathrm{ul}$ (s$^{-1}$)} & \colhead{$E_\mathrm{up}$ (K)} & \colhead{$g_\mathrm{up}$} & \colhead{SPW}}
% \decimals
% \startdata
% \hline 
% \multicolumn{7}{c}{Methanol (\methanol)} \\
% \hline 
% CH$_3$OH$^\ddagger$ & $2_{-1,2}$ -- $1_{-1,1}$ E $v_t=1$ & 96.492163 & -5.5963 & 298.42 & 20 & 7 \\
% CH$_3$OH$^\ddagger$ & $2_{0,2}$ -- $1_{0,1}$ E $v_t=1$ & 96.493551 & -5.4706 & 307.52 & 20 & 7 \\
% CH$_3$OH$^\ddagger$ & $2_{1,1}$ -- $1_{1,0}$ E $v_t=1$ & 96.501713 & -5.5967 & 420.32 & 20 & 7 \\
% CH$_3$OH & $2_{0,2}$ -- $1_{0,1}$ A $v_t=1$ & 96.513686 & -5.4709 & 430.6 & 20 & 7 \\
% CH$_3$OH$^\dagger$ & $2_{1,2}$ -- $1_{1,1}$ E $v_t=0$ & 96.739358 & -5.5923 & 12.54 & 20 & 9 \\
% CH$_3$OH$^\dagger$ & $2_{0,2}$ -- $1_{0,1}$ A $v_t=0$ & 96.741371 & -5.4676 & 6.96 & 20 & 9 \\
% CH$_3$OH & $2_{0,2}$ -- $1_{0,1}$ E $v_t=0$ & 96.744545 & -5.4676 & 20.09 & 20 & 9 \\
% CH$_3$OH & $2_{-1,1}$ -- $1_{-1,0}$ E $v_t=0$ & 96.755501 & -5.581 & 28.01 & 20 & 9 \\
% CH$_3$OH & $2_{1,1}$ -- $1_{1,0}$ A $v_t=0$ & 97.582798 & -5.5807 & 21.56 & 20 & 10 \\
% CH$_3$OH$^\dagger$ & $21_{6,16}$ -- $22_{5,17}$ A $v_t=0$ & 97.677684 & -5.8404 & 729.28 & 172 & 10 \\
% CH$_3$OH$^\dagger$ & $21_{6,15}$ -- $22_{5,18}$ A $v_t=0$ & 97.678803 & -5.8403 & 729.28 & 172 & 10 \\
% \hline 
% \multicolumn{7}{c}{Methanol (CH$_2$DOH)} \\
% \hline 
% CH$_2$DOH$^\ddagger$ & $2_{1,1}$ -- $2_{0,2}$ e0 & 86.668751 & -5.3322 & 10.6 & 5 & 5 \\
% CH$_2$DOH$^\ddagger$ & $4_{1,3}$ -- $4_{0,4}$ o1 & 97.870192 & -5.3363 & 44.47 & 9 & 10 \\
% CH$_2$DOH & $4_{0,4}$ -- $3_{1,3}$ e0 & 98.031213 & -5.616 & 21.45 & 9 & 10 \\
% \hline 
% \multicolumn{7}{c}{Methyl Formate (\methylformate)} \\
% \hline 
% CH$_3$OCHO & $7_{6,1}$ -- $6_{6,0}$ E $v_t=1$ & 85.157135 & -5.6248 & 228.15 & 30 & 2 \\
% CH$_3$OCHO$^\dagger$ & $7_{5,3}$ -- $6_{5,2}$ A $v_t=1$ & 85.185466 & -5.3602 & 220.9 & 30 & 2 \\
% CH$_3$OCHO$^\dagger$ & $7_{5,2}$ -- $6_{5,1}$ A $v_t=1$ & 85.186063 & -5.3602 & 220.9 & 30 & 2 \\
% CH$_3$OCHO & $7_{4,3}$ -- $6_{4,2}$ E $v_t=1$ & 85.506219 & -5.2142 & 214.56 & 30 & 1 \\
% CH$_3$OCHO & $7_{5,3}$ -- $6_{5,2}$ E $v_t=1$ & 85.55338 & -5.3527 & 220.04 & 30 & 1 \\
% CH$_3$OCHO & $7_{6,1}$ -- $6_{6,0}$ E $v_t=0$ & 85.919209 & -5.6138 & 40.44 & 30 & 0 \\
% CH$_3$OCHO$^\dagger$ & $7_{6,2}$ -- $6_{6,1}$ E $v_t=0$ & 85.926553 & -5.6137 & 40.42 & 30 & 0 \\
% CH$_3$OCHO$^\dagger$ & $7_{6,2}$ -- $6_{6,1}$ A $v_t=0$ & 85.927227 & -5.6136 & 40.42 & 30 & 0 \\
% CH$_3$OCHO$^\dagger$ & $7_{6,1}$ -- $6_{6,0}$ A $v_t=0$ & 85.927227 & -5.6136 & 40.42 & 30 & 0 \\
% CH$_3$OCHO & $7_{4,4}$ -- $7_{3,5}$ E $v_t=0$ & 96.507882 & -6.1542 & 27.16 & 30 & 7 \\
% CH$_3$OCHO & $8_{4,5}$ -- $8_{3,6}$ A $v_t=0$ & 96.709259 & -5.9454 & 31.89 & 34 & 9 \\
% CH$_3$OCHO & $7_{4,3}$ -- $7_{3,5}$ E $v_t=0$ & 96.776715 & -6.4523 & 27.17 & 30 & 9 \\
% CH$_3$OCHO & $5_{4,2}$ -- $5_{3,3}$ A $v_t=0$ & 96.794121 & -6.1026 & 19.47 & 22 & 9 \\
% CH$_3$OCHO & $8_{5,3}$ -- $7_{5,2}$ E $v_t=1$ & 97.577303 & -5.0823 & 225.36 & 34 & 10 \\
% CH$_3$OCHO & $8_{3,6}$ -- $7_{3,5}$ A $v_t=1$ & 97.597161 & -4.9359 & 214.94 & 34 & 10 \\
% CH$_3$OCHO$^\dagger$ & $8_{7,2}$ -- $7_{7,1}$ E $v_t=1$ & 97.65127 & -5.4962 & 240.82 & 34 & 10 \\
% CH$_3$OCHO$^\dagger$ & $10_{4,7}$ -- $10_{3,8}$ E $v_t=0$ & 97.65127 & -5.9015 & 43.17 & 42 & 10 \\
% CH$_3$OCHO & $8_{4,5}$ -- $7_{4,4}$ A $v_t=1$ & 97.661401 & -4.9937 & 219.6 & 34 & 10 \\
% CH$_3$OCHO$^\ddagger$ & $10_{4,7}$ -- $10_{3,8}$ A $v_t=0$ & 97.69426 & -5.8957 & 43.16 & 42 & 10 \\
% CH$_3$OCHO$^\ddagger$ & $12_{1,11}$ -- $12_{0,12}$ A $v_t=1$ & 97.727054 & -6.2029 & 234.75 & 50 & 10 \\
% CH$_3$OCHO & $8_{6,3}$ -- $7_{6,2}$ E $v_t=1$ & 97.738738 & -5.2243 & 232.1 & 34 & 10 \\
% CH$_3$OCHO & $8_{4,4}$ -- $7_{4,3}$ A $v_t=1$ & 97.752885 & -4.9924 & 219.61 & 34 & 10 \\
% CH$_3$OCHO$^\ddagger$ & $8_{4,4}$ -- $8_{3,5}$ E $v_t=1$ & 97.871147 & -5.967 & 219.26 & 34 & 10 \\
% CH$_3$OCHO$^\ddagger$ & $10_{4,7}$ -- $10_{3,8}$ A $v_t=1$ & 97.878933 & -5.8854 & 230.76 & 42 & 10 \\
% CH$_3$OCHO & $8_{5,4}$ -- $7_{5,3}$ E $v_t=1$ & 97.885663 & -5.0787 & 224.73 & 34 & 10 \\
% CH$_3$OCHO & $8_{4,4}$ -- $7_{4,3}$ E $v_t=1$ & 97.897118 & -4.9875 & 219.26 & 34 & 10 \\
% CH$_3$OCHO$^\ddagger$ & $21_{4,17}$ -- $21_{3,18}$ A $v_t=1$ & 98.066305 & -5.7833 & 337.99 & 86 & 10 \\
% CH$_3$OCHO$^\ddagger$ & $8_{4,5}$ -- $7_{4,4}$ E $v_t=1$ & 98.176293 & -4.9853 & 218.74 & 34 & 10 \\
% CH$_3$OCHO & $8_{7,1}$ -- $7_{7,0}$ E $v_t=0$ & 98.182336 & -5.4902 & 53.78 & 34 & 10 \\
% CH$_3$OCHO$^\dagger$ & $8_{7,2}$ -- $7_{7,1}$ A $v_t=0$ & 98.190658 & -5.4899 & 53.76 & 34 & 10 \\
% CH$_3$OCHO$^\dagger$ & $8_{7,1}$ -- $7_{7,0}$ A $v_t=0$ & 98.190658 & -5.4899 & 53.76 & 34 & 10 \\
% CH$_3$OCHO$^\dagger$ & $8_{7,2}$ -- $7_{7,1}$ E $v_t=0$ & 98.19146 & -5.49 & 53.76 & 34 & 10 \\
% CH$_3$OCHO & $8_{6,2}$ -- $7_{6,1}$ E $v_t=0$ & 98.270501 & -5.218 & 45.15 & 34 & 10 \\
% CH$_3$OCHO$^\dagger$ & $8_{6,3}$ -- $7_{6,2}$ E $v_t=0$ & 98.278921 & -5.2178 & 45.13 & 34 & 10 \\
% CH$_3$OCHO$^\dagger$ & $8_{6,3}$ -- $7_{6,2}$ A $v_t=0$ & 98.279762 & -5.2178 & 45.13 & 34 & 10 \\
% CH$_3$OCHO$^\dagger$ & $8_{6,2}$ -- $7_{6,1}$ A $v_t=0$ & 98.279762 & -5.2178 & 45.13 & 34 & 10 \\
% CH$_3$OCHO$^\dagger$ & $9_{0,9}$ -- $8_{1,8}$ A $v_t=1$ & 98.423165 & -5.701 & 212.64 & 38 & 10 \\
% CH$_3$OCHO$^\dagger$ & $8_{5,3}$ -- $7_{5,2}$ E $v_t=0$ & 98.424207 & -5.0722 & 37.86 & 34 & 10 \\
% CH$_3$OCHO$^\dagger$ & $8_{5,4}$ -- $7_{5,3}$ E $v_t=0$ & 98.431803 & -5.072 & 37.84 & 34 & 10 \\
% CH$_3$OCHO$^\dagger$ & $8_{5,4}$ -- $7_{5,3}$ A $v_t=0$ & 98.43276 & -5.0719 & 37.84 & 34 & 10 \\
% CH$_3$OCHO & $8_{5,3}$ -- $7_{5,2}$ A $v_t=0$ & 98.435802 & -5.0719 & 37.84 & 34 & 10 \\
% CH$_3$OCHO & $8_{4,5}$ -- $7_{4,3}$ E $v_t=0$ & 98.443186 & -6.4657 & 31.9 & 34 & 10 \\
% \hline 
% \multicolumn{7}{c}{Methyl Formate (CH$_3$O$^{13}$CHO)} \\
% \hline 
% CH$_3$O$^{13}$CHO$^\dagger$ & $8_{6,2}$ -- $7_{6,1}$ E $v_t=0$ & 97.541168 & -5.2186 & 44.76 & 34 & 10 \\
% CH$_3$O$^{13}$CHO$^\dagger$ & $8_{-6,3}$ -- $7_{-6,2}$ E $v_t=0$ & 97.549352 & -5.2185 & 44.74 & 34 & 10 \\
% CH$_3$O$^{13}$CHO$^\dagger$ & $8_{6,3}$ -- $7_{6,2}$ A $v_t=0$ & 97.55014 & -5.2185 & 44.74 & 34 & 10 \\
% CH$_3$O$^{13}$CHO$^\dagger$ & $8_{6,2}$ -- $7_{6,1}$ A $v_t=0$ & 97.550183 & -5.2185 & 44.74 & 34 & 10 \\
% CH$_3$O$^{13}$CHO$^\ddagger$ & $8_{-5,4}$ -- $7_{-5,3}$ E $v_t=0$ & 97.702479 & -5.0727 & 37.52 & 34 & 10 \\
% CH$_3$O$^{13}$CHO$^\ddagger$ & $8_{5,4}$ -- $7_{5,3}$ A $v_t=0$ & 97.703323 & -5.0726 & 37.51 & 34 & 10 \\
% CH$_3$O$^{13}$CHO$^\dagger$ & $8_{5,3}$ -- $7_{5,2}$ A $v_t=0$ & 97.706423 & -5.0726 & 37.51 & 34 & 10 \\
% CH$_3$O$^{13}$CHO & $8_{-3,6}$ -- $7_{-3,5}$ E $v_t=0$ & 97.874331 & -4.9221 & 27.04 & 34 & 10 \\
% CH$_3$O$^{13}$CHO$^\ddagger$ & $8_{3,6}$ -- $7_{3,5}$ A $v_t=0$ & 97.878443 & -4.9215 & 27.03 & 34 & 10 \\
% CH$_3$O$^{13}$CHO$^\ddagger$ & $8_{4,4}$ -- $7_{4,3}$ E $v_t=0$ & 98.019644 & -4.9931 & 31.65 & 34 & 10 \\
% \hline 
% \multicolumn{7}{c}{Acetaldehyde (\acetaldehyde)} \\
% \hline 
% CH$_3$CHO & $9_{1,8}$ -- $9_{0,9}$ E $v_t=1$ & 85.947624 & -5.559 & 249.81 & 38 & 0 \\
% CH$_3$CHO$^\dagger$ & $12_{2,11}$ -- $12_{1,11}$ E $v_t=1$ & 87.303557 & -6.1944 & 285.13 & 50 & 6 \\
% CH$_3$CHO$^\dagger$ & $11_{4,7}$ -- $12_{3,9}$ E $v_t=1$ & 87.303714 & -6.2602 & 303.24 & 46 & 6 \\
% CH$_3$CHO & $5_{2,3}$ -- $4_{2,2}$ E $v_t=0$ & 96.475524 & -4.6168 & 23.03 & 22 & 7 \\
% CH$_3$CHO & $11_{4,8}$ -- $12_{3,10}$ E $v_t=0$ & 96.48895 & -6.1541 & 97.14 & 46 & 7 \\
% CH$_3$CHO$^\dagger$ & $5_{3,3}$ -- $4_{3,2}$ A $v_t=2$ & 96.716114 & -4.7374 & 420.58 & 22 & 9 \\
% CH$_3$CHO$^\dagger$ & $5_{3,2}$ -- $4_{3,1}$ A $v_t=2$ & 96.717473 & -4.7374 & 420.58 & 22 & 9 \\
% CH$_3$CHO$^\dagger$ & $5_{2,3}$ -- $4_{2,2}$ A $v_t=1$ & 96.718409 & -4.6093 & 228.3 & 22 & 9 \\
% CH$_3$CHO & $7_{0,7}$ -- $6_{1,6}$ A $v_t=0$ & 96.765371 & -5.5556 & 25.78 & 30 & 9 \\
% CH$_3$CHO & $5_{2,4}$ -- $4_{2,3}$ E $v_t=1$ & 96.800291 & -4.613 & 226.45 & 22 & 9 \\
% CH$_3$CHO & $5_{3,2}$ -- $4_{3,1}$ E $v_t=2$ & 97.612131 & -4.9808 & 419.01 & 22 & 10 \\
% CH$_3$CHO & $19_{2,18}$ -- $18_{3,15}$ E $v_t=0$ & 97.796104 & -6.1248 & 183.87 & 78 & 10 \\
% CH$_3$CHO & $10_{4,7}$ -- $11_{3,8}$ A $v_t=1$ & 97.941422 & -6.1583 & 290.44 & 42 & 10 \\
% CH$_3$CHO & $21_{3,18}$ -- $20_{4,17}$ A $v_t=0$ & 98.20169 & -6.01 & 235.39 & 86 & 10 \\
% CH$_3$CHO & $6_{3,3}$ -- $7_{2,5}$ E $v_t=0$ & 98.368631 & -6.245 & 39.81 & 26 & 10 \\
% \hline 
% \multicolumn{7}{c}{Acetaldehyde (CH$_3^{13}$CHO)} \\
% \hline 
% CH$_3^{13}$CHO$^\ddagger$ & $5_{2,3}$ -- $4_{2,2}$ A $v_t=0$ & 96.494465 & -4.5971 & 22.74 & 11 & 7 \\
% \hline 
% \multicolumn{7}{c}{Acetaldehyde (CH$_3$CDO)} \\
% \hline 
% CH$_3$CDO & $5_{1,4}$ -- $4_{1,3}$ E $v_t=0$ & 97.81231 & -4.5235 & 15.86 & 11 & 10 \\
% CH$_3$CDO & $5_{1,4}$ -- $4_{1,3}$ A $v_t=0$ & 97.828514 & -4.5234 & 15.81 & 11 & 10 \\
% \hline 
% \multicolumn{7}{c}{Acetaldehyde (CH$_2$DCHO)} \\
% \hline 
% CH$_2$DCHO & $11_{2,9}i$ -- $11_{1,10}i$ & 98.412832 & -5.2329 & 66.88 & 23 & 10 \\
% \hline 
% \multicolumn{7}{c}{Dimethyl Ether (\dimethylether)} \\
% \hline 
% CH$_3$OCH$_3$$^\dagger$ & $16_{3,14}$ -- $15_{4,11}$ AA & 97.990629 & -5.8984 & 136.6 & 330 & 10 \\
% CH$_3$OCH$_3$$^\dagger$ & $16_{3,14}$ -- $15_{4,11}$ EE & 97.993382 & -5.8983 & 136.6 & 528 & 10 \\
% CH$_3$OCH$_3$$^\dagger$ & $16_{3,14}$ -- $15_{4,11}$ EA & 97.996098 & -5.8984 & 136.6 & 132 & 10 \\
% CH$_3$OCH$_3$$^\dagger$ & $16_{3,14}$ -- $15_{4,11}$ AE & 97.996174 & -5.8984 & 136.6 & 198 & 10 \\
% \hline 
% \multicolumn{7}{c}{Acetone (\acetone)} \\
% \hline 
% CH$_3$COCH$_3$$^\dagger$ & $17_{6,11}$ -- $17_{5,12}$ AE & 97.929123 & -4.7694 & 110.64 & 210 & 10 \\
% CH$_3$COCH$_3$$^\dagger$ & $17_{6,11}$ -- $17_{5,12}$ & 97.929247 & -4.7694 & 110.64 & 140 & 10 \\
% CH$_3$COCH$_3$$^\dagger$ & $17_{7,11}$ -- $17_{6,12}$ AE & 97.930235 & -4.7694 & 110.64 & 70 & 10 \\
% CH$_3$COCH$_3$$^\dagger$ & $17_{7,11}$ -- $17_{6,12}$ & 97.930344 & -4.7693 & 110.64 & 140 & 10 \\
% CH$_3$COCH$_3$$^\dagger$ & $17_{6,11}$ -- $17_{5,12}$ EE & 98.052399 & -4.7674 & 110.59 & 560 & 10 \\
% CH$_3$COCH$_3$$^\dagger$ & $17_{7,11}$ -- $17_{6,12}$ EE & 98.053535 & -4.7674 & 110.59 & 560 & 10 \\
% \hline 
% \multicolumn{7}{c}{Ethylene Oxide (\ethyleneoxide)} \\
% \hline 
% $c$-C$_2$H$_4$O$^\ddagger$ & $11_{9,3}$ -- $11_{8,4}$ & 96.501033 & -5.0737 & 146.75 & 115 & 7 \\
% $c$-C$_2$H$_4$O$^\ddagger$ & $12_{9,4}$ -- $12_{8,5}$ & 97.728742 & -4.9909 & 169.45 & 75 & 10 \\
% \hline 
% \multicolumn{7}{c}{Propenal (\propenal)} \\
% \hline 
% $t$-C$_2$H$_3$CHO$^\ddagger$ & $11_{2,10}$ -- $10_{2,9}$ & 97.815592 & -4.3286 & 36.42 & 23 & 10 \\
% $t$-C$_2$H$_3$CHO$^\ddagger$ & $11_{8,3}$ -- $10_{8,2}$ & 97.947054 & -4.6391 & 159.91 & 23 & 10 \\
% $t$-C$_2$H$_3$CHO$^\ddagger$ & $11_{8,4}$ -- $10_{8,3}$ & 97.947054 & -4.6391 & 159.91 & 23 & 10 \\
% $t$-C$_2$H$_3$CHO$^\ddagger$ & $11_{7,5}$ -- $10_{7,4}$ & 97.947549 & -4.5377 & 129.05 & 23 & 10 \\
% $t$-C$_2$H$_3$CHO$^\ddagger$ & $11_{7,4}$ -- $10_{7,3}$ & 97.94755 & -4.5377 & 129.05 & 23 & 10 \\
% $t$-C$_2$H$_3$CHO$^\ddagger$ & $11_{9,2}$ -- $10_{9,1}$ & 97.948051 & -4.793 & 194.87 & 23 & 10 \\
% $t$-C$_2$H$_3$CHO$^\ddagger$ & $11_{9,3}$ -- $10_{9,2}$ & 97.948051 & -4.793 & 194.87 & 23 & 10 \\
% $t$-C$_2$H$_3$CHO$^\ddagger$ & $11_{10,1}$ -- $10_{10,0}$ & 97.950017 & -5.0727 & 233.93 & 23 & 10 \\
% $t$-C$_2$H$_3$CHO$^\ddagger$ & $11_{10,2}$ -- $10_{10,1}$ & 97.950017 & -5.0727 & 233.93 & 23 & 10 \\
% $t$-C$_2$H$_3$CHO$^\ddagger$ & $11_{6,5}$ -- $10_{6,4}$ & 97.950286 & -4.4656 & 102.3 & 23 & 10 \\
% $t$-C$_2$H$_3$CHO$^\ddagger$ & $11_{6,6}$ -- $10_{6,5}$ & 97.950286 & -4.4656 & 102.3 & 23 & 10 \\
% $t$-C$_2$H$_3$CHO$^\ddagger$ & $11_{5,7}$ -- $10_{5,6}$ & 97.957003 & -4.4126 & 79.67 & 23 & 10 \\
% $t$-C$_2$H$_3$CHO$^\ddagger$ & $11_{5,6}$ -- $10_{5,5}$ & 97.957003 & -4.4126 & 79.67 & 23 & 10 \\
% $t$-C$_2$H$_3$CHO$^\dagger$ & $11_{4,8}$ -- $10_{4,7}$ & 97.972119 & -4.3735 & 61.15 & 23 & 10 \\
% $t$-C$_2$H$_3$CHO$^\dagger$ & $11_{4,7}$ -- $10_{4,6}$ & 97.972119 & -4.3735 & 61.15 & 23 & 10 \\
% $t$-C$_2$H$_3$CHO & $11_{3,9}$ -- $10_{3,8}$ & 98.001321 & -4.3451 & 46.74 & 23 & 10 \\
% $t$-C$_2$H$_3$CHO$^\ddagger$ & $11_{3,8}$ -- $10_{3,7}$ & 98.019044 & -4.3448 & 46.74 & 23 & 10 \\
% \hline 
% \multicolumn{7}{c}{Propanal (\propanal)} \\
% \hline 
% $s$-C$_2$H$_5$CHO$^\ddagger$ & $14_{4,11}$ -- $14_{3,12}$ E & 87.307042 & -5.1587 & 62.19 & 29 & 6 \\
% $s$-C$_2$H$_5$CHO$^\ddagger$ & $14_{4,11}$ -- $14_{3,12}$ A & 87.307594 & -5.1587 & 62.19 & 29 & 6 \\
% $s$-C$_2$H$_5$CHO$^\ddagger$ & $29_{6,23}$ -- $29_{5,24}$ E & 96.793208 & -4.8841 & 245.73 & 59 & 9 \\
% $s$-C$_2$H$_5$CHO$^\ddagger$ & $29_{6,23}$ -- $29_{5,24}$ A & 96.794186 & -4.8841 & 245.73 & 59 & 9 \\
% $s$-C$_2$H$_5$CHO$^\dagger$ & $12_{5,7}$ -- $12_{4,8}$ E & 98.378025 & -5.0192 & 53.26 & 25 & 10 \\
% $s$-C$_2$H$_5$CHO$^\dagger$ & $12_{5,7}$ -- $12_{4,8}$ A & 98.378025 & -5.0191 & 53.26 & 25 & 10 \\
% $s$-C$_2$H$_5$CHO$^\dagger$ & $29_{7,22}$ -- $29_{6,23}$ A & 98.39443 & -4.847 & 250.45 & 59 & 10 \\
% $s$-C$_2$H$_5$CHO$^\dagger$ & $29_{7,22}$ -- $29_{6,23}$ E & 98.394692 & -4.847 & 250.45 & 59 & 10 \\
% \hline 
% \multicolumn{7}{c}{Others} \\
% \hline 
% C$_2$H$_3$CN & $9_{1,8}$ -- $8_{1,7}$ & 87.312812 & -4.2778 & 23.13 & 57 & 6 \\
% SO$_2$$^\ddagger$ & $7_{3,5}$ -- $8_{2,6}$ & 97.702334 & -5.7413 & 47.84 & 15 & 10 \\
% OCS & $7$ -- $6$ & 85.139103 & -5.7658 & 16.34 & 15 & 2
% \enddata
% \tablenotetext{\dagger}{Blended with the other transitions of the same species.}
% \tablenotetext{\ddagger}{Blended with transitions of other species.}
% \end{deluxetable*}





\section{Disk-integrated Spectra}\label{appendix:spectra}
Figure \ref{fig:disk-integrated_spectra0}--\ref{fig:disk-integrated_spectra3} show the disk-integrated spectra for all SPWs. The line profile of each transition exhibits the double-peaked feature, which is typical for Keplerian rotation, with some spectral blending. Figure \ref{fig:disk-integrated_spectra_fit0}--\ref{fig:disk-integrated_spectra_fit3} shows the disk-integrated spectra with spectral alignment, overlaid by the best-fit model of the spectral fit (Section \ref{subsec:spectral_fit}). The double-peaked profiles are aligned and appeared to be single-peaked, which are well reproduced with the best-fit model including spectral blending. 

\begin{figure*}
\epsscale{1.15}
\plotone{disk-integrated_spectra_0.pdf}
\caption{Disk-integrated spectra toward V883 Ori without spectral alignment. The emission is integrated over the deprojected disk region with the outer radius of 0\farcs6 (or 240 au).}
\label{fig:disk-integrated_spectra0}
\end{figure*}

\begin{figure*}
\epsscale{1.15}
\plotone{disk-integrated_spectra_1.pdf}
\caption{Continuation of Figure \ref{fig:disk-integrated_spectra0}.}
\label{fig:disk-integrated_spectra1}
\end{figure*}

\begin{figure*}
\epsscale{1.15}
\plotone{disk-integrated_spectra_2.pdf}
\caption{Continuation of Figure \ref{fig:disk-integrated_spectra1}.}
\label{fig:disk-integrated_spectra2}
\end{figure*}

\begin{figure*}
\epsscale{1.15}
\plotone{disk-integrated_spectra_3.pdf}
\caption{Continuation of Figure \ref{fig:disk-integrated_spectra2}.}
\label{fig:disk-integrated_spectra3}
\end{figure*}

\begin{figure*}
\epsscale{1.15}
\plotone{disk-integrated_spectra_fit_0.pdf}
\caption{Disk-integrated spectra toward V883 Ori after the spectral alignment, overlaid with the best-fit model. The black solid lines indicate the full model composed of all detected species, while dashed colored lines are the model for each species shown in the legend. The gray-shaded region are removed from the fit (see Section \ref{subsec:spectral_fit} for details.)}
\label{fig:disk-integrated_spectra_fit0}
\end{figure*}

\begin{figure*}
\epsscale{1.15}
\plotone{disk-integrated_spectra_fit_1.pdf}
\caption{Continuation of Figure \ref{fig:disk-integrated_spectra_fit0}.}
\label{fig:disk-integrated_spectra_fit1}
\end{figure*}

\begin{figure*}
\epsscale{1.15}
\plotone{disk-integrated_spectra_fit_2.pdf}
\caption{Continuation of Figure \ref{fig:disk-integrated_spectra_fit1}.}
\label{fig:disk-integrated_spectra_fit2}
\end{figure*}

\begin{figure*}
\epsscale{1.15}
\plotone{disk-integrated_spectra_fit_3.pdf}
\caption{Continuation of Figure \ref{fig:disk-integrated_spectra_fit2}.}
\label{fig:disk-integrated_spectra_fit3}
\end{figure*}

% \begin{figure*}
% \epsscale{1.15}
% \plotone{CH3OH_spectra.pdf}
% \caption{\methanol spectrum for each transition without spectral line blending.}
% \label{fig:CH3OH_spectra}
% \end{figure*}

% \begin{figure*}
% \epsscale{1.15}
% \plotone{CH3OCHO_spectra.pdf}
% \caption{Same as Figure \ref{fig:CH3OH_spectra}, but for \methylformate.}
% \label{fig:CH3OCHO_spectra}
% \end{figure*}

% \begin{figure*}
% \epsscale{1.15}
% \plotone{CH3CHO_spectra.pdf}
% \caption{Same as Figure \ref{fig:CH3OH_spectra}, but for \acetaldehyde.}
% \label{fig:CH3CHO_spectra}
% \end{figure*}

\section{LTE spectral model}\label{appendix:spectral_model}
Here we describe the details of the spectral model which is fitted to the spectra corrected for Keplerian rotation for column density derivation. We assume that the excitation condition of observed COM emission is well approximated by local thermodynamic equilibrium (LTE), where the gas temperature is equal to the excitation temperature ($T_\mathrm{ex}$) of the emitting molecules. This assumption should be reasonable as the typical gas density in protoplanetary disks () is higher than the typical critical density of the COM transitions. For example, the critical density of xx-yy transition of \methanol is calculated to be zzz cm-3 at a gas kinetic temperature of 100 K \citep[ref][]{}. We further assume that different species are sufficiently cospatial to share the same emitting region size and excitation temperature. This assumption is also broadly supported by the similar emission extent of different COMs shown in Figure \ref{fig:moment_zero_gallery}. 

The model intensity $I_\nu$ are computed based on a solution of the basic radiative transfer equations for an isothermal, uniform slab:
\begin{equation}
    I_\nu = (B_\nu(T_\mathrm{ex}) - B_\nu(T_\mathrm{CMB})) (1 - e^{-\tau_\nu}),
\end{equation}
where $B_\nu$ is the Planck function for a blackbody radiation, $T_\mathrm{CMB} = 2.73$ K is the temperature of the cosmic microwave background, and $\tau_\nu$ is the optical depth of the line emission at a frequency $\nu$. Following the formulation described in Appendix A of \citet{Yamato2022}, the line optical depth $\tau_\nu$ is computed considering different transitions from various species as
\begin{equation}
    \tau_\nu = \sum_{i}\tau_{0, i} \exp\bigg(-\frac{(\nu - \nu_\mathrm{c})^2}{2 \sigma_\nu^2}\bigg),
\end{equation}
where $i$ is the indices for different transitions, $\tau_{0, i}$ is the optical depth at the line center of $i$th transition, $\nu_\mathrm{c}$ is the central frequency of the transition, and $\sigma_\nu$ is the line width in terms of frequency. The central frequency $\nu_\mathrm{c}$ is calculated as $\nu_\mathrm{c} = \nu_0 (1 - v_\mathrm{sys}/c)$, where $c$ is the speed of light and $v_\mathrm{sys}$ is the systemic velocity of the source, assumed to be $4.25$ km s$^{-1}$ for V883 Ori \citep{Tobin2023}. The frequency line width $\sigma_\nu$ is converted from the velocity line width $\sigma_v$ as $\sigma_\nu = \nu_0 \sigma_v / c$, where $\nu_0$ is the rest frequency of each transition listed in Table \ref{tab:transitions}. In practice, the full width of half maximum (FWHM) of the velocity $\Delta V_\mathrm{FWHM}$ is used as a parameter instead of $\sigma_\nu$ or $\sigma_v$. The optical depth at the line center ($\tau_\mathrm{0, i}$) is calculated as
\begin{equation}
    \tau_{0, i} = \frac{c^2A_\mathrm{u}N_\mathrm{u}}{8\pi\nu_0^2\sqrt{2\pi}\sigma_\nu}\Bigg(\exp\left(\frac{h\nu_0}{k_\mathrm{B}T_\mathrm{ex}}\right) - 1\Bigg),
\end{equation}
\begin{equation}
    \frac{N_\mathrm{u}}{N} = \frac{g_\mathrm{u}}{Q(T_\mathrm{ex})}\exp\left(-\frac{E_\mathrm{u}}{k_\mathrm{B}T_\mathrm{ex}}\right),
\end{equation}
where $A_\mathrm{ul}$ is the Einstein A coefficient for spontaneous emission, $g_\mathrm{u}$ is the upper state degeneracy, $N$ is the molecular column density, $Q$ is the partition function of the molecule, and $E_\mathrm{u}$ is the upper state energy of the transition. While this suit of formulation is similar to that of the eXtended CASA Line Analysis Software Suite (XCLASS; \citealt{Moller2017}), which is used in a previous work on the V883 Ori disk \citep{Lee2019}, we used an independent implementation by ourselves for technical flexibility.

The model intensity $I_\nu$ (in unit of Jy sr$^{-1}$) is then integrated over a solid angle of the emitting region $\Omega$, which is common for all transitions and species, to obtain the spectra of flux density (in unit of Jy). 
% In practice, we assumed an axisymmetric emission distribution in the disk and the radius of the emitting region $r$ (considering the disk inclination) is used as a parameter, i.e., $\Omega = \pi r^2 \cos i$. 
Finally, the model flux density spectra are convolved with a Lorentz function $f(\nu)$ with a width of $\gamma$,
\begin{equation}
    f(\nu) = \frac{1}{\pi\gamma} \frac{\gamma^2}{\gamma^2 + \nu^2},
\end{equation}
to take line broadening in the spectra corrected for Keplerian rotation. As shown in Figure \ref{fig:Lorentzian_demo}, the spectra corrected for Keplerian rotation still deviates from a simple Gaussian due to (1) the finite spatial resolution or beam smearing, which cannot fully resolve the highest velocity component of Keplerian rotation and (2) potential emission from the elevated disk surface of the back side and front side of the disk, which have different projected velocities. In our data, both (1) and (2) could be dominant causes of the deviation, since the spatial resolution is not so high to fully resolve the disk emission and the COM emission could originate from the warm disk surface in addition to the midplane, as already shown in the Band 6 data (M. Leemker et al. in preparation). Figure \ref{fig:Lorentzian_demo} demonstrates that the convolution with the Lorentz function well approximates the deviation from Gaussian while conserving the velocity-integrated flux density, which is relevant for column density derivation. Furthermore, to account for the finite spectral resolution, the model spectra is additionally convolved with a Gaussian function with a FWHM of the spectral resolution. This process also accounts for the different spectral resolutions (per channel width) between SPW 10 and others (see Table \ref{tab:cube_properties}). A similar approach has been employed in the spectral line analysis in protoplanetary disks \citep{Cataldi2021}, where they used only a Gaussian convolution to take the line broadening into account (see also \citealt{Bergner2021, Guzman2021}).



% In total, we consider 19 free parameters:  $r, T_\mathrm{ex}, \gamma, \Delta V_\mathrm{FWHM}$, and $N$ for 15 species. The fit 

\begin{figure}
% \epsscale{1.15}
\plotone{line_width_demonstration.pdf}
\caption{Fit of a Gaussian + Lorentzian model (i.e., Voigt profile; blue) and a Gaussian model (orange) to the observed spectra of a \acetaldehyde transition at 96.476 GHz (gray). The Gaussian model fit used only the core of the spectra ($|v_\mathrm{LSRK} - v_\mathrm{sys}| \leq 2$ km s$^{-1}$). The Gaussian + Lorentzian model better reproduces the wing of the spectra. The horizontal dashed line and vertical dotted line marks the zero-flux level and the systemic velocity ($v_\mathrm{sys} = 4.25$ km s$^{-1}$), respectively.}
\label{fig:Lorentzian_demo}
\end{figure}

\section{Reconstruction of Radial Intensity Profile}\label{appendix:line_profile_analysis_method}
Here we describe details of the forward modeling approach to reconstruct the radial intensity profiles from the line profiles (Section \ref{subsubsec:line_profile_analysis}). The method we employed is similar to the way described in \citet{Bosman2021} but slightly different. First we create the model emission component originating from an annulus at a given radius of the inclined disk. We made a grid of cube and calculated the line shape of the spatially integrated emission as a function of velocity assuming that the emission is completely originating from the midplane of the Keplerian-rotating disk, i.e.,
\begin{equation}\label{eq:line_profile_model}
    F_{r_i}(v) \propto \int_{\Omega_\mathrm{annulus}}I_{r_i}\exp\left[- \frac{(v - v_\mathrm{kep}(r, \theta) - v_\mathrm{sys})^2}{2\sigma_v^2}\right]d\Omega,
\end{equation}
where $\Omega_\mathrm{annulus}$ is the solid angle of the annulus, $I_{r_i}$ is the surface brightness (in Jy au$^{-2}$ km s$^{-1}$) at the radius of $r_i$, $v_\mathrm{kep}(r, \theta) = \sqrt{GM_\star/r}\,\cos\theta\,\sin i$ is the Keplerian velocity at the polar disk coordinate $(r, \theta)$, $v_\mathrm{sys}$ is the systemic velocity of the source (fixed to 4.25 km s$^{-1}$), and $\sigma_v$ is the local line width at each position within the disk. We assumed that the local line width is narrower than the spectral resolution (0.5 km s$^{-1}$), where 0.05 km s$^{-1}$ was used practically. The normalization of Equation (\ref{eq:line_profile_model}) is taken as the integration of $F_{r_i}(v)$ along the velocity axis being the spatial integration of $I_{r_i}$. The emission component at each radius $r_i$ is summed over $i$ to construct the disk-integrated spectra, i.e., 
\begin{equation}
    F(v) = \sum_i F_{r_i}(v),
\end{equation}
which is fitted to the observed disk-integrated spectra by varying each $I_{r_i}$ as free parameters. We used a regularly spaced grid in velocity space, which result in the unevenly spaced radial grid due to the non-linear relationship between Keplerian velocity and disk radius (see Equation \ref{eq:Keplerian_rotation}). The spacing of the grid is regulated by the spectral resolution: we used one grid point ($r_i$) per two velocity resolution components to obtain reasonable constraints on each $I_{r_i}$. This treatment resulted in a very sparse radial grid $r_i$ in larger radii, which makes it difficult to infer the radial distribution there. To mitigate this effect, we employed four fits with different radial grid created with different starting radii ($r_0 = 3, 5, 7, 9$\,au) and composed these four fits at the end to construct the final radial intensity profile. The actual fits were conducted by the MCMC algorithm implemented in the \texttt{emcee} package \citep{emcee}. The reconstructed intensity profiles are further convolved with the 1D Gaussian function with a 0\farcs1 FWHM to emulate the beam smearing effect and presented in Figure \ref{fig:radial_profile_from_line_profile}.      

\section{Spectroscopic data for molecules}\label{appendix:spectroscopic_data}
To employ the astronomical modeling of the observed spectra and estimate column density, spectroscopic data such as transition frequencies, intrinsic line strengths, and partition functions are essential information. They are basically retrieved from the Cologne
Database of Molecular Spectroscopy \citep[CDMS;][]{CDMS1, CDMS2, CDMS3} or the Jet Propulsion Laboratory (JPL) database \citep{JPL}. We describe the origin of these data and molecule-specific issues for each species below, with a particular care on the vibratinonal contributions to the partition functions. 

\subsection{Methanol}
Spectroscopic data for \methanol are taken from the CDMS, which are largely based on the theoretical calculation for the ground state ($v_t = 0$) and the first three torsional states ($v_t = 1, 2, 3$) by \citet{Xu2008}. The partition function of \methanol which considers the contributions from these states is also available in the CDMS. 

The data for the deuterated isotopologues, CH$_2$DOH, are taken from the JPL database, which is based on the laboratory experiment for the ground torsional state \citep{Pearson2012}. The partition function available in the JPL database only considers the ground torsional state. \citet{Taquet2019} estimated the vibrational correction factor of 1.15 for the partition function at 160~K based on the torsional dataset in \citet{Lauvergnat2009}. As the vibrational contributions are less dominant at lower temperatures, the correction factor for the partition function (and thus directly column density) should be smaller than $\sim1.1$ at the temperature of 100--110~K derived by the fit in the present work. This small contribution of the vibrational states does not significantly affect the overall results and discussions, in particular for the D/H ratios of \methanol. Therefore, we do not apply any specific correction factors to the partition function of CH$_2$DOH.  

\subsection{Acetaldehyde}
The data for \acetaldehyde are taken from the JPL database. This is based on the data listed in \citep{Kleiner1996} for the ground state ($v_t = 0$) and the first two torsional states ($v_t = 1, 2$). The partition function of \acetaldehyde available in the JPL database also considers these three torsional states. 

There are two isomers of $^{13}$C-acetaldehyde: $^{13}$CH$_3$CHO and CH$_3^{13}$CHO. The data of both isomers for the ground state ($v_t = 0$) and the first torsional state ($v_t = 1$) are available at the CDMS based on the laboratory measurement by \citep{Margules2015}. The partition function which considers the contributions from torsional states up to $v_t = 8$ is also available at the CDMS.

There are also two isomers of deuterated acetaldehyde: CH$_2$DCHO and CH$_3$CDO. While the data of the ground states ($v_t = 0$) and the first torsional states ($v_t = 1$) are available for CH$_3$CDO at the CDMS, only the data of the ground states ($v_t = 0$) exists for CH$_2$DCHO. These data are based on the laboratory experiments by \citet{Coudert2019}. The partition functions of these molecules are available at the CDMS. The partition function of CH$_2$CDO are calculated considering the contributions from the first three torsional modes and the approximate contributions from the lowest vibrational mode. The partition function for CH$_2$DCHO are, on the other hand, calculated considering the approximate contributions from the lowest torsional mode and the lowest vibrational mode.


\subsection{Methyl Formate}
We used the spectroscopic data for \methylformate taken from the JPL database. This data is based on the laboratory experiment for the ground states ($v_t = 0$) and the first torsional state ($v_t = 1$) by \citet{Ilyushin2009}. Similar to acetaldehyde, there are two $^{13}$C-isotopologues of methyl formate: $^{13}$CH$_3$OCHO and CH$_3$O$^{13}$CHO. While the data of the ground states ($v_t = 0$) and the first torsional state ($v_t = 1$) for CH$_3$O$^{13}$CHO are available at the CDMS, which is based on the laboratory measurements \citep[][and references therein]{Carvajal2010}, no CDMS entry for $^{13}$CH$_3$OCHO are found. Instead, we compiled the experimental data and theoretical predictions for $^{13}$CH$_3$OCHO transitions from literature \citep{Carvajal2009, Haykal2014, Favre2014}. 

For the partition function of these three molecules, we used the one calculated by \citet{Favre2014}. \citet{Favre2014} fully considered the contributions from torsional (up to $v_t=6$) and vibrational populations. The partition function data exist in the JPL database as well for the normal isotopologues (\methylformate), but only the contributions from $v_t = 0, 1$ states are considered. This resulted in a difference between the one by \citet{Favre2014} and the one in the JPL database by a factor of 1.2 at 150~K and 2.5 at 300~K \citep{Favre2014}. We note that there are difference in the treatment of the degeneracy, where \citet{Favre2014} ignore the nuclear spin degeneracy (i.e., $g_\mathrm{I} = 1$) while the CDMS and JPL entries take $g_\mathrm{I} = 2$. To take this difference into account, we multiply the partition function listed in \citet{Favre2014} by two in practice.

We also examined the data for deuterated methyl formate. Similarly, there are two isomers: CH$_2$DOCHO and CH$_3$OCDO. Spectroscopic data for both molecules are taken from the CDMS, which is based on the experimental measurement by \citet{Coudert2013} and \citet{Margules2009} for CH$_2$DOCHO, and \citet{Margules2010} and \citet{Duan2015} for CH$_3$OCDO. The partition functions of these molecules available in the CDMS only take the ground vibrational state into account. 
\textcolor{blue}{\citet{Manigand2019} uses the vibrational correction factor of 1.31 and 1.11 at 115~K for CH$_2$OCHO and CH$_3$OCDO, respectively, assuming that the correction factor is the same as $^{13}$C-isotopologues. We applied these factors to the partition functions of these molecules.}


\subsection{Dimethyl Ether}
The data for \dimethylether are taken from the CDMS, which is based on the experimental measurement by \citet{Endres2009} for the ground vibrational states ($v = 0$) with four substates (AA, AE, EA, and EE). The partition function is calculated considering the torsionally excited states ($v_{11} = 1$ and $v_{15} = 1$), C-O-C bending states ($v_7 = 1$), and $v_{11} + v_{15} = 2$, and available at the CDMS. 

The spectroscopic data for $^{13}$C isotopologue ($^{13}$\dimethylether) are also available at the CDMS and based on the laboratory experiments by \citet{Koerber2013}. The partition function in the CDMS is calculated only taking the ground vibrational states into account, and the vibrational correction factor is not yet available.   

\subsection{Acetone}
The data for \acetone are taken from the JPL. The JPL entry compiled the data from \citet{Peter1965}, \citet{Vacherand1986}, \citet{Oldag1992}, and \citet{Groder2002}. The partition function which takes into account the torsional and vibrational state are available at the JPL and used for the spectral fit. 

\subsection{Ethylene Oxide}
The data for \ethyleneoxide are taken from the CDMS. This CDMS entry is based on the experimental data in \citet{Creswell1974} and \citet{Muller2022}. The partition function available at the CDMS takes into account the ground vibrational state only, and the vibrational correction factor is not available.

\subsection{Propenal}
The data for \propenal are taken from the CDMS, which are largely based on the experimental data in \citet{Daly2015} with additional data from \citet{Winnewisser1975} and \citet{Cherniak1966}. The partition function is calculated by considering no vibrational states nor other conformers, and the vibrational and conformational correction factors are not yet available. We thus do not apply any correction factors to the partition function.

\subsection{Propanal}
The data for \propanal are taken from the CDMS, which are largely based on the experimental measurements by \citet{Zingsheim2017} with additional data from \citet{Hardy1982} and \citet{Demaison1987}. The partition function is calculated by considering no vibrational states nor other conformers, and the vibrational and conformational correction factors are not yet available. We thus do not apply any correction factors to the partition function.








%% For this sample we use BibTeX plus aasjournals.bst to generate the
%% the bibliography. The sample631.bib file was populated from ADS. To
%% get the citations to show in the compiled file do the following:
%%
%% pdflatex sample631.tex
%% bibtext sample631
%% pdflatex sample631.tex
%% pdflatex sample631.tex

\bibliography{sample631}{}
\bibliographystyle{aasjournal}

%% This command is needed to show the entire author+affiliation list when
%% the collaboration and author truncation commands are used.  It has to
%% go at the end of the manuscript.
%\allauthors

%% Include this line if you are using the \added, \replaced, \deleted
%% commands to see a summary list of all changes at the end of the article.
%\listofchanges

\end{document}

% End of file `sample631.tex'.
